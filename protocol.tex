\section{Protocols}
\subsection{Notations:} $A$ - Represents an attribute\\$dom(A)$ - Domain of attribute  $A$\\$s_A$ - Size of the dom($A$)\\$v_i, i \in [s_A]$-$dom(A)=\{v_1,v_2,\ldots, v_{s_A}\}$\\$ct_{A,i}$-\# records with value $v_i$ for attribute $A$\\m-\# number of data onwers\\$\mathcal{D}$- Encrypted database\\ $\mathcal{A}=\{\mathcal{A}_1,...\mathcal{A}_l\}$ - The set of attributes in the relational schema of $\mathcal{D}$\\
$x \times y$ table $T$ - A table  with $x$ rows/records and $y$ columns one for each  attribute, serves as one of the inputs to a transformation primitive
\\
$x' \times y'$ table $T'$ - A table  with $x'$ rows/records and $y'$ columns one for each  attribute, serves as the output of a transformation primitive\\
$\mathcal{E}(x)$- Represents the one-hot-coding for x\\
Boldface characterizes encrypted value\\
$\tilde{}$ characterizes that the data is in one-hot-coding\\
Throughout this section we will assume that there is an encryption
scheme $\mathcal{E}$ is a $4$-tuple $(Gen,Enc,Dec,Eval)$, which is
homomorphic with respect to all functions in a set $\mathcal{F}$.

\subsection{Parties}

There are three type of entities involved in the protocol: {\it
  analytics server (AS)}, {\it crypto-service provider (CSP)}, and
users. CSP runs the key-generation algorithm $Gen(\lambda)$ and 
obtains a pair of keys $(sk,pk)$. We assume that a secure mechanism 
is used to distribute $pk$ to  AS and the users. 

\subsection{Protocol 1 (Uses Homomorphic Encryption)}

\noindent
{\bf Step 1:} We assume that $U=\{ u_1, \cdots, u_n \}$ is the set of
users. Whenever a user $u_i$ needs to send a value $v$ to AS, he/she
sends the encrypted value $Enc_{pk} (v)$ to AS.

\noindent
{\bf Step 2:} Assume that AS has encryptions of some values
(i.e. $Enc_{pk}(v_1),\cdots,Enc_{pk}(v_m)$). AS generates a random
value $r$ according to some distribution (e.g. a random value
corresponding to the Laplace or Guassian distribution) and needs to
compute some function $G(f(v_1,\cdots,v_m),r)$ (we can think of $G$ as
DP version of the function $f$). For example, if we are computing the
sum of the values in a differentially private manner then
$f(v_1,\cdots,v_m) = \sum_{i=1}^m v_i$, $r$ is distributed according
to the Laplace distribution, and $G(x,y) = x+y$. Now assume that the
function $G(f(v_1,\cdots,v_m),r) \in \mathcal{F}$. Assuming that for
$1 \leq i \leq m$, $c_i = Enc_{pk}(v_i)$, AS computes
$Eval(G,c_1,\cdots,c_m,r)$ which yields $Enc_{pk}(G(f(m_1,\cdots,m_r),r))$.

\noindent
{\bf Step 3:} AS generates another random value $z$ (assuming that
addition is in the set of functions $\mathcal{F}$) and computes
$Enc_{pk}(G(f(m_1,\cdots,m_r),r)+z)$ and sends it to CSP. CSP decrypts
the value using the secret $sk$ and obtains $G(f(m_1,\cdots,m_r),r)+z$
and sends it to AS.  AS subtracts $z$ and obtains
$G(f(m_1,\cdots,m_r),r)$, which is the desired result.

\subsection{Protocol 2 (Uses SMC)}

If $G(f(v_1,\cdots,v_m),r) \not\in \mathcal{F}$, then we cannot simply
use homomorphic encryption and thus we use secure multiparty
computation (SMC).

{\it Inputs:} AS's input $i$ to the SMC protocol is
$Enc_{pk}(v_1),\cdots,Enc_{pk}(v_m)$ and a random variable $r$. CSP's
input $j$ is the secret key $sk$. The function $z (i,j)$ corresponds
to the following two steps:
\begin{itemize}
\item Decrypt $Enc_{pk}(v_1),\cdots,Enc_{pk}(v_m)$ using $sk$ and obtain
$v_1,\cdots,v_m$.
\item Compute $G(f(v_1,\cdots,v_m),r)$
\end{itemize}

{\it Output:} At the end of the protocol, AS receives the output of the
function $z(i,j)$ and CSP receives no output.

\am{Does the CSP get to see the plaintext of the original entries $v_1, \ldots, v_m$?}


