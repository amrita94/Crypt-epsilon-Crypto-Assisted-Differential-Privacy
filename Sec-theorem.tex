\section{\system Security Sketch}



In this section we provide a sketch of the security proof in the semi-honest model.
Our proof will follow the well established simulation argument \cite{Oded}. Let
$P(\mathcal{D})$ be the output of a program $P$ on dataset \am{should it be ... $P(\mathcal{D}, \epsilon)$ be the output of a differentially private program $P$? And this should also be a random variable?}
$\mathcal{D}$ and let $P^{CDP}(\mathcal{D},\epsilon)$ be the random
variable (rv) that corresponds to the output of the equivalent
\textsf{CDP}-version of $P$ using a privacy parameter $\epsilon$.

\begin{theorem}\label{thm:security}
\rm
Let $\Pi$ be the protocol corresponding to the execution of program $P$ in \system. The
views and outputs of \textsf{AS} and \textsf{CSP} are denoted follows:
\[
\begin{array}{cc}
View_1^{\Pi}(P,\mathcal{D},\epsilon) & Output_1^{\Pi}(P,\mathcal{D},\epsilon) \\
View_2^{\Pi}(P,\mathcal{D},\epsilon) & Output_2^{\Pi}(P,\mathcal{D},\epsilon) \\
\end{array}
\]
There exists Probabilistic Polynomial Time (PPT) simulators $Sim_1$
and $Sim_2$ such that:
\squishlist
\item $Sim_1 (P^{CDP}(\mathcal{D},\epsilon))$ is computationally indistinguishable ($\equiv_c$)
from $(View_1^{\Pi}(P,\mathcal{D},\epsilon),Output^{\Pi}(P,\mathcal{D},\epsilon))$, and
\item $Sim_2 (P^{CDP}(\mathcal{D},\epsilon))$ is $\equiv_c$
to $(View_2^{\Pi}(P,\mathcal{D},\epsilon),Output^{\Pi}(P,\mathcal{D},\epsilon))$.
\squishend
$Output^{\Pi}(P,\mathcal{D},\epsilon))$ is the combined output of the two
parties\footnote{Notice that the simulators are passed a random variable $P^{CDP}(\mathcal{D},\epsilon))$,
which essentially means that simulator is given a random draw from this distribution.}
\end{theorem}
The proof of this is presented in Appendix A. The statement of the theorem given above assumes
that \textsf{AS} and \textsf{CSP} do not collude with the users (the
data owners). However, if \textsf{AS} colludes with a subset of the
users, which essentially means $Sim_1$ ($Sim_2$)has to be given the data
corresponding to these users as additional parameter. This provides no
complications in the proof at all. \am{Rather than saying this sentence, can you cite a paper and say: This case can be handled by modifying the proof using arguments from \cite{??}.}  Similar argument goes for $Sim_2$. \am{drop this last line}


\am{add the following line here: Since every program $P$ expressible using \system primitives satisfies differential privacy, and it follows from Theorem~\ref{thm:security} that every execution of \system satisfies computational differential privacy}. 
\begin{corollary} 
	Protocol $\Pi$ satisfies computational differential privacy under the \textsf{SIM-CDP} notion \cite{CDP}.
\end{corollary}
