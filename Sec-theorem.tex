

\section{\system Security Sketch}


In this section we provide a sketch of the security proof in the
semi-honest model and provide a discussion on how to extend it to account for malicious adversaries.
\subsection{Semi-honest model}
The security proof for \system in the semi-honest follows the well established
simulation argument~\cite[Chapter 7]{Oded}. %We assume that the reader is familiar with standard concepts, such when two distributions are computationally indistinguishable ($\equiv_c$).
Let program $P$ be executed on a dataset $\mathcal{D}$
with privacy parameter $\epsilon$ and let
$P^{CDP}(\mathcal{D},\epsilon)$ denote the random variable
corresponding to the output of running $P$ in the \textsf{CDP} model. \noindent
\begin{theorem}\label{thm:security}
\rm
Let $\Pi$ be the protocol corresponding to the execution of program $P$ in \system. The
views and outputs of \textsf{AS} and \textsf{CSP} are denoted follows:
\[
\begin{array}{cc}
View_1^{\Pi}(P,\mathcal{D},\epsilon) & Output_1^{\Pi}(P,\mathcal{D},\epsilon) \\
View_2^{\Pi}(P,\mathcal{D},\epsilon) & Output_2^{\Pi}(P,\mathcal{D},\epsilon) \\
\end{array}
\]
There exists Probabilistic Polynomial Time (PPT) simulators $Sim_1$
and $Sim_2$ such that:
\squishlist
\item $Sim_1 (P^{CDP}(\mathcal{D},\epsilon))$ is computationally indistinguishable ($\equiv_c$)
from $(View_1^{\Pi}(P,\mathcal{D},\epsilon),Output^{\Pi}(P,\mathcal{D},\epsilon))$, and
\item $Sim_2 (P^{CDP}(\mathcal{D},\epsilon))$ is $\equiv_c$ to
  $(View_2^{\Pi}(P,\mathcal{D},\epsilon),Output^{\Pi}(P,\mathcal{D},\epsilon))$.
  \squishend $Output^{\Pi}(P,\mathcal{D},\epsilon))$ is the combined
  output of the two parties\footnote{Note that the simulators are
    passed a random variable $P^{CDP}(\mathcal{D},\epsilon))$, i.e., the simulator is given the ability to sample
    from this distribution.}
\end{theorem}
The proof of this is presented in Appendix \ref{sec:proof}. One of the main
ingredients of the proof is the composition theorem~\cite[Section
  7.3.1]{Oded}, which informally states: Suppose a protocol $\Pi_f^g$
implements functionality $f$ and uses function $g$ as an oracle (i.e.,
uses only input-output behavior of $g$).  Assume that $\Pi_g$ is a
protocol to implement $g$ and calls to $g$ in $\Pi_f^g$ are replaced
by instances of $\Pi_g$ (we refer to this as the composite
protocol). If $\Pi_f$ and $\Pi_g$ are correct (satisfy the aforementioned simulator
definition), then the composite protocol is
correct. Note that this means that the proof can be done in a modular
fashion as long as the underlying operators are
used in a blackbox manner (i.e. only the input-out behavior are used
and none of the internal state or messages are used).

Next we discuss some easy extensions to the theorem.  The statement in Theorem \ref{thm:security} assumes that \textsf{AS} and \textsf{CSP} do
not collude with the users (data owners). However, if \textsf{AS}
colludes with a subset of the users $U$ , which essentially means $Sim_1$
($Sim_2$) have to be given the data corresponding to users in $U$ as
additional parameters. This presents no complications in the proof at
all (see the proof in~\cite{LReg}). If a new user $u$ joins, their
data can be encrypted and simply added to the database. %Notice that because of the semantic-security property of the encryption scheme encryption of $d$ provides no information about encryptions of other data items. 
Since every program $P$
expressible using \system operators satisfies differential privacy,
it follows from Theorem~\ref{thm:security} that every execution of
\system satisfies computational DP.
 \noindent
\begin{corollary} 
	Protocol $\Pi$ satisfies computational differential privacy under the \textsf{SIM-CDP} notion \cite{CDP}.
\end{corollary}

\subsection{Malicious Model}
In \system, malicious adversarial behaviour can be characterized in two ways (1) Violating Integrity  (2) Violating Privacy 

\stitle{Malicious \AS: }The \AS has vested interest in learning the correct output and hence has no incentive of violating integrity (just like the receiver in the Private Set Intersection setting).
 However, it might try to violate privacy. In \system we have two measurement operators namely \textsf{Laplace} and \textsf{NoisyMax}. Since \textsf{NoisyMax} is implemented via a garbled circuit, the corresponding maliciously secure implementation can be obtained from standard approaches \cite{Wang:2017:AGE:3133956.3134053}. Hence, for the rest of the section, we limit ourselves to the \textsf{Laplace} operator. For this, the \AS might try to violate privacy by not adding noise to the cipher of the true answer before sending it to the \CSP. In this case since the \CSP is honest, \AS still gets $\epsilon$-DP output.
However, since the output can be public, the \CSP can compute the non-noisy answer.
Thus from the \AS's perspective it is \eat{gaining nothing itself but rather revealing the non-noisy answer to a third party \CSP - } a lose-lose situation for the \AS. Hence by a game theoretic argument, we observe that adding noise is the Nash equilibrium %(Table \ref{tab:Malicous_AS}) 
in this case.
Note that the case of adding less noise, (i.e. adding noise drawn from $Lap(\frac{1}{\epsilon'})$ while reporting a privacy budget of $\epsilon$, $\epsilon' > \epsilon$) is  similar to that of adding no noise.
Overall privacy budget violation is disallowed by the honest \CSP who monitors all the budget expenditures via the public ledger.
\eat{\begin{table}\caption{Gain Table for Malicious AS}
\begin{tabular}{|l|l|}
\hline
AS(M)\textbackslash CSP(H) & Add noise\\
\hline
 Don't add noise& $-\infty$    \\\hline
 Add noise & 0   \\
 \hline
\end{tabular}\label{tab:Malicous_AS}
\end{table}
}

\stitle{Malicious \CSP:}
As discussed in Section 3, the \CSP maintains a public ledger with the following information\\
(1) total privacy budget $\epsilon^B$ which is publicly known
\\(2) the privacy budget $\epsilon$ used up every time the \AS submits a cipher for decryption\\
Since the ledger is public, the \AS can verify whether the per program reported privacy budget is correct preventing any disparities in the privacy budget allocation.
For the other cases, a malicious \CSP can be accounted for in two ways --\\
(1) \stitle{Joint Laplace Noise Generation: } Instead of having  both the servers, \textsf{AS} and \textsf{CSP} add two separate instances of Laplace noise to the true answer, we can jointly compute a single instance of the Laplace noise in \system via a SMPC protocol ~\cite{Djoin,DworkOurData}. For this, the \CPS generates a garbled circuit that generates an instance of random noise, $\eta \sim Lap(\frac{1}{\epsilon})$ using the fundamental law of transformation of probabilities and returns $\boldsymbol{\eta}=labEnc_{pk}(\eta)$. 
The \AS can now simply add $\boldsymbol{\eta}$ to the cipher of the true answer and send it to the \CSP for decryption. A malicious \CSP could decrypt the ciphers incorrectly. However, this can be circumvented by standard techniques which involves the \CSP to efficiently prove the validity of the decryption in zero knowledge to the \AS.\\
(2) \stitle{Trusted Execution Environment (TEE): } Another approach can be removing a separate physical server for the \CSP altogether and instead capturing all its functionalities in a TEE \cite{Boneh2,Prochlo,Aïmeur2008}. The validity of each of \CSP's actions can be verified and attested to by the data owners in a TEE.
%\\Note that the violation of integrity by a malicious \CSP for the $genLabMult$ function can also be prevented by standard zero knowledge proofs.
%Hence, this approach adds just one instance of the Laplace noise, resulting the same accuracy guarantee as the \textsf{CDP} model. However owing to the garbled circuit, this implementation is computationally heavier. Therefore, we choose the two phase noise addition implementation for \system in this paper.For our \system setting, we assume that there is no harm in \CSP learning the final $\epsilon$-DP output but the \CSP has no dedicated use for the $\epsilon$-DP statistic (Section 3). Hence it has no incentive to cheat for the sake of getting a non-noisy output, i.e., violate privacy. The only scenario when the \CSP can be incentivized to cheat is when the \CSP has it in for the\ AS, i.e, it has some personal agenda in screwing up \AS's computations and wants the \AS to get incorrect output. Now, if the \CSP violates privacy, i.e., by not adding noise to the output, then again the \CSP still learns $\epsilon$-DP output but the \AS can now learn true answer. Given our above assumption, the \CSP thus has no incentive in aiding the \AS to learn the true output (Table \ref{tab:Malicous_CSP}). 
\begin{comment}\begin{table}\caption{Gain Table for Malicious \CSP}
\begin{tabular}{|l|l|}
\hline
\CSP(M)\textbackslash \AS(H) & Add noise\\
\hline
 Don't add noise& $-\infty$    \\\hline
 Add noise & 0   \\
 \hline
\end{tabular}\label{tab:Malicous_CSP}
\end{table}
However, it can violate the integrity by just decrypting wrong answer. For this we need to prove that the \CSP decrypts the given cipher correctly with a mask (which in this case is \CSP's instance of the $\epsilon$-DP noise) that is known only to the \CSP. This can be acheived via a zero-knwoledge-proof that works as follows-
\begin{enumerate}\item  \CSP samples noise $n$ from an appropriate Laplace distribution, and encrypts it $Enc(n)$. Let $x$ be the true cipher that the \AS wants to be decrypted.
The \CSP then commits to this value of $Enc(n)$ and send the commitment, $C_n$ to the \AS along with the correct answer $v_1=Dec(x)+n$. 
\item \AS  sends $c_1=x+Enc(r)$ to the \CSP. 
\item  \CSP returns $v_2=Dec(c_1)+n$ to the \AS and also opens its commitments to show that $C_n$ was indeed a commitment to $Enc(n)$ 
\item \AS now samples a random value $r'$ and sends $c_2= c1+Enc(r')+Enc(n)$ to the \CSP.
\item \CSP decrypts $c_2$ and send $v_3=Dec(c_2)$ to the \AS.
\item \AS accepts only iff $v_3-r'-r==v_1$
\end{enumerate}
The soundness and completeness of this protocol can be shown easily. The main point here is that due to $Enc(r)$ (Step 2), the challenge $c_1$ looks like an encryption of a random number for the \CSP. Hence there is no way the \CSP can cheat the \AS into accepting an incorrect answer. Also since only the initial commitment is done only to the encrypted value of the noise, the true noise added also still remains hidden from the \AS.
\end{comment}
