\section{Security Proof}
\begin{proof}
Recall any program in \system is a sequence of transformation primitives followed by either a $\textsf{Laplace}$ primitive or a \textsf{NoisyMax} primitive. First we prove for the case when program $P$ uses Laplace mechanism.

As discussed in section 4.2 and 4.3, the sensitivity of a \system program is computed as the product of the stabilities of the transformation primitives used in it.
Let the equivalent \system program for $P$ be denoted as $P_{Crypt\epsilon}$ and $stab(\mathcal{T})$ denote the stability of transformation $\mathcal{T}$. Let $\bar{\mathcal{T}}(\boldsymbol{\tilde{\mathcal{D}}})\mathcal{T}_l(\cdots\mathcal{T}_2(\mathcal{T}_1(\mathcal{D})))$ denote the sequence of transformation primitives in $P_{Crypt\epsilon}$. Thus the sensitivity of $P$ is given by $\Delta=\prod_{i=1}^l stab(\mathcal{T}_i)$.\\
\textbf{Correctness:}
 Using the homomorphic properties of the underlying encryption
scheme, it easy to verify that at the end of performing all the transformation primitives the \textsf{AS} knows $labEnc_{pk}(P(\mathcal{D})+\eta_1, \eta_1 \sim Lap(\frac{\Delta}{\epsilon})$.  Now clearly in the \textsf{Laplace} primitive, by the correctness of the decryption function, \textsf{CSP} can compute $\mathcal{F}_2(P,\mathcal{D},\epsilon)=P(\mathcal{D})+\eta_1$. Also trivially, \textsf{AS} can compute $\mathcal{F}_1(P,\mathcal{D},\epsilon)=P(\mathcal{D})+\eta_1$ from the message $P(\mathcal{D})+\eta_1+\eta_2$ it receives from the \textsf{CSP}. This completes our proof of correctness. \\
\textbf{Privacy:}
Let simulators $Sim_1$ and $Sim_2$ simulate the view of ${Adv}_1$ and $Adv_2$ respectively. 
Let $l$ be the bit-length of each data record represented in per-attribute one-hot-coding. \\
First we prove for the case when $P$ uses Laplace mechanism.
$\mathcal{S}_1=Sim_1(\{D_i, i \in S\}, P,\Delta=1,\epsilon, P(\mathcal{D})+\eta_2,P(\mathcal{D})+\eta_1)$
\begin{enumerate}\item Run $Gen(\kappa)\mapsto (pk,sk)$ 
\item Draw $\eta' \in Lap(\frac{1}{\epsilon})$
\item For all $i \in \{1,...,m\}$ if $i \in S$ then encrypt $D_i$. 
Otherwise compute $labEnc_{pk}(D_i)$ as a random $l$-lengthed vector $r_i \in \mathcal{C}^l$. [We abuse the notation slightly here, $labEnc_pk(D_i)$ denotes component wise encryption of the record represented in per-attribute one-hot-coding]
\item Output $\Big(\{D_i, i \in S\},P,\Delta=1,\epsilon,\eta',\encD',P(\mathcal{D})+\eta_2+\eta'\Big)$ where $\encD'=\{labEnc_{pk}(D_i)\},i \in \{1,...,m\} i \not \in S$ as computed in step 3.
\end{enumerate}
The R.H.S of eq \ref{adv1}  is given by 
\begin{gather}\Big(\{D_i, i \in S\},P,\Delta=1,\epsilon, \eta_1,\encD,P(\mathcal{D})+\eta_2+\eta_1\Big)\end{gather}
It follows from the semantic security of the encryption scheme that $\encD'$ and $\encD'$ are computationally indistinguishable.
Since, $\eta'$ is drawn from the same distribution as $\eta_1$, $P(\mathcal{D})+\eta'+\eta_2$ and $P(\mathcal{D})+\eta_1+\eta_2$ have the exact same distribution. This proves that the distribution of simulation output is computationally indistinguishable from that of the views of the corrupted parties in $Adv_2$
output has the same distribution as that of the views of the corrupted parties in $Adv_1$ in the
protocol $\Pi$.\\
$\mathcal{S}_2=Sim_2(\{D_i, i \in S\},P,\Delta=1,\epsilon,P(\mathcal{D})+\eta_1,P(\mathcal{D})+\eta_2)$
\begin{enumerate}\item Run $Gen(\kappa)\mapsto (pk,sk)$ \item Generate $\eta'' \in Lap(\frac{1}{\epsilon})$ \item Output $\Big\{D_i, i \in S\},P,\Delta=1,\epsilon, \eta'', labEnc_{pk}(P(\mathcal{D})+\eta_1), P(\mathcal{D})+\eta_1+\eta''\Big)$
\end{enumerate}
The R.H.S of eq \ref{adv2} is given by \begin{gather}\Big(\{D_i, i \in S\},P,\Delta=1,\epsilon,\eta_2,labEnc_{pk}(P(\mathcal{D})+\eta_1),P(\mathcal{D})+\eta_1+\eta_2\Big)\end{gather}
Again, clearly the simulation output has the same distribution of the
views of the corrupted parties in $Adv_2$ in the protocol $\Pi$. 
The proof for the case of NoisyMax is inherent from the correctness and semantic security of garbled circuits.
\end{proof}
