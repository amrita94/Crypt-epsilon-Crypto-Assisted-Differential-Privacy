%
% The first command in your LaTeX source must be the \documentclass command.
\documentclass[sigconf]{acmart}

%
% defining the \BibTeX command - from Oren Patashnik's original BibTeX documentation.
\def\BibTeX{{\rm B\kern-.05em{\sc i\kern-.025em b}\kern-.08emT\kern-.1667em\lower.7ex\hbox{E}\kern-.125emX}}
    
% Rights management information. 
% This information is sent to you when you complete the rights form.
% These commands have SAMPLE values in them; it is your responsibility as an author to replace
% the commands and values with those provided to you when you complete the rights form.
%
% These commands are for a PROCEEDINGS abstract or paper.
%\copyrightyear{2018}
\acmYear{2020}
%\settopmatter{printacmref=false}
%\setcopyright{acmlicensed}
%\acmConference[Woodstock '18]{Woodstock '18: ACM Symposium on Neural Gaze Detection}{June 03--05, 2018}{Woodstock, NY}
%\acmBooktitle{Woodstock '18: ACM Symposium on Neural Gaze Detection, June 03--05, 2018, Woodstock, NY}
%\acmPrice{15.00}
%\acmDOI{10.1145/1122445.1122456}
%\acmISBN{978-1-4503-9999-9/18/06}
%\usepackage{cite}
%\usepackage[bottom]{footmisc}
\usepackage{enumitem}
\usepackage{amsmath,amssymb,amsfonts}
\usepackage{algorithm}
%\usepackage{hyperref}
%\usepackage{algorithmic}
\usepackage{algcompatible}
\usepackage{graphicx}
\usepackage{comment}
\usepackage{multirow} 
\usepackage{amsthm}
\usepackage{textcomp}
\usepackage{xcolor}
\def\BibTeX{{\rm B\kern-.05em{\sc i\kern-.025em b}\kern-.08em
    T\kern-.1667em\lower.7ex\hbox{E}\kern-.125emX}}
\usepackage{amsmath}

\usepackage{xspace}
\usepackage{caption}
%\captionsetup[figure]{font=small,labelfont=small}
\usepackage{subcaption}
%\captionsetup[subfigure]{font=scriptsize,labelfont=scriptsize}
%\usepackage[color]{changebar}
%\cbcolor{red}
\newtheorem{theorem}{Theorem}
\newtheorem{definition}{Definition}
\newtheorem{exmp}{Example}[section]
\usepackage{mathtools}
\DeclarePairedDelimiter\ceil{\lceil}{\rceil}
\DeclarePairedDelimiter\floor{\lfloor}{\rfloor}
\newcommand\numberthis{\addtocounter{equation}{1}\tag{\theequation}}
%\DeclarePairedDelimiter\ceil{\lceil}{\rceil}
\usepackage{xspace}


\newcommand{\eat}[1]{}
\newcommand{\am}[1]{{\color{blue} \emph{[[AM: #1]]}}}
\newcommand{\arc}[1]{{\color{blue} \emph{[[ARC: #1]]}}}
\newcommand{\xh}[1]{{\color{purple} \emph{[[XH: #1]]}}}


\newcommand{\stitle}[1]{\vspace{.33em}\noindent\textbf{#1}~}

\newcommand{\system}{Crypt$\epsilon$\xspace}

\newcommand{\encD}{\boldsymbol{\tilde{D}}}
\newcommand{\crossproduct}{\times}
\newcommand{\project}{\pi}
\newcommand{\filter}{\sigma}
\newcommand{\countagg}{count}
\newcommand{\groupbystar}{\gamma^{*,count}}
\newcommand{\groupby}{\gamma^{count}}
\newcommand{\countdistinct}{count^*}
\newcommand{\encT}{\boldsymbol{\tilde{T}}}
\newcommand{\encB}{\boldsymbol{B}}
\newcommand{\encC}{\boldsymbol{C}}
\newcommand{\encV}{\boldsymbol{V}}
\newcommand{\lap}{Lap}
\newcommand{\noisymax}{NoisyMax}
\hypersetup{draft}

\begin{document}
\title{\system: Crypto-Assisted Differential Privacy on Untrusted Servers}
\author{}
\begin{abstract}
Differential privacy (DP) has steadily become the de-facto standard for achieving privacy in data analysis. It is typically implemented either in the ``central'' or ``local'' model. %In the former, a trusted centralized server collects the records \textit{in the clear} from the data owners and outputs differentially private statistics; while in the latter, the data owners individually randomize their inputs to ensure differential privacy.  
The local model has been more popular for commercial deployments as it does not require a trusted data collector (unlike the central model). This increased privacy, however, comes at the cost of strictly lower accuracy and restricted algorithmic expressibility as compared to the central model. 

In this work, we propose, \system, a  system and programming framework that (1) achieves the accuracy guarantees and algorithmic expressibility of the central model (2) without any trusted data collector like in the local model. \system achieves the ``best of both worlds'' by employing two non-colluding untrusted servers that run DP programs on encrypted data from the data owners. Although straightforward implementations of DP programs using secure computation tools can achieve the above goal theoretically, in practice they are beset with many challenges like poor performance, tricky security proofs etc. To this end, \system allows data analysts to author logical DP programs that are automatically translated to secure protocols that work on encrypted data. These protocols ensure that the untrusted servers learn nothing more than the noisy outputs, thereby guaranteeing $\epsilon$-DP for all \system programs.  \system
supports a rich class of DP programs that can be expressed via a small set of transformation and measurement operators followed by arbitrary post-processing. Further, we propose performance optimizations leveraging on the fact that the output is noisy. We demonstrate \system's feasibility for practical DP analysis with extensive empirical evaluations on real datasets. 
\end{abstract}
\eat{\begin{CCSXML}
<ccs2012>
<concept>
<concept_id>10002978.10002979.10002981.10011745</concept_id>
<concept_desc>Security and privacy~Public key encryption</concept_desc>
<concept_significance>300</concept_significance>
</concept>
<concept>
<concept_id>10002978.10002991.10002995</concept_id>
<concept_desc>Security and privacy~Privacy-preserving protocols</concept_desc>
<concept_significance>300</concept_significance>
</concept>
</ccs2012>
\end{CCSXML}
\ccsdesc[300]{Security and privacy~Public key encryption}
\ccsdesc[300]{Security and privacy~Privacy-preserving protocols}}
%\begin{IEEEkeywords}
\keywords{Crypto Service Provider, LHE, Two Server Model}
 \settopmatter{printfolios=true}
\maketitle
%\end{IEEEkeywords}
\section{Introduction}


There is a growing need for releasing aggregate properties from sensitive datasets in several domains  including social science, healthcare, and advertising. Differentially private algorithms \cite{dwork}, whose outputs are insensitive to adding or removing a single row in the input dataset, have become the gold standard for these situations. Differential privacy provides a provable and persuasive guarantee of privacy to individuals in a dataset, and has seen adoption by government \cite{machanavajjhala08onthemap,Vilhuber17Proceedings} and commercial organizations \cite{Rappor1,Apple, Samsung}. %It is defined with respect to a privacy parameter $\epsilon > 0$ where lower the value of $\epsilon$, greater is the privacy guaranteed.

Differential privacy (DP) is typically implemented in one of two models -- \textit{centralized differential privacy}, or \cdp, and \textit{local differential privacy}, or \ldp. In \cdp, data from individuals are collected and stored \textit{in the clear} in a \textit{trusted} centralized data curator. The trusted curator executes DP programs on the sensitive data  and releases outputs to a mistrustful data analyst. %A canonical algorithm in \cdp is the Laplace mechanism where the curator releases the output of a function $f$ by adding noise drawn from the Laplace distribution to hide the presence or absence of one row in the input database. 

%Depending on infrastructural constraints, like the viable trust model in a given setting, differential private algorithms in practice can have two different styles of implementation.  The more common and historically precedent implementation is the central differential privacy (\textsf{CDP}) model where a trusted data curator collates data from all individuals into a centrally held dataset and processes it in a privacy preserving way. %For example, the data curator could publish differentially private statistics of the data, that allows analysis on the data, without revealing individual information. 
%The curator is trusted to store the data \emph{in the clear} and mediates upon queries posed by a mistrustful analyst, interested in learning some synopsis of this dataset. Privacy is enforced by the curator by adding uncertainty to the answers for analyst's queries before releasing them. The other competing model of implementation, is the local differential privacy model (\textsf{LDP}) where the central data aggregating server is untrusted. Thus every data owner has to individually randomize his/her input before communicating it to the central aggregator. Hence the private data is concealed from the untrusted aggregator who attempts to infer statistics about the true population from the perturbed data instead. 

In \ldp, there is no trusted centralized data curator. Rather, each individual perturbs their own data using a (local) differentially private algorithm. The data analyst has direct access to these perturbed data, and uses it to infer aggregate statistics of the datasets. %A canonical \ldp algorithm is the randomized response mechanism \cite{RR} where each data owner flips some of his/her input bits based on a coin toss to provide \emph{plausible deniability} \cite{Dork}.


Recent deployments of DP in commercial organizations \cite{Rappor1, Apple} have preferred  \ldp over \cdp. \cdp's assumption of a trusted server that stores data in the clear is ill-suited for many applications as it constitutes a single point of failure for data breaches, and saddles the trusted curator with legal and ethical obligations to keep the user data secure. For instance, Google Chrome uses the \ldp model rather than \cdp to detect changes in browser properties of its userbase as it does not want the legal burden of storing highly sensitive browser fingerprints in the clear on its servers \cite{Rappor1}. 
  

However, \ldp's attractive security properties come with a utilitarian price tag. All DP algorithms ensure privacy by introducing noise into the computation. Under the \cdp model, one can release an aggregate count over the dataset having $n$ rows with an expected additive error of at most $\Theta(1/\epsilon)$ and ensure $\epsilon$-DP (e.g., using the Laplace mechanism \cite{dwork}). In contrast, under the \ldp model, at least $\Omega(\sqrt{n}/\epsilon)$ additive error in expectation must be incurred by any $\epsilon$-DP program for this task \cite{error1,error2,error3}, owing to the individual coin tosses of each data owner \cite{Prochlo,Rappor1,Rappor2,LDP1}. %This is because, each individual perturbs their data values independently in the \ldp model, contributing to the increased noise . 

There has been a growing line of work \cite{}{} aimed at 
Thus, on a dataset with a billion individuals, properties that are common to a population as large as 30,000 individuals can be missed under \ldp. 

The \ldp model imposes additional penalties in terms of the  algorithmic expressibility.  Kasiviswanathan et al. in \cite{Kasivi} showed that the power of \ldp is equivalent to that of the statistical query model \cite{SQ1} from learning theory and there exists an exponential separation between the accuracy and sample complexity of local and centralized DP algorithms.  As a consequence, \ldp requires enormous amounts of data to obtain reliable population statistics. Unsurprisingly, only large corporations  like Google \cite{Rappor1,Rappor2,Prochlo} and Apple \cite{Apple} have attempted deploying \ldp.
 
\eat{  Specifically, computation in \textsf{LDP} results in an additional error of $\Omega(\sqrt{n})$ where $n$ is the total number of data owners contributing to the noisy estimate \cite{error1,error2,error3}. In contrast, we get a constant error bound for \textsf{CDP}.
For e.g., for a single counting query, in the \textsf{CDP} model, the trusted data curator first computes the true count in the clear. Next, adding a single instance of noise from the distribution $Lap(\frac{1}{\epsilon})$ to this true count guarantees $\epsilon$-differential privacy. Since the s.t.d of the distribution $Lap(\frac{1}{\epsilon})$ is given by $\frac{1}{\epsilon}$, we get an expected error of $O(\frac{1}{\epsilon})$. On the other hand, owing to the independent coin flips of each reporting data owner, the resulting noise in \textsf{LDP} induces a binomial distribution. This binomial distribution can be approximated by a normal distribution; the magnitude of this random Gaussian
noise however can be very large, its
standard deviation grows in proportion to the square root of
the report count $ \Omega\big(\frac{\sqrt{n}}{\epsilon}\big)$, and the noise is in practice higher by an
order of magnitude \cite{Prochlo,Rappor1,Rappor2,LDP1}. %often growing with the data dimension

 Thus, if a billion individuals'
reports are analyzed, then a common signal from even
up to a million reports may be missed. 


The construct of the \textsf{LDP} model in fact imposes additional penalties in terms of the  algorithmic expressibility.  Kasiviswanathan et al. in \cite{Kasivi} showed that the power of \textsf{LDP} is equivalent to that of the statistical query model \cite{SQ1} from learning theory and there exists an exponential separation between the accuracy and sample complexity of local and central algorithms.  As a consequence, \textsf{LDP} requires enormous amounts of data \cite{Kasivi}
to obtain reliable population statistics. Unsurprisingly, only large corporations  like Google \cite{Rappor1,Rappor2,Prochlo} and Apple \cite{Apple}, with  billions of user base have had successful commercial deployment of \textsf{LDP}.
} %More abstractly, the power of the local model is equivalent tothe statistical query model from learning theory and t.

Conceptually, the aforementioned goal can be achieved by implementing the required differentially private mechanism using off-the-self secure-multiparty computation tools like \cite{EMP,MPCtools,ScaleMAMBA,ABP}. However this naive approach is rife with challenges when it comes to real-world deployment. \textit{Firstly}, the performance of any naive implementation can be prohibitively costly especially with large of data owners. \textit{Secondly}, 
\textit{Secondly}, 
In this paper, we strive to bridge the gap between \ldp and \cdp. We propose, \system, a new model for differential privacy that: 
\squishlist
\item never stores or computes on sensitive data in the clear, and still
\item achieves the accuracy guarantees and algorithmic expressibility of the \cdp model \squishend 
In \system a single trusted data curator is replaced by a pair of untrusted but non-colluding servers -- the Analytics Server (\textsf{AS}) and the Cryptographic Service Provider (\textsf{CSP}). The \textsf{AS} plays the role of  data curator in \cdp in executing the differentially private programs, but on \textit{encrypted} data records. The \textsf{CSP} initializes and manages the cryptographic primitives, and collaborates with the \textsf{AS} when the DP program needs to generate outputs. Under the assumption that the \textsf{AS} and the \textsf{CSP} are semi-honest and do not collude (a common assumption in the cryptography-assisted computation literature \cite{Boneh1,Boneh2,Ridge2,Matrix2,secureML,LReg,Ver}), \system is designed to reveal no extra information beyond what can be learned by the release of the outputs of a differentially private program. This is achieved using cryptographic primitives like linear homomorphic encryption and Yao's garbled circuits. 

\system permits analysts to author logical programs just like in the \cdp model. Like in prior work \cite{PINQ, FWPINQ, ektelo}, every logical program is guaranteed to satisfy differential privacy (in the \cdp model) by restricting access to the sensitive data via data transformations operators (inspired by relational algebra) and differentially private measurement operators ( Laplace mechanism and Noisy-Max \cite{Dork}). Programs can have constructs like looping and conditionals, and can arbitrarily post-process outputs of measurement operators. Such logical programs have been shown to express state-of-the art DP algorithms under the \cdp model. \system compiles these logical programs into \system protocols that can work on the encrypted data on the \textsf{AS} and \textsf{CSP}. 


The main contributions of this work are
\squishlist
\item \textbf{New Approach:} We present the design and implementation of \system, a novel system for executing DP programs over encrypted data on two non-colluding untrusted servers. %\system integrates the constant error bounds of \textsf{CDP} with the low trust assumption of \textsf{LDP}.
\item \textbf{Algorithm Expressibility:} \system supports data analysis via a rich class of state-of -the-art differentially private programs expressed in terms of a small set of transformation and measurement operators. Thus, \system achieves the accuracy guarantees of the \cdp model without the need for a trusted curator.  
\item \textbf{Performance Optimizations:} Existing techniques to compile logical \system programs into cryptographic protocols often result in inefficient programs. We present optimizations that speed up computation on encrypted data by at least an order of magnitude. A novel contribution of this work are DP indexing optimizations that leverage on the fact that differentially private programs can reveal statistical information about the data. 
\item \textbf{Practical for Real World Usage} We demonstrate the accuracy and practical efficiency of \system via extensive  evaluation. For the same tasks, \system programs achieve accuracy comparable to \textsf{CDP} and orders of magnitude (at least 2) more accuracy than that of \textsf{LDP}. Moreover, \system is efficient and runs within 5 min for a large class of programs on a dataset with 32,561 rows and 4 attributes. 
\item \textbf{Generalized Multiplication Using Linear Homomorphic encryption:} Our implementations leverages on an efficient method of performing $n$-way multiplications using linear homomorphic encryptions, which maybe of independent interest.
\squishend

\section{\system Overview}
\begin{figure}[t]
	\includegraphics[width=0.7\columnwidth]{CryptE_Diag_New.pdf}\vspace{-5mm}
	\caption{\label{fig:system} Crypt$\epsilon$ System}% Setting: The  \textsf{AS} runs the Crypt$\epsilon$ programs. The \textsf{CSP} manages the cryptographic primitves. }
\end{figure}

\subsection{System Architecture}
Here we briefly review the system architecture of \system (illustrated by Fig \ref{fig:system}). %A set of \textit{data owners} ${\textsf{DO}_i, i\in [m]}$ have private data records ${D_i, i \in [m]}$. \system permits data analysts to author and execute programs $P$ that satisfy DP under the \cdp model without storing or computing on the private data records in the clear. 
At the very outset, the %\textit{Cryptographic Service Provider},
\CSP records the total privacy budget, $\epsilon^B$, (provided by the data owners) and generates the key pair $(pk,sk)$ (details in Section \ref{sec:background}) for the encryption scheme. The data owners, ${\textsf{DO}_i, i\in [m]}$, use the public key (pk) to encrypt their data records, ${D_i, i \in [m]}$, in the appropriate representation (per attribute one-hot-encoding, details in Section \ref{sec:overview}) and send the encrypted records, $\boldsymbol{\tilde{D_i}},i \in [m]$, to the \AS
which aggregates them into a single encrypted database $\encD$. Next, the \AS inputs logical programs from the data analysts and translates them to \system's underlying implementation specific secure computation protocols that work on $\encD$.  A \system program typically consists of a sequence of transformation operators followed by a measurement operator. A majority of the transformations are executed wholly at the \textsf{AS}.%The transformation operators are designed in such a way that the \AS can perform majority of the associated computation on its own. 
 However, every measurement operator requires an interaction with the \textsf{CSP} as it requires (a) decryption of the answer, and (b) a check that %the noise is scaled by the correct sensitivity (verified from the \system program) and that 
the total privacy budget, $\epsilon^B$, is not exceeded. In this way, the \AS and the \CSP can compute the output of a \system program with the data owners being offline during the  program execution. 
\vspace{-0.5cm}
\subsection{\system Design Principles}\label{sec:design}
\stitle{Minimal Trust Assumptions:} As mentioned above, the overarching goal of \system is to mimic the \cdp model but without a trusted server. A natural solution for dispensing with the trust assumption of the \cdp model is via cryptographic primitives \cite{Prochlo,mixnets,amplification,Shi,Shi2,kamara,Rastogi,DworkOurData,BeimelSFE+DP}. Hence, to accommodate the use of cryptographic primitives, we assume a computationally bounded adversary in \system. However, a generic $m+1$ party SMPC would be computationally unwieldy. This necessitates a third party entity that can capture the requisite secure computation functionality in (at least) a 2-party protocol instead. %Moreover, since the data owners are no longer in the loop to monitor every query answering, the aforementioned entity should also watchdog the overall privacy budget expenditure.
This role is essayed by the \CSP in \system. For this two-server model, we assume semi-honest behaviour and non-collusion. This is a very common assumption for privacy preserving computations in the two-server model \cite{Boneh1,Boneh2,Ridge2,Matrix2,secureML,LReg,Ver} and can be imposed in practice via legal bindings.

\stitle{Programming Framework:}
Conceptually, the aforementioned goal of achieving the \textit{best of both worlds} can be obtained by implementing the required DP program using off-the-self secure multi-party computation (SMPC) tools like \cite{EMP,MPCtools,ScaleMAMBA,ABY}. %This notion is similar in spirit with the works in \cite{DworkOurData,BeimelSFE+DP}. 
However, when it comes to real world deployment, \system trumps such approaches due to the following reasons.
 
 \textit{Firstly}, without the support of a programming framework like that of \system, the implementation of every DP program (corresponding to different data queries) has to be ad-hoc and from scratch resulting in high turnaround time. Moreover, the problem is compounded in our setting as executing DP programs on untrusted servers comes with additional challenges like tracking the  \textit{sensitivity} of each DP operation, monitoring of the  total privacy budget expenditure across all programs et al. %For example, revisiting our above mentioned database schema, let us assume that the data analyst is additionally interested in learning the output of a second query "Count the number of age values having at least 100 records". Typically, both of these queries will require two separate garbled circuits. 
Thus, in a practical query answering set-up, this is feasible only if %to be able to answer multiple queries on-the-fly
the data analyst is well-versed in both DP and SMPC techniques. In contrast, \system's programming framework empowers the data analyst
with an expressive language; consequently, for any program (supported by \system's language) in \system, the only task for an analyst is to encode the DP program logic in terms of the \system operators.  \system abstracts out all the low-level implementation details like the choice of input data representation, translating queries to that representation, choice of SMPC operators, key management, privacy budget monitoring etc
from the analyst thereby reducing his/her burden of complex decision making. Thus every program in \system is automatically translated to protocols corresponding to the underlying implementation. %In principle, any function can be computed on encrypted data via secure computation, and thus \system can support any differentially private algorithm. However, we currently limit the expressibility of the programs supported in \system to those that operate on the sensitive data with efficiently implementable operators, and whose privacy can be easily tracked by the \textsf{CSP}. Nevertheless, as shown later in the paper, \system can already support a rich class of state-of-the-art DP programs.%In fact, owing to this separation of the logical and physical layers, \system's underlying implementation can be based on cryptographic operators of choice. As mentioned above, in this paper we present a prototype \system that uses LHE and garbled circuits.

\textit{Secondly}, computation of naive SMPC protocols can be prohibitively costly for practical usage. Thus, the SMPC protocols corresponding to the different DP programs, have to be individually fine-tuned for performance improvement. In \system, on the other hand, %any optimization to an operator will be inherited by all programs using it. Thus 
ensuring optimal performance for just the small set of \system operators translates into practical efficiency for all \system programs. 

\textit{Thirdly}, a DP program can be typically divided into segments that perform noisy measurements of private data followed by segments of post-processing operations that can be performed in the clear. Thus, in order to implement a given DP program using SMPC operators, the data analyst has to first segregate the operations on the basis of whether it falls inside or outside the trust boundary. Otherwise, one has to implement the entire DP program inside a single protocol which will degrade performance. For  example, in the paper \cite{AHP}, the authors present an $\epsilon$-DP algorithm for releasing a histogram that works as follows. First, a preliminary estimate of the histogram, $\hat{H}$, is computed using budget $\epsilon_1$. This is followed by post-processing steps of thresholding, sorting and clustering $\hat{H}$ based on which the final histogram, $\tilde{H}$, is computed with the remaining privacy budget $\epsilon-\epsilon_1$. Implementing this entire algorithm in a single SMPC protocol would be extremely computation heavy; an implementation in EMP toolkit \cite{EMP} takes 810s for a dataset of size $\approx 33,000$.  On the other hand,
in \system the access to the sensitive data is limited via two measurement operators thereby making the separation between the trust boundaries in a program explicit. For the above example program, \system operators allow the computation of the two histograms, $\hat{H}$ and $\tilde{H}$ in a secure (private) fashion while the intermediate steps can be processed in the clear. A \system program for this takes 238s ($3.4\times$ less time than the EMP implementation)

\textit{Fourthly}, the security (privacy) proofs for just stand-alone cryptographic and DP mechanisms can be notoriously tricky \cite{BellareCryptoError,DPSVTProof}. Combining the two thus exacerbates the technical complexity, making the design vulnerable to faulty proofs \cite{He:2017:CDP}. For example, given any arbitrary DP program written under the \cdp model, the distinction between intermediate results that can be released beyond the trust boundary and the ones which has to be kept private is often ambiguous. An instance of this is observed in the Noisy-Max algorithm, where the array of intermediate noisy counts cannot be released.  However, note that these intermediate noisy counts in fact correspond to valid query responses. Thus, an incautious analyst, in a bid to improve performance, might reuse a previously released noisy count query output for a subsequent execution of the Noisy-Max algorithm leading to privacy leakage. In contrast, \system is designed to reveal nothing other than the outputs of the DP programs to the untrusted servers; every \system program comes with an automatic proof of security (privacy). Referring back to the aforementioned example, in \system, the Noisy-Max algorithm is implemented as a secure measurement operator thereby preventing any accidental privacy leakage. 


\stitle{Data Owners are Offline:}
 Recall, \system's goal is to mimic the \cdp model with untrusted  servers. Hence, it is designed so that the data owners are offline after submitting their encrypted records to the \textsf{AS}. %This is beneficial as the \textsf{AS} does not need to maintain communication channels with the data owners over a long period of time. 
 If the data owners were online, the efficiency of some programs could be improved as some of the computation %from the \textsf{AS} and \textsf{CSP} 
 can be offloaded to the data owners.



\stitle{Low burden on CSP:} \system views the \textsf{AS} as an extension of the analyst, and that it has a vested interest in obtaining the result of the computation.   Thus in \system, we require the \textsf{AS} to  shoulder the major chunk of the workload for any \system program execution; interactions with the \textsf{CSP} should be minimal and only related to data decryption. Keeping this in  mind, we design the \textsf{AS} to perform most of the data transformations by itself (Table \ref{perf}). Specifically for every Crypt$\epsilon$ program, the \textsf{AS} processes the whole database and transforms it into concise representations (like an encrypted scalar or a short vector) which is then decrypted with the help of the \textsf{CSP}. An example setting in the real world can be when Google assumes the role of the \AS and Symantec can provide the services of the \CSP. It is interesting to note that we could have had an alternative implementation for Crypt$\epsilon$ where the private database is equally shared between the two servers and they engage in a secret share based SMPC protocol for computing the DP answers. However, this would require both the servers to do almost equal amount of work for every program. Such an arrangement would be justified only if both the servers are equally invested in learning the DP statistics and is ill-suited for Crypt$\epsilon$. A real world analogy for this can be when Google and Baidu decide to compute some statistics on their combined user bases. We speculate that the above scenario is much less likely than the other setting. Additionally, oblivious transfer (OT) is the major bottleneck for most SMPC implementations \cite{OT:bottleneck:1,OT:bottleneck:2,OT:bottleneck:3} and in a secret share based computation, each AND (equivalently multiplication) gate involves an OT step. Thus a secret share based implementation would be much more communication intensive resulting in a performance hit. 

\stitle{Separation of logical programming framework and underlying physical implementation:} The programming framework is independent from the underlying cryptographic operator specific implementation. This allows certain flexibility in the choice for lower level physical implementation. 
 For example, the prototype \system presented in this paper, uses per attribute one-hot-encoding as the input data representation. However, one can easily switch over to any other encoding scheme like multi-attribute one-hot-encoding, range based encoding etc. %Quite evidently, different encoding scheme will have different storage requirements. For instance, a k-attribute one-hot-encoding will be of size $d^k$ where $d$ is the domain size of each attribute. We choose one-hot-coding for our prototype \system implementation because it is a general representation that can be easily translated to other encoding schemes.  
 Similarly, although in this paper, we use $\epsilon$-DP (pure DP) for our privacy analysis, \system can be easily extended to accommodate other DP notions like $(\epsilon,\delta)$-DP, R\`enyi differential privacy (RDP) \cite{RDP} etc. For example, designing an operator for adding Gaussian noise would suffice to allow RDP analysis in \system. As discussed above, due to our design choice of having low burden on the \CSP, we implement \system using LHE and garbled circuits. However translation to an implementation based on other cryptographic operators is straighforward. For example, the optimized HE scheme in \cite{Blatt2019OptimizedHE} can be used in place of LHE. Similarly, garbled circuits can be replaced by the mixed protocol ABY framework \cite{Demmler2015ABYA}. Of course, as mentioned above, a two server secret share based implementation is also possible. 



%\input{SMCvsCryptE}

\section{Background}
In this section we give a brief introduction to definitions and primitives relevant to \system. 

\subsection{Differential Privacy}
\begin{definition} \label{def:dp}
An algorithm $\mathcal{A}$
satisfies $\epsilon$-differential privacy ($\epsilon$-DP), where $\epsilon > 0$ is a privacy parameter, iff
 for any two datasets $D$ and $D'$ that differ in a single record, we have
\begin{gather}
\forall t \in Range(\mathcal{A}), Pr \big[\mathcal{A}(D) = t\big] \leq e^{\epsilon}Pr\big[\mathcal{A}(D') = t\big]
\end{gather}
where $Range(\mathcal{A})$ denotes the set of all possible outputs
$\mathcal{A}$.
\end{definition} 
When applied multiple times, the differential privacy guarantee degrades gracefully as follows.
\begin{theorem}(Sequential Composition). \label{theorem:seq}
If $\mathcal{A}_1$ and
$\mathcal{A}_2$ are $\epsilon_1$-DP and $\epsilon_2$-DP algorithms that use independent randomness, then releasing the outputs $\mathcal{A}_1(D)$ and
$\mathcal{A}_2(D)$ on database $D$ satisfies $\epsilon_1+\epsilon_2$-DP.\end{theorem} 
%There exist advanced composition theorems that give tighter privacy lossbounds, but weuse Theorem 1 for our paper.
Another important result for differential privacy is that any post-processing computation performed on the noisy output of a differentially private algorithm does not cause any loss in privacy.
\begin{theorem}[Post-Processing]\label{post}
Let $\mathcal{A}: D \mapsto R$ be a randomized
algorithm that is $\epsilon$-DP. Let $f : R \mapsto R'$ be an
arbitrary randomized mapping. Then $f \circ \mathcal{A} : D \mapsto R'$ is $\epsilon$-
DP. \end{theorem}
We also define the \emph{stability} of a data transformation operation.
\begin{definition}\label{def:stability}
A transformation $\mathcal{T}$ is defined to be $t$-stable if for two datasets $D$ and $D'$, we have\begin{gather}|\mathcal{T}(D)\ominus \mathcal{T}(D')| \leq t \cdot |D\ominus D'|  \end{gather} where  (i.e.,  $D \ominus D' = (D-D') \cup (D'-D)$. \end{definition}
Transformations with bounded stability scale the differential privacy guarantee of their outputs, by their stability constant \cite{PINQ}.
\begin{theorem}\label{theorem:stability}
Given $\mathcal{T}$ be an arbitrary $t$-stable transformation on dataset $D$ and an $\epsilon$-DP algorithm $\mathcal{A}$ which takes output of $\mathcal{T}$ as input, the composite computation $\mathcal{A} \circ \mathcal{T}$ provides $(\epsilon \cdot t)$-DP.\end{theorem}

%For other basic properties of differential privacy we refer the readers to the brilliant book by Dwork et al. \cite{Dork}.
\subsection{Cryptographic Primitives}
\stitle{Linearly Homomorphic Encryption (\textsf{LHE}).}
Let $(\mathcal{M}, +)$ be a finite group. A \textsf{LHE} scheme
for messages in $\mathcal{M}$ is defined  as \squishlist
\item \textbf{Key Generation }($Gen$) -This  algorithm takes the security parameter $\kappa$ as input and outputs
a pair of secret and public keys, $(s_k, p_k) \leftarrow Gen(\kappa)$.
\item \textbf{Encryption} ($Enc$) - This is a randomized algorithm that encrypts a message $m \in \mathcal{M}$ via the public key $p_k$, to generate ciphertext $\mathbf{c} \leftarrow Enc_{pk}(m)$.
\item \textbf{Decryption} ($Dec$) - This is a deterministic function that uses the secret key $s_k$ to
recover the plaintext $m$ from ciphertext $\mathbf{c}$.
\squishend
In addition, \textsf{LHE} supports the operator $\oplus$ that allows the summation of ciphers as follows:
\\ \textbf{Operator} $\oplus$ - Let $c_1 \leftarrow Enc_{pk}(m1), \ldots, c_a \leftarrow Enc_{pk}(m_a), a \in \mathcal{Z}_{>0}$. Then we have  $Dec_{sk}(c_1\oplus c_2 ...\oplus c_a)=    m_1 + \ldots   + m_a$.  \\
One can multiply a cipher $c\leftarrow  Enc_{sk}(m)$ by a plaintext positive integer $a$ by $a$ repetitions of $\oplus$. We denote this operation by $cMult(a,c)$ such that $Dec_{sk}\big(cMult(a,c)\big)=a\cdot m$.\\
\stitle{Labeled Homomorphic Encryption(\textsf{labHE}).}
Let $(Gen,Enc,Dec)$ be an \textsf{LHE} scheme with security parameter $\kappa$ and message space $\mathcal{M}$. Assume that a multiplication operation is given in $\mathcal{M}$, i.e., is a finite ring. Let also $\mathcal{F}:\{0,1\}^s \times \mathcal{L}\rightarrow \mathcal{M}$ be a pseudo-random function with seed space $\{0,1\}^s$( s= poly($\kappa $)) and the label space $\mathcal{L}$. A \textsf{labHE} scheme is defined as
\squishlist
 \item $\textbf{labGen}(\kappa)$ - On input $\kappa$, it runs $Gen(\kappa)$ and outputs $(sk,pk)$
\item $\textbf{localGen}(pk)$ -  For each user $i$ and with the public key as input, it samples a random seed $\sigma_i \in \{0,1\}^s$ and computes $pk_i = Enc_{pk}(\underline{\sigma_i})$ where $\underline{\sigma_i}$ is an  encoding of $\sigma_i$ as an  element of $\mathcal{M}$. It outputs $(\sigma_i,pk_i)$.
\item $\textbf{labEnc}_{pk}(\sigma_i, m , \tau)$: On input a message $m \in \mathcal{M} $ with label $\tau \in \mathcal{L}$  from user $i$, it computes $b=\mathcal{F}(\sigma_i, \tau)$ (known as the mask) and outputs the labeled ciphertext $\mathbf{c}=(a,d) \in \mathcal{M} \times \mathcal{C}$ with $ a= m- b$  (known as the hidden message) in $\mathcal{M}$ and $d=Enc_{pk}(b)$. For brevity we just use notation $\textbf{labEnc}_{pk}(m)$ to denote the above functionality, in the rest of paper. 
\item $\textbf{labDec}_{sk}(\mathbf{c})$ - This functions inputs a cipher $\mathbf{c}=(a,d) \in \mathcal{M} \times \mathcal{C}$ encrypted under  \textsf{labHE} and decrypts it as $m=a-Dec_{sk}(d)$.\squishend
Both \textsf{LHE} and \textsf{labHE} provides semantic security guarantee \cite{Katz}.
\stitle{Secure Computation}
%\arc{Not final:placeholder}
Garbled circuit, also known as Yao's protocol \cite{Yao,Yao2},  is a generic method for secure computation. Two data owners with respective private inputs $x_1$ and $x_2$ run the protocol such that, no party learns more  
than what is revealed from the output value $f(x_1,x_2)$ for a function $f$.  One of the data owners, called
generator, builds a "garbled" version of a circuit computing $f$ and sends it over to the other data owner, called evaluator, alongside the garbled input values for $x_1$.  Upon receiving the circuit, the evaluator 
engages in an oblivious transfer protocol with the generator to obliviously obtain the garbled input for $x_2$ and subsequently computes the  output $f(x_1, x_2)$.
%A more detailed discussion is presented in Appendix A.




\section{\system Overview}\label{sec:overview}
In this section, we describe \system's workflow (Section~\ref{sec:wf}), its modules (Section~\ref{sec:modules}) and trust assumptions (Section~\ref{sec:trust}). % We end with a brief discussion justifying the design of \system (Section~\ref{sec:discuss-arch}). %The key notations used in this paper are summarized in Table~\ref{Notations}.

\subsection{\system Workflow}\label{sec:wf}
\eat{ The architecture of \system is illustrated in Figure~\ref{fig:system}. A set of \textit{data owners} ${\textsf{DO}_i, i\in [m]}$ have private data records ${D_i, i \in [m]}$. \system permits data analysts to author and execute programs $P$ that satisfy DP under the \cdp model without storing or computing on the private data records in the clear. \system achieves this by aggregating encrypted private records at the \textit{Analytics Server} (\textsf{AS}). Keys needed for encrypting private records and decrypting answers are managed by the \textit{Cryptographic Servide Provider} (\textsf{CSP}) so that the data owner need not participate in the differentially private computation.}

\system operates in three phases: \\
(1) \textbf{Setup Phase}: At the outset, data owners initialize the \textsf{CSP} with a privacy budget $\epsilon^B$, which is stored in its \textit{Privacy Engine} module. Next, the  \textsf{CSP}'s \textit{Key Manager} module generates key pair $(sk,pk)$ for labHE, publishes $pk$ and stores $sk$. 
\\(2) \textbf{Data Collection Phase}: In the next phase, each data owner encodes and encrypts his/her record using the \textit{Data Encoder} and \textit{Data Encryption} modules and sends the encrypted data records to the \textsf{AS}. The data owners are relieved of all other duties and can go completely offline. The \textit{Aggregator} module of the \textsf{AS} then aggregates these encrypted records into a single encrypted database $\boldsymbol{\tilde{\mathcal{D}}}$.
\\(3) \textbf{Program Execution Phase}: In this phase, the \textsf{AS} executes a \system program provided by the data analyst. \system programs (details in Sections~\ref{sec:operators} and \ref{sec:implementation}) access the sensitive data via a restricted set of transformation operators, that filter, count or group the data, and measurement operators, which are DP operations to release noisy answers. Measurement operators need interactions with the \textsf{CSP} as they require (a) decryption of the answer, and (b) a check that the privacy budget is not exceeded. These  functionalities are achieved by \CSP's \textit{Data Decryption} and \textit{Privacy Engine} modules.

 The \emph{Setup} and \emph{Data Collection} phases occur just once at the very beginning, every subsequent program  is handled via the corresponding  \emph{Program Execution} phase. %We next detail the roles of the different modules in the data owners, the Analytics Server and the Cryptographic Service Provider.  
\eat{
\begin{table}[t]
\centering
\caption {Notations}
\scalebox{0.8}{
 \begin{tabular}{l|l}  \toprule
 \multicolumn{1}{c}{\textbf{Symbol} } &  \multicolumn{1}{c}{\textbf{Explanations}}\\\midrule
\textbf{Boldface}& \text{- represents encrypted data}\\
$\tilde{}$ & \text{- represents one-hot-coding}  \\  $\hat{}$ & - represents a differentially private output  \\ $A$ &- an attribute  \\ $s_A$ &- size of domain of attribute $A$
\\$dom(A)=\{v_1,\ldots,v_{s_A}\}$ & - domain of attribute $A$\\ $ct_{A,i}$  &- \# \text{records with value $v_i$ for attribute} A\\ $m$   &- \text{\# number of data onwers}\\ $\boldsymbol{\tilde{\mathcal{D}}}$  &- \text{encrypted database with records in}\\&\text{  per-attribute one-hot-coding } \\ %$\mathcal{A}=\{\mathcal{A}_1,...\mathcal{A}_l\}$   &- \text{set of attributes in the schema of $\boldsymbol{\tilde{\mathcal{D}}}$}\\
$x \times y \text{ table } \mathbf{T}$   &- \text{an encrypted table  with $x$ records in}\\&\text{ one-hot-coding and $y$ columns one for}\\&\text{ each attribute; serves as one of the }\\&\text{ inputs to a transformation operator}\\ $\mathbf{B}$&- \text{a vector of length $m$ such that each entry}\\&\text{ $\textbf{B}[i]$ represents whether record $r_i, i \in [m]$}\\& \text{is relevant to the program at hand} \\ $V$ & -\text{ represents a vector}\\$c$ &- \text{represents a scalar}\\$\mathcal{P}$ & - \text{represents a set}\\
 \bottomrule
 \end{tabular}}
 \label{Notations}
\end{table}}
\subsection{\system Modules}\label{sec:modules}

\stitle{Cryptographic Service Provider (\textsf{CSP})}\\
(1)\textbf{ Key Manager }- The foremost duty of the \textsf{CSP} is to initialize the encryption scheme of Crypt$\epsilon$. This task is handled by the \textit{Key Manager} module which generates the key pair $(sk,pk)$ for the \textsf{labHE} scheme. It stores the secret key, $sk$ with itself and releases the public key, $pk$. Note that since the \textsf{CSP} has exclusive access to the secret key $sk$, it is the only entity capable of decryption in \system.\\
(2)\textbf{ Privacy Engine }- Crypt$\epsilon$ starts of with a total privacy budget of $\epsilon^B$ which is unanimously agreed upon by all the data owners. Note that the mechanism of deciding $\epsilon^B$ should be piloted by social prerogatives \cite{e1,e2}
and is currently outside the scope of Crypt$\epsilon$. For executing any program, the \textsf{AS} has to interact with the \textsf{CSP} at least once (for decrypting the noisy answer) thereby giving the \textsf{CSP} the opportunity to monitor the \textsf{AS}'s actions in terms of privacy budget expenditure. The \textit{Privacy Engine} module hence maintains a public ledger that records the privacy budget spent in executing each program. Once the privacy cost incurred reaches 
$\epsilon^B$, the \textsf{CSP} refuses to decrypt any further answers thereby ensuring that the privacy budget is not exceeded.  The ledger is completely public allowing any data owner to verify it.\\ %as and when desired.
(3)\textbf{ Data Decryption }- The \textsf{CSP} being the only entity capable of decryption,  any measurement of the data (even noisy) has to involve the \textsf{CSP}. The \textit{Data Decryption} module is tasked with handling all such interactions with the \textsf{AS}. 

\stitle{Data Owners (\textsf{DO})}\\
(1)\textbf{ Data Encoder} -  Each data owner $\textsf{DO}_i, i \in [m]$ has a private data record $D_i$ of the form $\langle A_1,...A_l\rangle$ where ${A}_j$ is an attribute. At the very outset, every data owner  $\textsf{DO}_i$ represents his/her private record $D_i$ in its respective per attribute one-hot-encoding format. The one-hot-encoding is a way of representation for categorical attributes and is illustrated by the following example. 
If the database schema is given by  $\langle Age,Gender\rangle$ then the corresponding one-hot-encoding representation for a data owner $DO_i, i \in [m]$ with the record $\langle 30, Male\rangle$, is given by $\tilde{D_i}=\langle[\underbrace{0,\ldots,0}_{29},1,\underbrace{0,\ldots,0}_{70}],[1,0]\rangle$. \\
(2)\textbf{ Data Encryption} - The \textit{Data Encryption} module stores the public key $pk$ of \textsf{labHE}  which is announced by the \CSP. Each data owner performs an element-wise encryption of his/her per attribute one-hot-codings using $pk$ and sends the encrypted record to the \textsf{AS} via a secure channel. \eat{In our aforementioned example, we get
\begin{eqnarray}
\mathbf{\tilde{D}}&=&\langle[\underbrace{labEnc_{pk}(0),\ldots}_{29},labEnc_{pk}(1),\nonumber \\
&& \underbrace{\ldots,labEnc_{pk}(0)}_{70}], [labEnc_{pk}(1),labEnc_{pk}(0)]\rangle. \nonumber \end{eqnarray}}
 This is the only interaction that a data owner ever participates in and remains offline through all subsequent program executions.

\stitle{Analytics Server (\textsf{AS)}}\\
(1)\textbf{  Aggregator} - The \textit{Aggregator} collects the encrypted records $\mathbf{\tilde{D}}_i$ from each of the data owners $\textsf{DO}_i$ and collates them into a single encrypted database $\boldsymbol{\tilde{\mathcal{D}}}$. %Note that in contrast, the server in the \textsf{CDP} model, being trusted, stores the data in the clear whereas in the \textsf{LDP} model the untrusted server stores appropriately randomized (noisy) data.
\\(2)\textbf{ Program Executor }- The \textit{Program Executor} takes as input a logical \system program and privacy parameter $\epsilon$ from a data analyst, translates it to the implementation specific secure computation protocol and publishes the noisy output computed with the assistance of the \textsf{CSP}. %\system programs can access the sensitive data using  operators of two types -- transformation  and measurement. Transformation operators allow certain modifications on the encrypted data and are performed almost entirely by the \textsf{AS}. The measurement operators reveal some noisy measurement of the data. %\system supports two types of measurement operators implementing two of the most popular DP mechanisms, Laplace mechanism and Noisy-Max \cite{Dork}. %A typical \system program execution consists of  a series of transformation on the encrypted data followed by a measurement operator. \system programs can also involve programming constructs like loops and conditionals, and permit arbitrary post-processing of the outputs of the measurement operators.  
\subsection{Trust Model}\label{sec:trust}
We assume that the servers \textsf{AS} and \textsf{CSP} are semi-honest, i.e., they follow the protocol honestly, but their contents and computations can be observed by an adversary.
Thus from the data owners perspective the trust assumption is similar to that of \textsf{LDP}; the data owners need not place their trust in any external entity.
However there are two modest differences in the \system setting from the \textsf{LDP} setting.
\squishlist
\item We assume that the \textsf{AS} and the \textsf{CSP} are non-colluding and follow the \emph{honest-but-curious} (or \textit{semi-honest}) adversary model. %That is, they always follow the instructions of the protocol faithfully but strive to learn extra information about the private records from the messages received during the execution of the protocol. 
We also assume that each data owner has a private channel with the \textsf{AS}. %This is to prevent any third party (including the \textsf{CSP}) from eavesdropping.
\item The adversary is now reduced to a \textit{computationally bounded} one as opposed to the information theoretic one  in \textsf{LDP}.
 \squishend
\eat{Table~\ref{DPCompare} summarizes the distinctions between \ldp , \cdp and \system.
\begin{table}[t]
\centering
\caption {Comparative analysis of different DP models}
\scalebox{0.7}{ \begin{tabular}{|c| c c c|}  \toprule
\multicolumn{1}{|c}{\textbf{Features}} & \textbf{LDP}  & \textbf{CDP}  & \textbf{Crypt$\epsilon$}  \\ [0.5ex]
 \hline \hline\# Centralized Servers & 1& 1 & 2\\\hline
Trust Assumption & & & Untrusted \\   for Centralized & Untrusted & Trusted & Semi Honest \\ Server &  &   &  Non-Colluding  \\ \hline
Data Storage & \multirow{2}{*}{N/A} & \multirow{2}{*}{Clear} & \multirow{2}{*}{Encrypted} \\in Server & &  &  \\\hline
\multirow{2}{*}{Adversary} & Information & Information & Computationally \\& Theoretic & Theoretic & Bounded\\\hline
 Error on Statistical Counting Query& $O\Big(\frac{\sqrt(n)}{\epsilon}\Big)$& $O\Big(\frac{1}{\epsilon}\Big)$ & $O\Big(\frac{1}{\epsilon}\Big)$\\
  [1ex]
 \bottomrule
 \end{tabular}}\label{DPCompare}
\end{table}}

 

\eat{\subsection{\system Design Principles}\label{sec:discuss-arch}
The design of \system is guided by the following principles. 

\stitle{Expressibility:} \system is designed to ensure that state-of-the-art DP programs can be executed on sensitive data by the \textsf{AS}. This necessitates adding noise to functions computed on the entire database (like in \cdp), and not just to individual records (like in \ldp). In principle, any function can be computed on encrypted data via secure computation, and thus \system can support any DP algorithm. However, we currently limit the expressibility of the programs supported in \system to those that operate on the sensitive data with efficiently implementable operators, and whose privacy can be easily tracked by the \textsf{CSP}. Nevertheless, as shown later in the paper, \system can already support a variety of state-of-the-art DP programs that provide orders of magnitude higher error than their \ldp counterparts. %The separation between \textsf{LDP} and \system is discussed further in Appendix \ref{app:sepldp}.

\stitle{Minimal trust assumptions:} The only assumptions we make are that adversaries are computationally bounded and that the \textsf{AS} and  \textsf{CSP} do not collude. The former allows us to use cryptographic tools, and is in tune with a growing line of work that seeks to address \cdp's trust assumptions via cryptographic operators \cite{Prochlo,mixnets,amplification,Shi,Shi2,kamara,Rastogi}. The latter assumption of non-colluding servers is a popular model for privacy preserving computations \cite{Boneh1,Boneh2,Ridge2,Matrix2,secureML,LReg,Ver}.

\stitle{Data owners are offline:} \system's goal is to mimic the \cdp model with untrusted centralized servers. Hence, it is designed so that the data owners are offline once they submit their encrypted records to the \textsf{AS}. This is beneficial as the \textsf{AS} does not need to maintain communication channels with the data owners over a long period of time. If the data owners were online, some of our programs can be made more efficient as some of the computation from the \textsf{AS} and \textsf{CSP} can be offloaded to the data owners.


\stitle{Low burden on \textsf{CSP}:} \system views the \textsf{AS} as an extension of the analyst, and that it has a vested interest in obtaining the result of the computation.   Thus in \system, we require the \textsf{AS} to  shoulder the major chunk of the workload for any \system program execution; interactions with the \textsf{CSP} should be minimal and only related to data decryption. Keeping this in  mind, we design the \textsf{AS} to perform most of the data transformations by itself (Table \ref{perf}). Specifically for every Crypt$\epsilon$ program, the \textsf{AS} processes the whole database and transforms it into concise representations (like an encrypted scalar or a short vector) which is then decrypted with the help of the \textsf{CSP}. An example setting in the real world can be when Google assumes the role of the \AS and Symantec can provide the services of the \CSP. It is interesting to note that we could have had an alternative implementation for Crypt$\epsilon$ where the private database is equally shared between the two servers and they engage in a secret share based secure computation protocol for computing the differentially private answers. However, this would require both the servers to do almost equal amount of work for every program. Such an arrangement would be justified only if both the servers are equally invested in learning the differentially private statistics and is ill-suited for Crypt$\epsilon$. A real world analogy for this can be when Google and Baidu decide to compute some statistics on their combined user bases. We speculate that the above scenario is much less likely than the other setting. Additionally, oblivious transfer (OT) is the major bottleneck for most SMPC implementations \cite{OT:bottleneck:1,OT:bottleneck:2,OT:bottleneck:3} and in a secret share based computation, each AND (equivalently multiplication) gate involves an OT step. Thus a secret share based implementation would be much more communication intensive resulting in a performance hit. 

\stitle{Separation of the logical programming framework and the underlying physical implementation:} The programming framework (i.e., the set of \system operators) is independent from the underlying cryptographic operator specific implementation. This separation between the logical and physical layers allows certain flexibility in the choice for lower level physical implementation as follows --
\\(1) \stitle{Choice of input data representation: } For the prototype \system presented in this paper, as discussed above the input data is represented in per attribute one-hot-encoding. However, we can easily switch to any other encoding scheme like multi-attribute one-hot-encoding, range based encoding etc. Quite evidently, different encoding scheme will have different storage requirements. For example, a k-attribute one-hot-encoding will be of size $d^k$ where $d$ is the domain size of each attribute. We choose one-hot-coding for our prototype \system implementation because it is a general representation that can be easily translated to other encoding schemes.  
\\(2) \stitle{Choice of cryptographic operators: } As discussed above, due to our design choice of having low burden on the \CSP, we implement \system using LHE and garbled circuits. However translation to an implementation based on other cryptographic operators is straighforward. For example, the optimized HE scheme in \cite{Blatt2019OptimizedHE} can be used in place of LHE. Similarly, garbled circuits can be replaced by the mixed protocol ABY framework \cite{Demmler2015ABYA}. Of course, as mentioned above, a two server secret share based implementation is also possible. 
\\ (3) \stitle{Choice of DP operators: }Although in this paper, we use $\epsilon$-DP (pure DP) for our privacy analysis, \system can be easily extended to accommodate other DP notions like $(\epsilon,\delta)$-DP, R\`enyi differential privacy (RDP) \cite{RDP} etc. For example, designing a operator for adding Gaussian noise would suffice to allow RDP analysis in \system.
}
\section{Crypt$\epsilon$ Primitives}
\subsection{Primitive Definitions} 
The input to Crypt$\epsilon$ is an encrypted instance of a database $\boldsymbol{\tilde{\mathcal{D}}}$ with a single relational schema $\langle \mathcal{A}_1,\mathcal{A}_2, . . . ,\mathcal{A}_l\rangle$. Each attribute $\mathcal{A}_i$ is assumed to be discrete
(or suitably discretized) and represented in one-hot-coding form. 
There are two types of Crypt$\epsilon$ primitives namely transformations and measurement.
\subsubsection{\textbf{Transformation}}
 Transformation primitives take
as input an encrypted data source variable (a table of size $x \times y, x,y \in \mathcal{Z}_{\geq 0}$) and output
a transformed data source (again  a table $x' \times y', x',y' \in \mathcal{Z}_{\geq 0}$) that is still encrypted. Typically $x$ and $x'$ are equal to $m$, the total number of data owners, i.e., every tuple in the data source tables corresponds to the record of a single data owner.
The transformation primitives are mostly carried out by the \textsf{AS} on its own; this is enabled by our use of labeled homomorphic encryption scheme which allows us to perform certain operations, specifically multiplication and addition, directly over the encrypted data. %Only two transformations  (\textsf{GroupBy} and \textsf{CountDistinct}) need to be computed via a secure computation protocol between the \textsf{AS} and the \textsf{CSP}. 
Since these primitives work entirely on encrypted data, they do not expend the privacy budget. However these operators can affect the privacy analysis through their stability. Every transformation in Crypt$\epsilon$ has a well-established stability.
For each record of the database $\boldsymbol{\tilde{\mathcal{D}}}$ ( i.e., data corresponding to a single data owner) we maintain an encrypted bit which indicates whether the record is relevant to the query at hand. Let $\mathbf{B}$ represent this bit vector where $\mathbf{B}[i]$ corresponds to this indicator bit for the $i^{th}$ record.  If $\mathbf{B}[i] =Enc_{pk}(1)$, then the $i^{th}$ record is to be considered for answering the current query and vice versa. Only one of the transformation, \textsf{Filter} alters the bit vector $\mathbf{B}$. Before every program execution, $\mathbf{B}$ is initialized to a 1-vector. 
\begin{enumerate}

	\item \textsf{CrossProduct} ($\tilde{\mathbf{T}}, A_i, A_j$) - Given encrypted one-hot-codings for two different attributes $A_1$ and $A_2$ of domain sizes $s_{A_1}$ and $s_{A_2}$ respectively, the goal of this transformation is to compute the encrypted one-hot-coding for the entire two-dimension domain of the new $\lq$attribute' $A_1\times A_2$ of size $s_{A_1}\cdot s_{A_2}$. Thus this transformation takes as input a $x \times y $ table, $\tilde{\mathbf{T}}$ defined over attribute set $A=\{A_1,A_2,...,A_y\}$ where each cell $\tilde{\mathbf{T}}[i,j] , i \in [x], j \in [y], 2 \leq y \leq k$ corresponds to the encrypted one-hot-coding for attribute $A_j$ for the data owner $\textsf{DO}_i$ and outputs a $x \times (y-1)$ table with attribute set $\{A_1\times A_2,A_3,\ldots,A_{y}\}$.  Note that the construction of the one-hot-coding of the full $y$-dimension domain can be computed by repeated application of this transform. 	
    
	
	\item \textsf{Project}($\tilde{\mathbf{T}},A^*$)- In addition to the $ x \times y$ table, $\tilde{\mathbf{T}}$ over attribute set $\{A_1, A_2, ..., A_y\}$, the \textsf{Project} transformation takes a set of attributes $A*=\{A^*_1,...A^*_p\}, p < y$ as inputs. The result of the transformation is defined as the $x \times p$ data source table where each record is just restricted to the attribute set $A^*$, i.e., it discards all other attributes. 
	%Infact it is analogous to the operation of marginalization which is described as follows.
	%Assuming  $A$ and $B$ to be two attributes with finite domains, let $x$ be a vector of counts representing a histogram over the cross product of the domain (with $|A|*|B|$ entries).
	%Marginalization over the attribute $B$ results in a vector of counts on the attribute $A$ alone by adding up counts corresponding to the same value of $A$.  
   
  \item \textsf{Filter}($\tilde{\mathbf{T}},\phi$) - Let $\tilde{\mathbf{T}}$ be an encrypted table of one-hot-codings over attribute set $A=\{A_1,...,A_k\}$, $\phi$ be a  predicate defined over a subset of attributes $A^*\subseteq A$ and $\mathbf{B}$ be the current state of the indicator vector which is stored by the \textsf{AS}. The predicate $\phi$ has to be expressed as a conjunction of range conditions over $A^*$, i.e.,\begin{gather}\phi = \bigwedge_{A \in A^*}(A \in \{v_{1},\ldots,v_{t}\} ) \label{phi} \end{gather} If for some attribute $A \in A^*$, the condition is a equality condition as $A==v$ instead of a range condition, then simply put $v_1=v_t$. For each record $r_i, i \in [m]$, the Filter transformation zeros the corresponding indicator bit $\mathbf{B}[i] $ if $\phi(r_i)=False$. $\mathbf{B}[i] $ is kept unchanged otherwise. Thus the \textsf{Filter} transformation suppresses all the records that are extraneous to answering the program at hand (i.e., doe snot satisfy $\phi$) by explicitly zeroing the corresponding indicator bits and outputs the updated indicator vector. %It takes as input a $x \times 1$ table $\tilde{\mathbf{T}}$, whose every row is an encrypted one-hot-encoding (of the form $\mathbf{\tilde{R}}$) for the attribute of concern $A$, and a vector $\mathbf{C}$ which has encryptions of appropriate non-zero weights for indices that satisfy $\phi$. %Note that $A$ need not be an attribute of the original attribute set $\mathcal{A}$ but can be a new multi-dimension 'attribute' constructed over $\mathcal{A}^* \subseteq \mathcal{A}$, i.e., $A= \prod_{A*_i \in \mathcal{A}^*  }A^*_i$. 
    \item{Count($\mathbf{T}$) } - The count transformation outputs the encrypted value of the non-noisy true count for the program at hand. For answering linear counting queries, typically \textsf{Count}  is the last transformation to be applied and is immediately preceded by a \textsf{Filter} transformation. Recall that the \textsf{Filter} transformation sets bit $i \in [m]$ to be 1 (encrypted) if the $i^{th}$ record satisfies the filter condition and 0 otherwise and outputs this encrypted $m\times 1$ vector. Hence the \textsf{Count} primitive simply adds up all the entries of this bit vector $\hat{\mathbf{B}}$ and  outputs the sum which is a single encrypted value. 
    \item{GroupBy*($\mathbf{\tilde{T}},A$)}- The purpose of the \textsf{GroupBy*} transformation is to essentially bucket the input $x\times y$ table $\mathbf{\tilde{T}}$ into groups of records having the same value for an attribute of choice $A$. The output of this transformation is an $1\times s_A$ encrypted table (a vector)  where each element $\mathbf{T'}[i], i \in [s_A]$ represents the encrypted count of the number of records in $\boldsymbol{\tilde{\mathcal{D}}}$ having value $v_{i,A}$.
    This primitive serves as a preceding transformation for other Crypt$\epsilon$ primitives like \textsf{NoisyMax}, \textsf{CountDistinct} et al.
     \item{GroupBy($\mathbf{\tilde{T}},A$)-} The \textsf{GroupBy} transformation is similar to the aforementioned \textsf{GroupBy*} transformation. The only difference between the two is that, the former outputs the encrypted one-hot-coding of the respective counts. That is, the output of \textsf{GroupBy}($A$)  is an $s_A$ lengthed encrypted vector $\tilde{\mathbf{V}}$ such that each element, $\tilde{\mathbf{V}}[i], i \in [s_A]$ represents the encrypted one-hot-encoding of the number of records in $\boldsymbol{\tilde{\mathcal{D}}}$ having value $v_{i,A}$. This transformation allows us to answer queries based on the count of a particular value for attribute $A$.
     %Note that since for \textsf{GroupBy} we need to create the one-hot-coding of the counts, this requires an interaction with the \textsf{CSP}.
     \item {CountDistinct($\mathbf{V}$)-} As mentioned before, the \textsf{CountDistinct()} primitive takes as input an encrypted vector $\mathbf{V}$ which is the output of a \textsf{GroupBy*}($A$) primitive for some attribute $A$. Thus the \textsf{CountDistinct}() primitive  returns the number of distinct values of $A$ that appear in the records of $\boldsymbol{\tilde{\mathcal{D}}}$ by counting the non-zero entries of $V$.  
\end{enumerate}
Note that the first four transformations namely \textsf{CrossProduct, Project, Filter} and Count are performed by the \textsf{AS} alone. Only for transformation \textsf{GroupBy} the \textsf{AS} engages in a secure computation protocol with the \textsf{CSP}.
\subsubsection{Measurement} The measurement operators involve implementing the differentially private algorithms of laplace mechanism and noisy max.  Thus they expend the privacy budget and require secure computation between the \textsf{AS} and the \textsf{CSP}. 
\begin{comment} All measurement operators must involve joint computation with the \textsf{CSP}. Note that the requisite noise to be added to ensure differentially privacy has to be jointly added by both the \textsf{AS} and the \textsf{CSP}. It is so because, had only either one of the servers added the noise, then that server would be able to retrieve the true non-noisy answer by simply de-noising the published differentially private answer. This means that the sensitivity of the program being executed should be known to both the servers. This poses no hindrance in our setting  since the program is public, the  sensitivity computation can be performed very easily by observing the sequence of the preceding transformations.
\end{comment}
\begin{enumerate}
	\item Laplace($\mathbf{V},\epsilon$) - In the Laplace mechanism, in order
to publish $f(D)$ where $f : D \mapsto R$, $\epsilon$-differentially private mechanism $\mathcal{M()}$ 
publishes $f(D) + Lap\Big(\frac{\Delta f}{\epsilon}\Big)$  
where $\Delta f = \max_{D,D'}||f(D)-f(D')||_1$ is known as the sensitivity of the query. The p.d.f of $Lap(b)$ is given by\begin{gather}\mathbf{f}(x)={\frac  {1}{2b}}e^{ \left(-{\frac  {|x-\mu |}{b}}\right)}\end{gather} The sensitivity of the function $f$ basically captures the magnitude by which a single individual’s data can change the function $f$ in the worst case. Therefore, intuitively, it captures the uncertainty in the response that we must introduce in order to hide the participation of a single individual. For counting queries the sensitivity is 1. The Laplace primitive enables the \textsf{AS} and the CSP to add two separate instances of random laplace noise to the true result of a counting query for generating a differentially private output. It takes an input an encrypted vector $\mathbf{V}$ (could be a scalar too) and adds two instances of noise drawn from $[Lap(\frac{1}{\epsilon}]^{|V|}$ to it.

	\item NoisyMax($\mathbf{V},\epsilon$)-Noisy-Max is a type of differentially-private selection mechanism where the goal is to determine the counting query with the highest value out of $n$ different counts.  
	The algorithm works as follows. First, generate each of the counts and then add independent Laplace noise from the distribution $Lap(\frac{1}{\epsilon})$ to each of them. The index of the largest noisy count is then reported as the noisy max.
	This has two fold advantage over the naive implementation of finding the maximum count.
Firstly, noisy-max applies "information minimization" as rather than releasing all the noisy counts
and allowing the analyst to find the max and its index, only the
index corresponding to the maximum is made public.
Secondly, the noise added is much smaller than that in the case of the naive implementation (it has sensitivity $\Delta f=m$).  

\end{enumerate}
\subsection{Implementations}\label{implementation}
In this section we describe the implementation of Crypt$\epsilon$.
 \paragraph*{General n-way multiplication using $labMult$- $genLabMult()$}
In this paper we also propose a method to extend the $labMult()$ operation of labHE to support $n > 0$ multiplicands. 
Consider the case where we want to multiply the respective ciphers of  $n$ messages $\{m_1,...m_n\} \in \mathcal{M}^n$. Note that the reason why we can't simply use $labMult()$ for a generic $n-$ way multiplication is because, the "multiplication" cipher $\mathbf{d}=labMult(\mathbf{c_1},\mathbf{c_2})$ does not have  a corresponding label. Thus for generalizing the $labMult()$ operation for $n$ multiplicands what we have to do is to generate a label and a seed for every intermediary product of two multiplicands. This can be done in the following way-  \\
Consider two ciphers $\mathbf{c_1}$ and $\mathbf{c_2}$ corresponding to messages $m_1$ and $m_2$. AS computes 
$d'=labMult(\mathbf{c_1,c_2}) \odot Enc(r)$ where r is a random mask and sends $d'$ to CSP. Now CSP decrypts it and subtracts $b_1\times b_2$ to get plaintext $\tilde{d}=m_1\times m_2+r$. Note that the mask r protects the value of $m_1\times m_2$ from the CSP. The CSP next assigns a seed $\sigma'$ and label $\tau'$ to the product and returns the value $\tilde{c}=(\tilde{a},c')$ to AS where $\tilde{a}=m_1\times m_2 -b' +r$, $b'=PRF(\sigma',\tau')$ and $c'=Enc_{pk}(b')$ to the AS. AS can subtract r from $\tilde{a}$ to obtain the  true value of the hidden message $a'=m_1\times m_2 - b'$. However since $b'$ is not known to AS, $m_1\times m_2$ remains hidden from the AS as well. Now with the true cipher $\mathbf{c}=(a',c')$ the AS can compute further multiplications on it. \\
Thus for a generic $n-way$ multiplication the order of multiplication can be parallelized as follows to require a total of $\lceil \log n\rceil$ rounds of communication with the CSP. 
\begin{gather*}
\underbrace{m^j_1\times m^j_2 \hspace{1cm}      m^j_3\times m^j_4}    \hspace{1cm}    ... \hspace{1cm}\underbrace{m^j_{n-3} \times m^j_{n-2} \hspace{1cm}  m^j_{n-1}\times m^j_n}\\
\end{gather*}
\arc{TO-DO complete diagram}\\
Now let us explicate the implementation details of the aforementioned Crypt$\epsilon$ primitives. 
\begin{enumerate}\item CrossProduct($\tilde{\mathbf{T}}, A_i, A_j$): Let $D_1$ and $D_2$  be the encrypted one-hot-coding corresponding to two  values $v_1$ and $v_2$ (integral representation) for attributes $A_1$ and $A_2$ respectively. The corresponding encrypted one-hot-encoding for the two-dimensional attribute $A_1\times A_2$ is given by  \begin{gather} D_{1\times 2}[(i-1)*s_{A_2}+j] = labMult(D_1[i], D_2[j])\\ i \in [s_{A_1}], j \in [s_{A_2}]\end{gather} For this particular case, only $D_{1 \times 2}[(v_1-1)*s_{A_2}+v_2]=Enc(1)$ while all other indices will equate to $Enc(0)$. Note that when computing the one-hot-encoding for a t-dimensional attribute $t > 2$,  for the actual implementation, instead of calling $t$ iterative instances of CrossProduct() we use the genLabMult() operator of labeled homomorphic encryption to speed up the computation. \item Project($\tilde{\mathbf{T}}, A*$)- The implementation of the Project transformation is very straightforward, it simply drops off all the attributes from $\tilde{\mathbf{T}}$ and returns the truncated table. \item Filter($\mathbf{\tilde{T}},\phi$)-  As discussed in the preceding section, the predicate $\phi$ is expressed in a special form of conjunctions of range conditions given by eq \ref{phi}. Now for a range condition $A \in \{v_1,...v_t\}$, assuming $\mathbf{\tilde{R}_A}[i]$ is the corresponding one-hot-coding for the $i^{th}$ record's value for attribute $A$,  consider the following \begin{gather}\mathbf{c}_A^i=\bigoplus_{j=1}^{t}\tilde{\mathbf{R}}_{A}[i][v_1]\end{gather} where $\tilde{\mathbf{R}}_{A}[i][v]$ is the $v^{th}$ index of corresponding one-hot-coding. Clearly if the $i^{th}$ record satisfies the condition $A \in \{v_1,...v_t\}$, then exactly one of the values in $\{\tilde{\mathbf{R}}_{A}[i][v_j]\}, j \in \{1,...,t\}$ will be a cipher for $1$. Thus $c_A^i=1$ if record $i$ satisfies the range condition and 0 otherwise. If the condition is instead an equality predicate $A==v$ then $\mathbf{c}_A^i=\tilde{\mathbf{R}}_{A}[i][v]$. Now considering $\phi$ is given by eq \ref{phi}, let us define\begin{gather}\mathbf{c}^i=genLabMult(\mathbf{c}^i_{A1},...,\mathbf{c}^i_{Ar})\\A^*=\bigcup_{j=1}^rA_j\end{gather} It is easy to see that $c^i$=1 iff record $i$ satisfies $\phi$. Let $\mathbf{B}'$ be the indicator vector before the execution of the current instance of the filter transformation. The final step is to multiply the $\mathbf{c}^i$s with the corresponding indicator bits and obtain the updated indicator vector $\mathbf{B}$ as follows \begin{gather}\mathbf{B}[i]=labMult(\mathbf{c}^i,\mathbf{B}'[i])\end{gather} 
The above step zeros out some additional records which were found to be extraneous by some preceding filter conditions. Clearly $V$ is the output of the filter transformation.
\\\textbf{Avoid Indicator Vector Multiplication}\\
When the Filter transformation is applied for the very first time in a $Crypt\epsilon$ program and the input predicate is conditioned on a single attribute $A \in \{v_1,...,v_k\}$, then we can do the following optimization. Consider \begin{gather}\mathbf{b}[i]=\bigoplus_{j=1}^k \mathbf{\tilde{R}}_A[i][v_j], i \in [m]\end{gather} where $\mathbf{\tilde{R}}_A[i]$ is the one-hot-coding for attribute $A$ for the $i^{th}$ record. Since this is the first instance of the Filter primitive, the current indicator vector $\mathbf{B}$  will be all 1-vector. Thus $\mathbf{b}$ is itself the updated indicator vector  and we can avoid the unnecessary multiplication $labMult(\mathbf{b[i]},\mathbf{B}[i])$.   \item Count($\mathbf{T}$) - Recall that the Count transformation is always preceded by a Filter transformation. Hence it inputs the $m \times 1$ indicator vector and simple adds up its entries to return the true encrypted count  \begin{gather}\mathbf{C}=\bigoplus_{i=1}^m\mathbf{T}[i]\end{gather}%\item GroupBy*($\mathbf{V},sk$)- This primitive is an extension of the previous GroupBy transformation. 
 \item GroupBy*($\mathbf{\tilde{T}},A$) - The GroupBy* transformation   makes use of three other transformations Project, Filter and Count and is implemented as follows
\begin{enumerate} [label=(\alph*)]\item $\mathbf{\tilde{T}}_1$=Project($\mathbf{\tilde{T}}$, $A$) \item $\mathbf{B}$ =  current indicator bit vector \item  for $i = 1:100 $ \\Intialize bit vector to $\mathbf{B}$  \\$\phi_i= (Age=i) $ \\$\mathbf{B}'$ = Filter($\mathbf{\tilde{T}}_1, \phi_i$)\\ $\mathbf{C}[i]$ = Count($\mathbf{B}'$) \\ end for \item Output $\mathbf{C}$ 
\end{enumerate}
\item GroupBy($\mathbf{\tilde{T}},A$)- The initial part of the  GroupBy transformation is the exact same as that  for GroupBy* and goes as follows   \begin{enumerate}\item $\mathbf{V}$=GroupBy*$(\boldsymbol{\tilde{\mathcal{D}}},A)$ \\Note that since each entry of is a count of records, its value ranges from $\{0,...,m\}$\item AS creates a mask vector drawn uniformly at random from $[m]^{s_A}$, i.e.,  \begin{gather} M[i] \in_R [m]\end{gather} \item AS masks the encrypted true count vector $\mathbf{V}$ for attribute $A$ as follows \begin{gather}\boldsymbol{\mathcal{V}}[i]= \mathbf{V}[i] \oplus Enc_{pk}(M[i])\end{gather} and sends $\boldsymbol{\mathcal{V}}$ to CSP.\item CSP decrypts $\boldsymbol{\mathcal{V}}$, converts each entry to its corresponding one-hot-coding and encrypts it. \begin{gather*}\mathcal{V}[i]=Dec_{sk}(\boldsymbol{})\\\tilde{\mathcal{V}[i]}=\mathcal{E}(\mathcal{V}[i])\\\boldsymbol{\tilde{\mathcal{V}}}[i]=Enc_{pk}(\tilde{\mathcal{V}[i]})\end{gather*}\item Notice that each entry of $\boldsymbol{\tilde{\mathcal{V}}}$ is a $m$ -lengthed one-hot-coding vector. AS now simply rotates every entry by its corresponding mask value to obtain the desired  encrypted one-hot-coding $\boldsymbol{\tilde{V}}[i]$. \begin{gather}\boldsymbol{\tilde{V}}[i]=RightRotate(\boldsymbol{\tilde{\mathcal{V}}},M[i])\end{gather}  \end{enumerate} Note that the GroupBy primitive could have an alternative implementation using a Yao's garbled circuit that takes an input the encrypted vector and outputs the corresponding one-hot-coding representation. However this would require the circuit to decrypt and re-encrypt $O(m)$ data inside it which would be computationally heavy for larger values of m. 
\item CountDistinct($\mathbf{V},\epsilon$) - The CountDistinct() primitive is implemented as follows \begin{enumerate}\item Firstly the AS creates a mask vector drawn uniformly at random from $[m]^{s_A}$, i.e.,  \begin{gather} M[i] \in_R [m]\end{gather} \item AS masks the encrypted true count vector $\mathbf{V}$  as follows \begin{gather}\boldsymbol{\mathcal{V}}[i]= \mathbf{V}[i] \oplus Enc_{pk}(M[i])\end{gather} and sends it to the CSP \item CSP decrypts the masked encrypted vector as \begin{gather*}\mathcal{V}[i]=Dec_{sk}(\mathbf{V}[i]), i \in [|V|]\end{gather*} \item Next the CSP generates the following garbled circuit that\begin{enumerate}  \item takes the mask $M$ as an input from the AS \item takes a random number $r$  as an input from the CSP\item takes the decrypted masked vector $\mathcal{V}$ as an input from the CS \item removes the mask $M$ from $\mathcal{V}$ as \begin{gather*}V[i]=\mathcal{V}[i]-M[i], i \in [|V|]\end{gather*}\item  counts the number of non-zero entries of $V$ as C \item adds the laplace noises \begin{gather*}\mathcal{C}=C+r\end{gather*} and returns $\mathcal{C}$ \end{enumerate} \item The AS evaluates the above circuit and gets output $\mathcal{C}$ \item The AS gets $Enc_{pk}(r)$ from the CSP and generates $Enc_{pk}(\mathcal{C})$ to compute\begin{gather*}\mathbf{C}=Enc_{pk}(\mathcal{C})-Enc_{pk}(r)\end{gather*} \end{enumerate} \item LaplaceMechanism($\mathbf{V},\epsilon$)- Recall that both AS and CSP have to add Laplace noise to the output in Crypt$\epsilon$. Hence the Laplace primitive has two components. The first component is executed by the AS wherein,
\begin{enumerate} \item AS generates a noisy vector $\eta$ such that $\eta \in [Lap(\frac{1}{\epsilon})]^{|V|}$ \item encrypts $\eta$ and adds it to the input vector as \begin{gather*}\boldsymbol{\eta}=Enc_{pk}(\eta)\\\mathbf{\hat{V}}[i]=\mathbf{V}[i]\oplus \boldsymbol{\eta}[i], i \in [|V|]\end{gather*} \end{enumerate} This encrypted noisy vector $\mathbf{\hat{V}}$ is the input for the second phase of the Laplace primitive which is executed by the CSP as follows \begin{enumerate}\item Decrypts $\mathbf{\hat{V}}$ \begin{gather*}\hat{V}=Dec_{sk}(\mathbf{\hat{V}})\end{gather*}  \item Generates a noisy vector $\eta'$ such that $\eta' \in [Lap(\frac{1}{\epsilon})]^{|\hat{V}|}$ \item Finally adds the noise $\eta'$ to $\hat{V}$ \begin{gather*}\hat{\mathcal{V}}[i]=\hat{V}[i]+\eta'[i], i \in [|\hat{V}|]\end{gather*}\end{enumerate}  % Note that in the Crypt$\epsilon$ implementation we need to add two instances of the Laplace noise as opposed to a single instance in the standard central differential privacy setting. After the addition of the first instance of the laplace noise, $\eta$ (by the AS),  the encrypted answer is sent to the CSP. becuse of CSP has only a differential private view Hence the addition of the second instance of the laplace noise can be looked upon as a post-processing step  However and differential privacy is immune to post processing 
\item NoisyMax- The input to the NoisyMax primitive is an encrypted vector $\mathbf{V}$ where each entry $V[i]$ is a count. The primitive is implemented via the following steps.  \begin{enumerate}
\item First the AS adds noise to the input encrypted vector as follows \begin{gather*} \eta \in [Lap(\frac{1}{\epsilon})]^{|V|}\\\boldsymbol{\eta}=labEnc_{pk}(\eta)\\\mathbf{\hat{{V}}}[i]=\mathbf{V}[i]+ \boldsymbol{\eta} \end{gather*} \item Next the AS creates a mask vector drawn uniformly at random from $[m]^{s_A}$, i.e.,  \begin{gather} M[i] \in_R [m]\end{gather} \item AS masks the encrypted noisy vector $\mathbf{\hat{V}}$  as follows \begin{gather}\boldsymbol{\mathcal{V}}[i]= \mathbf{\hat{V}}[i] \oplus Enc_{pk}(M[i])\end{gather} and sends it to the CSP \item CSP decrypts the masked encrypted noisy vector as \begin{gather*}\mathcal{V}[i]=Dec_{sk}(\mathbf{\hat{V}}[i]), i \in [|V|]\end{gather*} \item Next, the following garbled circuit is evaluated which
    \begin{enumerate}[label=(\alph*)]\item takes noisy masked  vector $\mathcal{V}$ as an input from the CSP \item takes mask $M$ as the input from AS  \item removes the mask from  $\mathcal{V}$  as \begin{gather*} \hat{V}[i]=\mathcal{V}[i]-M[i], i \in [|V|]\end{gather*}  \item computes the maximum element over  $\hat{V}$ and returns $\arg\max{\hat{V}[i])}$
    \end{enumerate} \end{enumerate}\end{enumerate}
    \begin{comment}
\subsection{Query Answering}
In this section, we will show how can we use the aforementioned primitives to answer an arbitrary counting query. Consider a query $q$ given by $q(\mathcal{D})=\sum_{i=1}^k c_i\phi_i(\mathcal{D})$. Let $Attribute(\phi)$ denote the set of all attributes in $\mathcal{A}$ that appear in the boolean condition $\phi$. For e.g., if $\phi = \big((\mathcal{A}_1==v_1) \land \mathcal{A}_2==v_2) \vee \mathcal{A}_3==v_3 \big)$, then  we have $Attribute(\phi)=\{\mathcal{A}_1, \mathcal{A}_2,\mathcal{A}_3\}$.\begin{enumerate}\item Firstly, The AS finds the attribute set of all the boolean conditions $\phi_i, i \in [k]$, i.e., it computes $A^* =\bigcup_{i=1}^k Attribute(\phi_i)$. 
\item Next the AS performs a project transformation on inputs attribute set $A^*$ and the entire encrypted database $\boldsymbol{\tilde{\mathcal{D}}}$. 
\item Let $A^*= \{A^*_1,A^*_2,\ldots,A^*_t\}, t \leq k$. Next the AS constructs the encrypted one-hot-coding over the entire $t$-dimension 'attribute' $\mathcal{A}^*=\prod_{i=1}^t A^*_i$ by $(t-1)$ iterative application of the cross product transformation. 
\item Note that the result of the preceding step is a $m\times 1$ table where the $i^{th} , i \in [m]$ record corresponds to the encrypted one-hot-coding over the entire $t$-dimension domain space of $\mathcal{A}^*$ of data owner $DO_i$. The AS computes the vector $\mathbf{\hat{C}}$ (an input to the filter transformation, sec 1.1) as follows. Clearly length of $\mathbf{\hat{C}}$ is $s^*=\prod_{i=1}^ts^*_i$ where $s^*_i=|dom(A^*_i)|$ and it is initialized to be encrypted zero-vector. For each boolean condition $\phi_i, i \in [k]$, the AS fills in all such entries  $\mathbf{\hat{C}}[j]=Enc_{pk}(c_i), j \in [s^*]$ such that $v_j$ satisfies $\phi_i$ where $v_j \in dom(\mathcal{A}^*) $. 
\item This is followed by performing the count transformation by the AS to obtain the encrypted count $\boldsymbol{Ct}$.
\item Next, the AS and the CSP jointly compute the appropriate laplace noise, $\boldsymbol{\eta}$ via the LaplaceMechanism primitive and adds it to $\boldsymbol{Ct}$ to obtain $\boldsymbol{\bar{Ct}}=\boldsymbol{Ct}+\boldsymbol{\eta}$. Note that the sensitivity of the query $q$ is $ \Delta_q=\max\{c_i\}$.
\item Finally CSP decrypts $\boldsymbol{\bar{Ct}}$ to reveal the plaintext noisy count.   
\end{enumerate}


\begin{comment}\begin{exmp}
\textit{Query 1:} Count the number of records satisfying $(Age=50 \wedge Gender=Male)\vee Salary=500K$\\
\textit{Crypt$\epsilon$ Program:}
Let $\phi=(Age=50 \wedge Gender=Male)\vee Salary=500K$ and let $\mathcal{I}(\phi)$ denote the set of indices in the one-hot-encoding of  the 3-dimensional attribute $ A'=Age \times Gender \times Salary$ that satisfies $\phi$.
\begin{enumerate}\item $\mathbf{\tilde{T}}_1$=AS.Project($\boldsymbol{\tilde{\mathcal{D}}}$, $Age, Gender, Salary $)  \item  $\mathbf{\tilde{T}}_2$ = AS.CrossProduct($\mathbf{\tilde{T}}_1$, Age, Gender)\item $\mathbf{\tilde{T}}_3$ = AS.CrossProduct($\mathbf{\tilde{T}}_2$, Age $\times$ Gender, Salary)\item  $ \textit{for } i \in [s_{A'}]$\\$\mathbf{C}[i]=
  \begin{cases} 
   labEnc_{pk}(1),  & \text{if } i \in \mathcal{I}(\phi)  \\
   labEnc_{pk}(0)      & \text{otherwise } 
  \end{cases}$ 
  \item $\tilde{\mathbf{T}}_4=AS.Filter(\tilde{\mathbf{T}}_3, \mathbf{C})$\item $\mathbf{Ct}=AS.Count(\tilde{\mathbf{T}}_4)$\item $\bar{\mathbf{Ct}}=AS-CSP.LaplaceMechanism(1,\epsilon) \oplus \mathbf{Ct}$\item$ \bar{Ct}=CSP.Dec_{sk}(\tilde{\mathbf{Ct}})$\item Return $\bar{Ct}$\end{enumerate}
\end{exmp}
\begin{exmp}
\textit{Query 2 :} Sum of the salaries of all the departments who have more than 50 employees group by department. \\
\textit{Crypt$\epsilon$ Program:}
\arc{Need to do}
\end{exmp}
\begin{exmp}
\textit{Query 3:} Partition the age in ranges of 10 bins.  Design a $\epsilon$-differentially private mechanism to report the age-bracket with the maximum count. \\
\textit{Crypt$\epsilon$ Program:}
\begin{enumerate}\item $\boldsymbol{\tilde{T}}_1$=AS.Project($\boldsymbol{\tilde{\mathcal{D}}}$, $Age$) \item for i = 1:10\\ $\mathbf{\hat{C}}[j]= \begin{cases} 
   labEnc_{pk}(1),  & \text{if } j \in [(i-1)\cdot 10,i\cdot 10-1] \\
   labEnc_{pk}(0)      & \text{otherwise } 
  \end{cases}$\\$\thinspace \thinspace \thinspace \thinspace \thinspace\thinspace\thinspace\thinspace\thinspace\thinspace\thinspace\thinspace\thinspace\boldsymbol{\tilde{T}}_2[i]$=AS.Filter$(\boldsymbol{\tilde{T}}_1$, $\mathbf{\hat{C}})$ \\ end for \item for i=1:100\\$\mathbf{V}[i]=$AS.Count($\boldsymbol{\tilde{T}}_2[i]$)\\end for \item M=AS-CSP.NoisyMax$(\mathbf{V})$\item Return M\end{enumerate}
\end{exmp}

\end{comment}


\subsection{Program Examples}
Consider a database schema $\langle Age$, $Gender$, $NativeCountry$, $Department$, $Salary\rangle$. We show several \system program examples in Table~\ref{tab:programexamples} over this database.
\begin{table*}[t]
\caption{Examples of \system Program}\label{tab:programexamples}
\scalebox{0.67}{
\begin{tabular}{|l|l|}
\hline
 {\bf \system Program} & {\bf Description} \\ \hline \hline
P1:  $\lap_{\epsilon,1}(\countagg(\filter_{Age\in [50,60]}(\project_{Age}(\encD))))$ &  Counts the number of records satisfying $Age \in [50,60]$      \\ \hline
P2: $\noisymax^5_{\epsilon,1}(\groupbystar_{Age}(\encD))$               &       Outputs the 5 most frequent age values. \\ \hline
P3: $\lap_{\epsilon,2}(\groupbystar_{Age\times Gender}(\project_{Age \times Gender}(\crossproduct_{Age,Gender\rightarrow{Age \times Gender}}(\encD))))$     &       Outputs the marginal over the attributes $Age$ and $Gender$. \\ \hline
P4: $\lap_{\epsilon,2}(\groupbystar_{Age\times Gender}(\filter_{NativeCountry=Mexico}(\project_{Age\times Gender, NativeCountry}(\crossproduct_{Age,Gender\rightarrow{Age \times Gender}}(\encD)))))$                          & Outputs the marginal over $Age$ and $Gender$ for Mexican employees.     \\ \hline
P5: $\lap_{\epsilon,1}(\countagg(\filter_{Age=30 \wedge Gender=Male \wedge NativeCountry=Mexico}(\project_{Age,Gender,NativeCountry}(\encD))))$                                       &   Counts the number of male employees of Mexico having age 30.    \\ \hline
P6:   $\lap_{\epsilon,2}(\countdistinct(\groupbystar_{Age}(\filter_{Gender=Male}(\project_{Age , Gender}(\encD)))))$ & Counts the number of distinct age values for the male employees.       \\ \hline
P7:  $\lap_{\epsilon,2}(\countagg(\filter_{Count\in[10,m]}(\groupby_{Age}(\project_{Age}(\encD)))))$
                                     & Counts   the number of  age values having at least 10 records.   \\ \hline
\end{tabular}}
\end{table*}


\section{Implementation}\label{sec:implementaion}
%To demonstrate the use of Crypt$\epsilon$ primitives let us look at the following example.
\label{implementation}
In this section we describe the implementation of Crypt$\epsilon$. First we discuss our novel proposed technique for extending the $labMult$ operation of \textsf{labHE} to support $n > 0$ multiplicands. Then we describe the implementations for each primitive. Last, we use the example programs from the previous section to illustrate the performance of \system.

\subsection{\textbf{General n-way Multiplication for \textsf{labHE}}}\label{genlab}
In addition to the operations supported by a \textsf{LHE}  scheme, \textsf{labHE} supports multiplication of two labeled ciphers. 
\\ $\bullet \textbf{labMult}(\mathbf{c}_1,\mathbf{c}_2)$ -
On input two labeled ciphers $\mathbf{c}_1=(a_1,d_1)$ and $\mathbf{c}_2=(a_2,d_2)$, it computes a "multiplication" ciphertext $\mathbf{e}=labMult(\mathbf{c_1,c_2})=Enc_{pk}(a_1,a_2)\oplus cMult(d_1,a_2) \oplus cMult(d_2,a_1)$. Observe that $Dec_{sk}(\mathbf{e})=m_1\cdot m_2 -b_1 \cdot b_2$.\\
 $\bullet \textbf{labMultDec}_{sk}(d_1,d_2,\mathbf{e})$ - On input two encrypted masks $d_1,d_2$ of two labHE ciphers $\mathbf{c_1},\mathbf{c_2}$ and the output $\mathbf{e}$ of $labMult(\mathbf{c_1},\mathbf{c_2})$, it decrypts the product as $m_3=Dec_{sk}(\mathbf{e})+Dec_{sk}(d_1)\cdot Dec_{sk}(d_2) = m_1\cdot m_2$ .   \\
In this paper we propose an efficient way of extending the $labMult$ operation for a $n$-way multiplication.
\begin{algorithm}
\caption{$genLabMult$ - generate label for $labMult$}
\begin{algorithmic}[1]
\STATEx
\textbf{Input}: $\mathbf{c_1}=labEnc_{pk}(m_1)=(a_1,d_1)$  where\STATEx $\hspace{0.9cm} a_1= m_1-b_1, d_1=Enc_{pk}(b_1)$\STATEx $\hspace{0.9cm} \mathbf{c_2}=labEnc_{pk}(m_2)$
\STATEx$\hspace{0.9cm} a_2= m_2-b_2, d_2=Enc_{pk}(b_2)$
\STATEx \textbf{Output}: $\mathbf{e}=labEnc_{pk}(m_1\cdot m_2)$ 
\STATEx \textbf{\textsf{AS}:} \STATE Computes $\textbf{e}'=labMult(\mathbf{c_1,c_2}) \oplus Enc_{pk}(r)$ where $r$ is a random mask \STATE Sends $\mathbf{e'},\mathbf{c_1},\mathbf{c_2}$ to \textsf{CSP}
\STATEx \textbf{\textsf{CSP}:}
\STATE Decrypts $\mathbf{e'}$, to get $Dec_{sk}(\mathbf{e}')=m_1\cdot m_2 -b_2\cdot b_1 + r$
\STATE Computes $b_1 \cdot b_2$ from $d_1$ and $d_2$.
\STATE Removes $b_1\cdot b_2$ from $e'$ to compute $e''=m_1\cdot m_2+r$
\STATE Picks a seed $\sigma'$ and label $\tau'$
\STATE Computes $\bar{a}=e''-b'=m_1\cdot m_2 +r -b'$, $b'=\mathcal{F}(\sigma',\tau')$ and $d'=Enc_{pk}(b')$
\STATE Send $\bar{e}=(\bar{a},d')$ to \textsf{AS}\STATEx \textbf{\textsf{AS}:}
\STATE Computes true cipher $\mathbf{e}=(a',d')$ where $a'=\bar{a}-r=m_1\cdot m_2 - b'$
 \end{algorithmic}
\end{algorithm}
Consider the case where we want to multiply the respective ciphers of  $n$ messages $\{m_1,...m_n\} \in \mathcal{M}^n$. Note that the reason why we can't simply use $labMult$ for a generic $n-$ way multiplication is because, the "multiplication" cipher $\mathbf{e}=labMult(\mathbf{c_1},\mathbf{c_2})$ does not have  a corresponding label. Thus for generalizing the $labMult$ operation for $n$ multiplicands what we have to do is to generate a label and a seed for every intermediary product of two multiplicands. This can be done as shown by Algorithm 1. Note that the mask $r$ protects the value of $m_1\cdot m_2$ from the \textsf{CSP} in step 5. Similarly since $b'$ is not known to the \textsf{AS}, $m_1\cdot m_2$ remains hidden from the \textsf{AS} in step 9. Now with the true \textsf{labHE} cipher $\mathbf{c}=(a',d')$ for the product the \textsf{AS} can compute further multiplications on it.
For a generic $n-way$ multiplication the order of multiplication can be, in fact, parallelized as  shown in Figure ~\ref{genlab-fig} to require a total of $\lceil \log n\rceil$ rounds of communication with the \textsf{CSP}.
\begin{figure}\includegraphics[height=4cm,width=8cm]{kk.png} \caption{ $genLabMult()$ - Batching of multiplicands for \textsf{labHE}} \label{genlab-fig}\end{figure}\\
\subsection{Primitive Implementation}
%Now let us explain the implementation details of the aforementioned Crypt$\epsilon$ primitives.  
In this section we explain the implementation details of two of the aforementioned \system primitives, the rest are covered in appendix section B.\\ (1)\textbf{ \textsf{GroupByCount }}$\groupbystar_A(\mathbf{\tilde{T}})$- The \textsf{GroupByCount} primitive is implemented by Algorithm ~\ref{groupby-imp}. \begin{algorithm}
\small
\caption{\textsf{GroupByCount }$\groupby_A(\mathbf{\tilde{T}})$}
\begin{algorithmic}[1]
\STATEx
\textbf{Input}: $\mathbf{\tilde{T}}$
\STATEx \textbf{Output}: $\tilde{\encV}$
\STATEx \textbf{\textsf{AS}:} \STATE Computes $\mathbf{V}=\groupbystar_{A}(\encT)$.
\STATE Masks the encrypted histogram $\mathbf{V}$ for attribute $A$ as follows \begin{gather*}\boldsymbol{\mathcal{V}}[i]= \mathbf{V}[i] \oplus labEnc_{pk}(M[i])\\M[i] \in_R [m], i \in [|V|]\end{gather*}
\STATE Sends $\boldsymbol{\mathcal{V}}$ to \textsf{CSP}.
\STATEx \textbf{\textsf{CSP}:}
\STATE Decrypts  $\boldsymbol{\mathcal{V}}$ as $\mathcal{V}[i]=labDec_{sk}(\boldsymbol{\mathcal{V}})$.\STATE Converts each entry of $\mathcal{V}$ to its corresponding one-hot-coding and encrypts it. \begin{gather*}\boldsymbol{\tilde{\mathcal{V}}}[i]=labEnc_{pk}(\tilde{\mathcal{V}[i]})\end{gather*} 
\STATE Sends $\boldsymbol{\tilde{\mathcal{V}}}$ to \textsf{AS}.
\STATEx \textbf{\textsf{AS}}:
\STATE  Rotates every entry by its corresponding mask value to obtain the desired  encrypted one-hot-coding $\boldsymbol{\tilde{V}}[i]$. \begin{gather*}\boldsymbol{\tilde{V}}[i]=RightRotate(\boldsymbol{\tilde{\mathcal{V}}},M[i])\end{gather*} 
 \end{algorithmic} \label{groupby-imp}
\end{algorithm} 
In the first step the \textsf{AS} uses the $\textsf{GroupByCount}^*$ primitive to generate the encrypted histogram $\encV$ for attribute $A$. Note that since each entry of $\mathbf{V}$ is a count of records, its value ranges from $\{0,...,m\}$. The \textsf{AS} then masks $\encV$ (step 2) and sends it to the \textsf{CSP}. The purpose of this mask is to hide the true histogram from the \textsf{CSP}. Next the \textsf{CSP} generates the encypted one-hot-coding representation for this masked histogram $\boldsymbol{\tilde{\mathcal{V}}}$ (steps 4-5) and returns it back to the \textsf{AS}. Notice that each entry of $\boldsymbol{\tilde{\mathcal{V}}}$ is a $m$-lengthed vector. The \textsf{AS} can simply rotate $\boldsymbol{\tilde{\mathcal{V}}}[i], i \in [|V|]$ by its respective mask value $M[i]$ (step 7) and get back the true encrypted histogram in one-hot-coding $\tilde{\encV}$.
Note that the \textsf{GroupByCount} primitive could have an alternative implementation using a Yao's garbled circuit that takes an input the encrypted vector and outputs the corresponding one-hot-coding representation. However this would require the circuit to decrypt and re-encrypt $O(m)$ data inside it which would be computationally heavy for larger values of $m$. 
 \\ (2)\textbf{ \textsf{Laplace }}$\lap_{\epsilon,\Delta}(\mathbf{V})$ -  
\begin{algorithm}
\small
\caption{\textsf{Laplace }$\lap_{\epsilon,\Delta}(\mathbf{V})$}
\begin{algorithmic}[1]
\STATEx
\textbf{Input}: $\encV$
\STATEx \textbf{Output}: $\hat{V}$
\STATEx \textbf{\textsf{AS}:} \STATE Generates a noisy vector $\hat{\encV}$  as \begin{gather*}\hat{\mathbf{V}}[i] = \mathbf{V}[i]\oplus labEnc_{pk}(\eta[i]),\\ \eta \sim [Lap(\frac{1}{\epsilon})]^{|V|}, i \in [|V|] \end{gather*}
\STATE Sends $\hat{\mathbb{\mathcal{V}}}$  to \textsf{CSP}
\STATEx \textbf{\textsf{CSP}:}
\STATE Decrypts $\mathbf{\hat{\mathcal{V}}}$ to get $\hat{\mathcal{V}}[i]=labDec_{sk}(\mathbf{\hat{\mathcal{V}}}[i]), i \in [|V|]$
\STATE Generates a the final noisy vector $\hat{V}$ as follows 
\begin{gather*} \hat{V}[i]=\hat{\mathcal{V}}[i]+\eta'[i], i \in [|V|], \eta' \sim [Lap(\frac{1}{\epsilon})]^{|\hat{V}|} \end{gather*}
\STATE Returns $\hat{V}$ to \textsf{AS}
 \end{algorithmic} \label{lap}
\end{algorithm} The implementation for the \textsf{Laplace} primitive is given by Algorithm ~\ref{lap}. Recall that both \textsf{AS} and \textsf{CSP} have to add Laplace noise to the output in Crypt$\epsilon$. Hence the \textsf{Laplace} primitive has two phase. In the first phase,  the \textsf{AS} adds an instance of encrypted Laplace noise to the encrypted input (step 1 in Algorithm ~\ref{lap}) to generate $\mathbf{\hat{\mathcal{V}}}$. This acts as the input to the second phase which is executed by the \textsf{CSP}. Here the \textsf{CSP} decrypts $\mathbf{\hat{\mathcal{V}}}$ and adds a second instance of the Laplace noise to generate the final noisy output $\hat{V}$ in the clear (steps 3-4). The \textsf{Laplace} primitive with an encrypted scalar $\encC$ as the input is implemented in a similar way. 

\subsection{Classification of \system Programs}
Crypt$\epsilon$ programs can be classified into three classes based on the number and type of interaction required between the \textsf{AS} and the \textsf{CSP}.  \\
(1)\stitle{ \textbf{Class I - Single Decrypt Interaction Programs:}}\\
Recall that the output of all the transformation primitives are encrypted.  Since  the \textsf{CSP} has exclusive access to the secret key, it is the only entity in the Crypt$\epsilon$ setting capable of decryption. Thus for releasing any result (albeit noisy) in the clear, we need to interact at least once with the \textsf{CSP} so that it can decrypt the encrypted noisy answer. Crypt$\epsilon$ supports this type of interactions via the two measurement primitives. Some Crypt$\epsilon$ programs require only this one round of interaction at the very end to release the noisy output. All other transformations can be performed by the \textsf{AS} via homomorphic operations on the encrypted data records. Typically these programs are counting queries on a single attribute or noisy max on a single attribute. Examples of this type of programs are P1 and P2 from Table \ref{tab:programexamples}.\\
(2)\stitle{ \textbf{Class II : \textsf{LabHE} Multiplication Interaction Programs-}}\\
Recall that labeled homomorphic encryption allows multiplication of two ciphers. Generalization to a $n$-multiplicand, $n > 2$ case can be done according to the protocol described in section \ref{genlab}. However it requires intermediate interactions with the \textsf{CSP}. Thus all Crypt$\epsilon$ programs that require multiplication of more than two ciphers need interaction with the \textsf{CSP}. 
All programs with more than three attributes in its boolean predicate would thus fall under this class. P3, P4 and P5 from table 3 fall in this class of \system programs.\\
(3)\stitle{ \textbf{Class III : Other Interaction Programs-}}\\
 The \textsf{GroupBy} primitive requires an intermediate interaction with the \textsf{CSP} (for generating the encrypted one-hot-coding). The \textsf{CountDistinct} primitive also uses a Yao's garbled circuit (see section C) and hence requires  interaction with the \textsf{CDP}. This in addition to the interaction required for decrypting the noisy answer (as explained above). Thus any program with the \textsf{GroupBy} or \textsf{CountDistinct} primitive requires two rounds of interaction in the least. P6 and P7 from Table \ref{tab:programexamples} are examples of this class of \system programs. 

\section{Optimizations}
In this section we propose two optimizations that leverages on the fact that we allow a differentially private view of our private database. This DP based optimizations help in reducing the performance cost significantly which would have been impossible with just secure multi-party computation.

\begin{enumerate}
\item Index- Consider a conjunctive query predicate $A_1==v_1 \wedge A_2==v_2 \wedge A_3==v_3 $. If it so happens that the number of records satisfying $A_1==v_1$ is very low compared to the total number of records, i.e., $ct_{A_1,v_1} << m$, then if a selection operation is performed alone on attribute $A_1$ then the size of the dataset to be considered for the subsequent clause $A_2==v_2 \wedge A_3==v_3$ reduces to only $ct_{A1,v1}$ as opposed to the whole dataset (size $m$). Our index optimization leverages on this idea - we create an index by sorting the database on an attribute of choice $A$. There are two heuristics that can be considered for selecting this indexing attribute $A$ - firstly $A$ should be a very frequently queried upon attribute. This is intuitive as this would mean a larger fraction of the queries will benefit from this optimization. Secondly if $\{v_1,...v_n\} \subset dom(A)$ is the set of most frequently queried values for attribute $A$, then $ct_{A,v_i}, i \in [n] << m$. This would ensure that the first selection operator performed alone on $A$ will filter out majority of the records and reduce the  dataset size to be considered for the subsequent predicate. The optimization is implemented via a garbled circuit that \begin{enumerate}\item takes the entire database $\boldsymbol{\mathcal{\tilde{D}}}$ as an input and the attribute $A$ as an input from the AS.
\item takes the secret key $sk$ as an input from  the CSP \item Decrypts $\boldsymbol{\mathcal{\tilde{D}}}$ \item Sorts the decrypted database on $A$, i.e., the first $ct_{A,1}$ rows are the ones with value $v_1$ for attribute $A$, the next $ct_{A,2}$ are  the records with value $v_2$ for attribute $A$ and so on. \item  re-encrypts the sorted database \item Construct a $s_A$ lengthed vector $\hat{V}$ such that $\bar{\hat{V}}[i]=ct_{A,i}+\eta_i, i \in [s_A]$ where $\eta_i$ is a random laplace noise drawn from the distribution $Lap(\frac{1}{\epsilon'}), \epsilon' < \epsilon $ \item Return $\bar{\hat{V}}$ and sorted $\boldsymbol{\mathcal{\tilde{D}}}_{sort}$\end{enumerate}
Next the AS computes a CDF, $\bar{\hat{\mathcal{C}}}$ over the noisy counts in $\bar{\hat{V}}$ using inference based on the Non Negative Least Squares technique \arc{[add ref]}.
For answering queries of the form $\phi=A_1==v_1\wedge \ldots \wedge A==v \wedge \ldots \wedge A_n==v_n$, ideally we just need to compute for $A_1==v_1\wedge \ldots \wedge A_n==v_n$ on $ct_{A,v}$ number of records starting from position $\sum_{i=1}^{i=v-1}ct_{A,i}$ of $\boldsymbol{\mathcal{\tilde{D}}}_{sort}$. 

However the AS has access only to the noisy CDF over $ct_{A,i}$. Note that when $\bar{i}_{start}=\bar{\hat{\mathcal{C}}}[v-1] < \sum_{i=1}^{i=v-1}ct_{A,i}$ and $\bar{i}_{end}= \bar{\hat{\mathcal{C}}}[v-1] > i_{start}+ct_{A,v}$, i.e., the indices computed from the noisy values  saddle over the true records satisfying $A==v$, then although we end up loosing in performance a bit, we are still guaranteed to compute the exact non-noisy count for records satisfying $\phi$. 

In all other cases, we end up disregarding some of the records that satisfy $A==v$, some of these rejected records in fact might additionally satisfy $A_1==v_1 \wedge \ldots \wedge A_n==v_n$. Thus we might get inaccurate answer for query predicate $\phi$ (note that here we are talking about the encrypted true count of the given query predicate that is computed by the AS via a series of transformations before applying the LaplaceMechanism primitive).  An effective heuristic to tackle this can be to compensate for the expected laplacian error as follows  $\bar{i}_{start}= \bar{\hat{\mathcal{C}}}[v-1]-\frac{2}{\epsilon}$ and $\bar{i}_{end}=\bar{\hat{\mathcal{C}}}[v]+\frac{2}{\epsilon}$. Also note that answering differentially private  range queries   on attribute $A$ can also be directly done from the noisy CDF $\bar{\hat{\mathcal{C}}}$
 %\item GroupBy*($\mathbf{V},sk$)- This primitive is an extension of the previous GroupBy transformation. 
 \item Range Tree- Range queries constitute a very popular category of queries for typical databases and range trees are a popular data structure constructed to speed up range query answering. A 1-dimensional range tree for an attribute $A$ is an ordered data structure such that the leaf nodes correspond to the individual counts $ct_{A,i}$, $i$ increasing from left to right while the parent node stores the sum of the counts of its children. Hence an useful optimization for our setting can be pre-computing the range tree for some attributes. In Crypt$\epsilon$ we construct a noisy range tree for some of the attribute. The sensitivity for each such noisy count is $\log k$ where $k$ is the domain size of the attribute. For answering any arbitrary range query, we need to access at most $2\log k$ nodes of the range tree. Thus to answer all possible range queries for the given attribute, the total squared error accumulated is $O(\frac{k^2\log }{\epsilon^2})$. In contrast for the naive case, we would have incurred error $O()$. Hence this range tree optimization not only gives us a huge performance boost but also results in better answer accuracies. \end{enumerate}
\paragraph*{Optimized Crypt$\epsilon$ programs}
Let us reconsider the example programs covered in section. Both program 1 and program 2 can be optimized by constructing a range tree over attribute $Age$. Programs  4,5 and 6 on the other hand can be improved by the differentially private index over attribute $Age$.
\section{\system Security Sketch}



In this section we provide a sketch of the security proof in the semi-honest model.
Our proof will follow the well established simulation argument \cite{Oded}. Let $P$ be a program that is to be run on dataset $\mathcal{D}$ with privacy parameter $\epsilon$
 and let $P^{CDP}(\mathcal{D},\epsilon)$ denote the random
variable (rv) that corresponds to the output of running $P$ on the 
\textsf{CDP} model.

\begin{theorem}\label{thm:security}
\rm
Let $\Pi$ be the protocol corresponding to the execution of program $P$ in \system. The
views and outputs of \textsf{AS} and \textsf{CSP} are denoted follows:
\[
\begin{array}{cc}
View_1^{\Pi}(P,\mathcal{D},\epsilon) & Output_1^{\Pi}(P,\mathcal{D},\epsilon) \\
View_2^{\Pi}(P,\mathcal{D},\epsilon) & Output_2^{\Pi}(P,\mathcal{D},\epsilon) \\
\end{array}
\]
There exists Probabilistic Polynomial Time (PPT) simulators $Sim_1$
and $Sim_2$ such that:
\squishlist
\item $Sim_1 (P^{CDP}(\mathcal{D},\epsilon))$ is computationally indistinguishable ($\equiv_c$)
from $(View_1^{\Pi}(P,\mathcal{D},\epsilon),Output^{\Pi}(P,\mathcal{D},\epsilon))$, and
\item $Sim_2 (P^{CDP}(\mathcal{D},\epsilon))$ is $\equiv_c$
to $(View_2^{\Pi}(P,\mathcal{D},\epsilon),Output^{\Pi}(P,\mathcal{D},\epsilon))$.
\squishend
$Output^{\Pi}(P,\mathcal{D},\epsilon))$ is the combined output of the two
parties\footnote{Notice that the simulators are passed a random variable $P^{CDP}(\mathcal{D},\epsilon))$,
which essentially means that simulator is given a random draw from this distribution.}
\end{theorem}
The proof of this is presented in Appendix A. The statement of the theorem given above assumes
that \textsf{AS} and \textsf{CSP} do not collude with the users (the
data owners). However, if \textsf{AS} colludes with a subset of the
users, which essentially means $Sim_1$ ($Sim_2$)has to be given the data
corresponding to these users as additional parameter. This provides no
complications in the proof at all (see the proof in \cite{LReg}). 
 Since every program $P$ expressible using \system primitives satisfies differential privacy, it follows from Theorem~\ref{thm:security} that every execution of \system satisfies computational differential privacy.
 \noindent
\begin{corollary} 
	Protocol $\Pi$ satisfies computational differential privacy under the \textsf{SIM-CDP} notion \cite{CDP}.
\end{corollary}

\section{Experimental Evaluation}
In this section we experimentally evaluate  Crypt$\epsilon$ to assess its practical utility  based on two parameters, the accuracy and the performance of the Crypt$\epsilon$ programs. Specifically the experiments were designed to answer the following questions:
\begin{enumerate}\item Does Crypt$\epsilon$ programs have constant error bounds depending only on the privacy parameter $\epsilon$  which is at least $O(\sqrt{m})$ times lower than that for the corresponding state-of-the-art LDP implementation? \item Is the program execution timings for Crypt$\epsilon$ practical? \item Does the proposed DP- optimizations provide substantial performance improvement over the base case Crypt$\epsilon$ implementation? \end{enumerate}

\paragraph{Methodology:} To answer the aforementioned questions we ran the experiments on the 7 Crypt$\epsilon$ programs previously outlined in section 5. We ran our experiments on a dataset which is generated from the UCI Adult dataset by randomly sampling 1000 records. The experiment numbers are reported as the mean value after 10 repetitions.
\subsection{Accuracy}
\begin{figure*}
    \centering
    \begin{subfigure}[b]{0.2\textwidth}
        \includegraphics[width=5cm,height=5cm]{test1_1.pdf}
        \caption{ Program 1}
        \label{fig:gull}
    \end{subfigure}\quad \qquad\quad \\%
    ~ %add desired spacing between images, e. g. ~, \quad, \qquad etc.
      %(or a blank line to force the subfigure onto a new line)
    \begin{subfigure}[b]{0.3\textwidth}
       \qquad \includegraphics[width=5cm,height=5cm]{test2.pdf}
        \caption{ Program 2}
        \label{fig:tiger}
    \end{subfigure}
    ~ %add desired spacing between images, e. g. ~, \quad, \qquad etc.
      %(or a blank line to force the subfigure onto a new line)
    \begin{subfigure}[b]{0.3\textwidth}
    \qquad    \includegraphics[width=5cm,height=5cm]{test3.pdf}
        \caption{Program 3}
        \label{fig:mouse}\end{subfigure}
          \begin{subfigure}[b]{0.3\textwidth}
    \qquad    \includegraphics[width=5cm,height=5cm]{test4.pdf}
        \caption{Program 4}
        \label{fig:mouse}\end{subfigure}
          \begin{subfigure}[b]{0.3\textwidth}
    \qquad    \includegraphics[width=5cm,height=5cm]{test55.pdf}
        \caption{Program 5}
        \label{fig:mouse}\end{subfigure}
          \begin{subfigure}[b]{0.3\textwidth}
    \qquad    \includegraphics[width=5cm,height=5cm]{test66.pdf}
        \caption{Program 6}
        \label{fig:mouse}\end{subfigure}
          \begin{subfigure}[b]{0.3\textwidth}
    \qquad    \includegraphics[width=5cm,height=5cm]{test7.pdf}
        \caption{Program 7}
        \label{fig:mouse}
    \end{subfigure}
%    \caption{a) External alignment is the neighborhood of read $X$ in the full genome and $\lambda$ is the \% of $X$ that correctly identifies it. (b) Internal alignment is the index-to-index alignment within the neighborhood of $X$ and $\sigma=avg_i\frac{TBScore_{opt}-TBScore_i}{TBScore_{opt}}$ where $TBScore_{i}=\sum_{[i,j] \in \textit{Traceback path}}\mathbf{M}[i,j]$. 
%   (c) $\rho$ is the number of reads required to cover a single index such that Pr( SNPs are aligned correctly in $Y) > 0.95$}
   \caption{Accuracy Analysis}
\end{figure*}
\subsection*{Crypt$\epsilon$ Program 1}
Program 1 counts the number of records having age in range [50,60].  
\\\textbf{\textsf{LDP} Competitor} - The competing \textsf{LDP} implementation uses the frequency oracle of \cite{LDP1}. 
\\\textbf{Error Measure} - The error measure used is  $Error = |C-\hat{C}|$ where $C$ is the true count and $\hat{C}$ is the noisy differentially private output. We report the mean error observed over 10 repetitions. \\
\textbf{Optimization}- For this program, the optimized implementation uses the range tree to answer the counting query.
\\\textbf{Observations} - For Program 1 we report our experimental results in Figure 2a). The first observation is that the error for the base case Crypt$\epsilon$ implementation is approximately $\frac{2}{\epsilon}$ (error=$17.60$ for $\epsilon=0.1$, mean error = $0.1$ for $\epsilon=1.9$). This is in cohorts with our expectation as we add two instances of Laplace noise at the \textsf{AS} and the \textsf{CSP} and s.t.d of $Lap(\frac{1}{\epsilon})$ is given by $\frac{1}{\epsilon}$. In contrast, the error corresponding to the $\textsf{LDP}$ implementation is at least $10^2 >\sqrt{m}=  \sqrt{1000} = 31.62$ times worse. For e.g., the error for $\textsf{LDP}$ is $967 \times$  higher than that of Crypt$\epsilon$ for $\epsilon=1.9$. Another observation is that the accuracy of the range tree based Crypt$\epsilon$ implementation is poorer as compared to that of the base case. For e.g., for $\epsilon=0.3$, the error for the optimized implementation is $25.2$ while that for the base case is $5$. This is so because the sensitivity of the range tree is $\log k$ where $k$ is the number of leaves and a range query can take up to $2 \log k$ nodes for answering. Thus for a single query, the base case implementation will give better accuracy results. The accuracy gain for range trees kicks in for answering multiple range queries as showcased in Fig .  \arc{TO-DO New exp to showcase range tree accuracy gain}
\subsection*{Crypt$\epsilon$ Program 2}
Program 2 counts the top 5 most frequent age values.
\\\textbf{\textsf{LDP} Competitor} - The competing \textsf{LDP} implementation uses the frequency oracle of \cite{LDP1}. 
\\\textbf{Error Measure} - The error measure used is the F1 score of the classification.  \\
\textbf{Optimization}- The optimized implementation reads of the values of all the leaves in  the range tree (each leaf corresponds to the count of a single age value) and returns age values with the top 5 counts.
\\\textbf{Observations} - The experimental results of  Program  2 are reported in  Figure 2b. 

\subsection*{Crypt$\epsilon$ Program 3}
Program 3 reports the marginal on attributes \textsf{Age} and \textsf{Gender}.  
\\\textbf{\textsf{LDP} Competitor} - For the  \textsf{LDP} implementation, we construct a frequency oracle based on \cite{LDP1} over the marginal attribute $\textsf{Age}\times\textsf{Gender}$. 
\\\textbf{Error Measure} - For Program 3 we use the L1-norm error metric $ Error=\sum_{i}|C[i]-\hat{C}[i]|, i \in [200]$ where $C$ is the true marginal and $\hat{C}$ is the noisy one. 
\\\textbf{Observations} - Figure 2c reports the experimental results for Program 3. Attribute $Age$ has domain size $100$ while attribute $Gender$ is of size $2$. Hence $Age\times Gender$ is of size $200$. We observe that the errors for Crypt$\epsilon$ is approximately $\frac{200}{\epsilon}$(error=$2127$ for $\epsilon=0.1$, error=$641$ for $\epsilon=0.3$) which is the expected value. As for the \textsf{LDP} implementation, the error is way worse, for e.g. for $\epsilon=1.9$ its error is $32 \approx \sqrt{1000} \times$ higher than that of Crypt$\epsilon$.

\subsection*{Crypt$\epsilon$ Program 4}
Program 4 reports the marginal on attributes \textsf{Age} and \textsf{Gender} with \textsf{NativeCountry} Mexico.  
\\\textbf{\textsf{LDP} Competitor} -  The  \textsf{LDP} implementation constructs a frequency oracle based on \cite{LDP1} over the marginal attribute $Age\times Gender\times\textsf{NativeCountry}$. 
\\\textbf{Error Measure} - The error metric used is  the L1-norm  $ Error=\sum_{i}|C[i]-\hat{C}[i]|, i \in [200]$ where $C$ is the true marginal and $\hat{C}$ is the noisy one. \\\textbf{Optimization}- For Program 4 we construct a DP index over the attribute \textsf{NativeCountry} with $\epsilon=1$ and 5 bins.
\\\textbf{Observations} - The experimental results are reported in Fig 2d. The error for the base Crypt$\epsilon$ implementation is as expected approximately around $\frac{200}{\epsilon}$. For e.g., for $\epsilon=0.1$, the error is $2012$. On the other hand, the \textsf{LDP} implementation has much higher error rates. For e.g., for $\epsilon=0.1$ the error for \textsf{LDP} is almost $20\times$ worse than that for Crypt$\epsilon$. We also report the error measures for the optimized Crypt$\epsilon$ implementation. For  this privacy parameter $\epsilon=1.1$ means that $\epsilon=1$ is invested in the DP index construction while $0.1$ is expended in the subsequent count. Observe that the accuracy of the optimized implementation for Crypt$\epsilon$ is much lower than that of the base case, for e.g., the accuracy is $3\times$ lower for $\epsilon=1.9$. However, it is still $4.2 \times$ higher than that of the \textsf{LDP} implementation.  
\subsection*{Crypt$\epsilon$ Program 5}
Program 5 counts the number of natively Mexican male employees of age 30.
\\\textbf{\textsf{LDP} Competitor} -  The  \textsf{LDP} implementation constructs a frequency oracle based on \cite{LDP1} over the marginal attribute $Age\times Gender\times NativeCountry$. 
\\\textbf{Error Measure} - The error metric used is  the L1-norm  $ Error=\sum_{i}|C[i]-\hat{C}[i]|, i \in [200]$ where $C$ is the true marginal and $\hat{C}$ is the noisy one. \\\textbf{Optimization}- The optimized Crypt$\epsilon$ program uses a DP index over the attribute $NativeCountry$ with $\epsilon=1$ and 5 bins.
\\\textbf{Observations} - We report the experimental results for Program 5  in Fig 2e. The error for the base Crypt$\epsilon$ implementation is as expected at most $\frac{2}{\epsilon}$. For e.g., $\epsilon=0.1$ results in error = $21.8$ while for $\epsilon=1.9$ we get an error of only $0.3$. In comparison, the \textsf{LDP} implementation has  at least $25 \times$ higher error values. For e.g., for $\epsilon=0.1$ \textsf{LDP} has $30 \times$ higher error values. The error values for the optimized implementations of Crypt$\epsilon$  is at most $18.8 \times$ worse than that of the base case and at least $2\times$ better than that of the \textsf{LDP} implementation.
\subsection*{Crypt$\epsilon$ Program 6}
Program 6 counts the number of distinct age values for male employees.  
\\\textbf{\textsf{LDP} Competitor} - The competing \textsf{LDP} implementation uses the frequency oracle of \cite{LDP1} $Age\times Gender$ and reports the number of non-zero counts after suitable adjustment for thresholding. 
\\\textbf{Error Measure} - The error measure used is  $Error = |C-\hat{C}|$ where $C$ is the true count and $\hat{C}$ is the noisy differentially private output. \\
\textbf{Optimization}- For this program, the optimized implementation uses a DP index on $Gender$ constructed with $\epsilon=1$ and $5$ bins.
\\\textbf{Observations} - From Figure 2f we observe that the error values for Crypt$\epsilon$ is at least $38.5 \times$ less than that for the \textsf{LDP} implementation. The true answer for the program for our experimental setup happens to be $47$. For $\epsilon=0.1$ the relative error ( $\frac{|C-\hat{C}|}{C}$) for the base case Crypt$\epsilon$ implementation is given by  $0.4$ while \textsf{LDP} has error $1.2$. For the optimized implementation, the error values reduce sharply as we increase $\epsilon$. For e.g., $\epsilon=1.9$ error values of the optimized implementation is only $6\times$ worse than that of the base case. 
\subsection*{Crypt$\epsilon$ Program 7}
Program 7 counts the number of age values with at least 10 records.  
\\\textbf{\textsf{LDP} Competitor} - The competing \textsf{LDP} implementation uses the frequency oracle of \cite{LDP1}. 
\\\textbf{Error Measure} - The error measure used is  $Error = |C-\hat{C}|$ where $C$ is the true count and $\hat{C}$ is the noisy differentially private output. 
\\\textbf{Observations} - Figure 2g reports the results for Program 7. Observe that for $\epsilon=0.3$ the error for Crypt$\epsilon$ is $6.7$ while that for the \textsf{LDP} implementation is $26.2$. The true non-noisy answer for the program for our experimental setup is $38$. Thus the relative error $\frac{|C-\hat{C}|}{C}$ for Crypt$\epsilon$ is $0.18$ as compared to that of $0.69$ for the \textsf{LDP} implementation.  For $\epsilon=1.9$, Crypt$\epsilon$ gives an error of just $0.5$. In contrast,  for \textsf{LDP} we still get an error of $10.6$. 

\subsection{Performance}
In Table 2 we report the computation time of running the aforementioned $7$ Crypt$\epsilon$ programs. For Program 1 we see that the total time taken for execution for the base case Crypt$\epsilon$ implementation is about 0.5 seconds while using the range tree optimization we get a $138\times$ speed up in timings. Note that the time required by the \textsf{AS} becomes almost negligible because it just simply needs to do a memory fetch to read of the answer from the pre-computed range tree instead of computing it from the entire encrypted database. The time for the \textsf{CSP} remains the same in both the cases, it is the decryption cost. For Program 2, we observe a $7\times$ performance improvement with the range tree optimization. Again, even in this case the \textsf{AS} can simply read off the leaf nodes of the range tree instead of computing their counts from the database. The total execution time for Program 3 is roughly 2 hours.  The reason behind this comparatively higher timings as compared to that of the previous two programs is that the \textsf{CrossProduct} primitive requires  multiplication of the ciphers  which is costlier than the addition operator $\bigoplus$. For Program 4, observe that the base case implementation takes around 3.3 hours to run. However the DP index optimization reduces the execution time to less than 5 min giving us a $42\times $ speedup. It is so because, only 11\% of the data records satisfy $NativeCountry$=Mexico. Thus the index reduces the number of records to be considered for the dataset drastically thereby resulting in s huge performance boost. Similarly for Program 5 again the DP index optimization results in a $6.6\times$ speedup with the total running time being less than 6 seconds. For Program 6 the DP index is constructed over attribute $Gender$. While the base case takes about 40 minutes to execute, the optimized code needs only 7 minutes to run thereby reducing the time  $5.7\times$. Finally Program 7 takes about 24 minutes to execute. Thus we see that Crypt$\epsilon$ program execution is very efficient with most programs taking less than an hour. Moreover the range tree and the DP index optimizations are extremely helpful and result in drastic performance boost.
\begin{table}[ht]
\caption{Computation Time Analysis for Crypt$\epsilon$ Programs}
\centering
\begin{tabular}{c c c c c c c}
\toprule
Program &  \multicolumn{3}{c}{Base} & \multicolumn{3}{c}{Optimized} \\ 
 & AS &  CSP & Total & AS & CSP & Total \\ &(s)&(s)&(s)&(s)&(s)&(s)\\ % inserts table %heading
\midrule
1 & 0.49& 0.0027& 0.4927 & 0.0002 &0.0027 & 0.0029 \\
2 &  6.12 & 0.3  &6.42 &0.57&0.32& 0.89\\ %197 the communication rounds
3&  3859.52 & 3661.29 & 7520.81&N/A&N/A &N/A \\4  &8227.89&3761.98&11,989.87&190.57&88.75& 279.32\\5&18.86&17.52&36.38&2.78&2.7&5.48\\6&1910.01&571.11&2481.12&333.91&96.01&429.92\\7&6.35 & 1393.89 & 1400.24 &  N/A & N/A & N/A\\ [1ex]
\bottomrule
\end{tabular}
\label{c}
\end{table}





\section{Related Work}\label{sec:related-short}
\textit{Differential Privacy }- Introduced by Dwork et al. in \cite{Dork}, differential privacy has enjoyed immense attention from both academia and industry in the last decade. Some of the most recent directions in the \textsf{CDP} model include \cite{MVG,Blocki,AHP,DAWA,hist1,hist2,hist3,hist4,hist6,hist7,hist8,A1,A2,A3,A4,A5,A6,A7,A8,u1,u2,MWEM}. The most prominent work in \textsf{LDP} include \cite{LDP1, LDP2, Rappor1,HH,Rappor2,HH2,Cormode, CALM,15,itemset}.
Recently, it has been showed that augmenting the local differential privacy setting by an additional layer of anonymity can improve the privacy
guarantees (or equivalently decrease error bound) \cite{mixnets,Prochlo,amplification}.  An important point to be noted here is that the power of this new model (known as shuffler/mixnet model) lies strictly between that of traditional \textsf{LDP} and \textsf{CDP}. The two-server model of Crypt$\epsilon$ differs from this line of work in three ways. Firstly, Crypt$\epsilon$ results in no reduction in expressibility as compared to that of the \textsf{CDP} model (see Appendix \ref{app:sepldp}). Secondly, the shuffler/mixnet model results in an approximate DP guarantee $(\epsilon\sqrt{\frac{\log\frac{1}{\delta}}{n}},\delta)$ which incurs an expected error of $O(\epsilon\sqrt{\log\frac{1}{\delta}})$.  In practice, $\delta$ has to be at least $\frac{1}{n}$ in order to get meaningful privacy. In contrast Crypt$\epsilon$ achieves the the same order of accuracy guarantees as that of \textsf{CDP}. Finally, the shuffler/mixnet model and \system have certain differences in their respective trust assumptions. For more details see Appendix D.2. %Google's implementation relies on a trusted intermediary shuffler which they implement via trusted hardware enclaves. However truly secure hardware enclaves are notoriously difficult to achieve in practice \cite{Foreshadow}. The mixnet model on the other hand requires a  mix network or mixnet which is a protocol involving several computers that inputs a sequenceof encrypted messages, and outputs a uniformly random permutation of those messages' plaintexts.  Their trust assumption is that at least one of the servers needs to behave honestly. For Crypt$\epsilon$ both the servers are completely untrusted under the constraint that they are non-colluding and follow the protocols semi-honestly.
\\\textit{Two-Server Model} - The two-server model is a popular choice especially for privacy preserving machine learning approaches where typically one of the servers manages the cryptographic primitives while the other handles computation. Examples of this include \cite{Boneh1,Boneh2,Ridge2,Matrix2,secureML,LReg,Ver}. \\\textit{Homomorphic Encryption } - Recently, there has been a surge in  privacy preserving solutions using homomorphic encryptions due to improved primitives. A lot of the aforementioned two-server models employ homomorphic encryption \cite{Boneh1,Boneh2,LReg,Matrix2}.  Additionally it is used in \cite{CryptoDL,CryptoNet,NN, Irene2, grid}.
%A detailed discussion on related work is presented in  Appendix \ref{app:related}.



\section{Future Extensions and Conclusion}
%There are a number of future work directions for Crypt$\epsilon$. 
In this paper we have proposed a new implementation setting for differential privacy that allow us to achieve the constant error bounds of the central differential privacy setting without the need for any trusted server. This is achieved via the assistance of cryptographic primitives specifically linear homomorphic encryption and Yao's garbled circuits. 
\par  One possible extension of the current work can be the development of a
Crypt$\epsilon$ compiler. Recall that currently the data analyst spells out the explicit Crypt$\epsilon$ program  (i.e., the sequence of Crypt$\epsilon$ primitives and their arguments) to the \textsf{AS}. Thus a useful future work can be constructing a Crypt$\epsilon$ compiler that takes as input only a user specified query in a high-level-language and a total privacy budget for the query. The Crypt$\epsilon$
compiler should then be able to formalize an optimized Crypt$\epsilon$ program expressed in terms of Crypt$\epsilon$ primitives with automated sensitivity analysis and subsequent optimal per measurement primitive privacy budget allocation. 
Another very logical future work can be to support a larger class of programs in \system. For e.g., extension of the current functionality of \system to include aggregation operators like sum, median, average etc should be easily achievable. Supporting  multi-table queries like joins in \system based on existing works along the lines of \emph{elastic sensitivity} \cite{elastic} etc would also be an useful extension.  Yet another interesting direction can be enabling learning algorithms on \system.   Comparatively simpler algorithms like linear regression can based on a previous work by  Giacomelli et al. \cite{LReg} which also uses \textsf{LHE} and a two-server model. For this, we need to extend \system with a new primitive that performs matrix multiplications. For more involved models like deep learning, we might need to combine the differential privacy results of \cite{DLDP} with the homomorphic encryption techniques of  CryptoNet \cite{CryptoNet}. As mentioned in section 3.6, an alternative implementation for \system  can be based on secret shares modulo the assumption that both the servers are benefit from learning the differential privacy output. Hence another useful extension might be reimplementing \system with  secret shares. For this, the functionality of the existing primitives would mostly be the same, only the respective implementations will change. 
Yet another direction of exploration can be removing the second server (\textsf{CSP}) altogether and instead capture its functionalities within a trusted execution environment (TEE).

%\bibliographystyle{ACM-Reference-Format}
\bibliographystyle{abbrv}
\bibliography{references.bib}

\appendix

\section{Security Proof}
In this section we present the formal proof for Theorem~\ref{thm:security}.
\begin{proof}
 We have {\it nine primitives} in our paper (see Table~\ref{tab:primitives}).
\begin{itemize}
\item \textsf{NoisyMax} and \textsf{CountDistinct} use
  ``standard'' garbled circuit construction and their security proof
  follows from the proof of these schemes.

\item All other primitives except \textsf{Laplace} essentially use 
homomorphic properties of our encryption scheme and thus there 
security follows from semantic-security of these scheme.

\item The proof for the \textsf{Laplace} primitive is given below.
\end{itemize}
The proof for an entire program $P$ (which is a composition 
of these primitives) follows from the composition theorem~\cite[Section 7.3.1]{Oded}

We will prove the theorem for the \textsf{Laplace} primitive.
In this case the views are as follows (the outputs of the
two parties can simply computed from the views):
\begin{eqnarray*}
View_1^{\Pi}(P,\mathcal{D},\epsilon) & = & (\encD,\eta_1,P(\mathcal{D})+\eta_2+\eta_1) \\
View_2^{\Pi}(P,\mathcal{D},\epsilon) & = & (\eta_2,labEnc_{pk} (P(\mathcal{D})+\eta_1))
\end{eqnarray*}
The random variables $\eta_1$ and $\eta_2$ are random variables
generated according to the Laplace distribution with the requisite
parameters. The simulators $Sim_1 (z_1)$ (where $z_1$ is the random
variable distributed $P^{CDP}(\mathcal{D},\epsilon))$) performs the
following steps:
\begin{itemize}
\item Generates a pair of keys $(pk_1,sk_1)$ for the encryption scheme
  and generates random data set $\mathcal{D}_1$ of the same size as $\mathcal{D}$ and
  encrypts it using $pk_1$ to get $\encD_1$.

\item Generates $\eta'_1$ according to the Laplace distribution.
\end{itemize}
The output of $Sim_1 (z_1)$ is $(\encD_1,\eta'_1,z_1+\eta'_1)$.
Recall that the view of the \textsf{AS} is
$(\encD,\eta_1,P(\mathcal{D})+\eta_2+\eta_1)$.  The computational
indistinguishability of $\encD_1$ and $\encD$ follows from the
semantic security of the encryption scheme. The tuple
$(\eta'_1,z_1+\eta'_1)$ has the same distribution as
$(\eta_1,P(\mathcal{D})+\eta_2+\eta_1)$ and hence the tuples are
computationally indistinguishable.  Therefore, $Sim_1 (z_1)$ is
computational indistingushable from $View_1^{\Pi}(P,\mathcal{D},\epsilon)$.


The simulators $Sim_2 (z_2)$ (where $z_2$ is the random
variable distributed according to $P^{CDP}(\mathcal{D},\epsilon))$) performs the following steps:
\begin{itemize}
\item Generates a pair of keys $(pk_2,sk_2)$ for our encryption scheme.

\item Generates $\eta'_2$ according to the Laplace distribution.
\end{itemize}
The output of $Sim_2 (z_2)$ is $(\eta'_2,labEnc_{pk}(z_2)+\eta'_2)$.
By similar argument as before $Sim_2 (z_2)$ is computationally indistinguishable 
from $View_2^{\Pi}(P,\mathcal{D},\epsilon)$.
\end{proof}















\section{Additional Implementation Details}\label{app:implement}

\subsection{Primitive Implementation}\label{app:implement_primitives}

\stitle{\textsf{CrossProduct}} $\crossproduct_{A_i,A_j\rightarrow A'}(\cdot)$: This primitive replaces the two attributes $A_i$ and $A_j$ by a single attribute $A'$. Given the encrypted input table $\encT$, where all attributes are in one-hot-encoding and encrypted, the attributes of $\encT$ except $A_i$ and $A_j$ remain the same. For every row in $\encT$, we denote the encrypted one-hot-encoding for $A_i$ and $A_j$ by $\tilde{\bf{v}}_1$ and $\tilde{\bf{v}}_2$.  Let $s_1$ and $s_2$ be the domain sizes of $A_i$ and $A_j$ respectively. Then the new one-hot-encoding for $A'$, denoted by $\tilde{\bf{v}}$, has a length of $s=s_1\cdot s_2$. For $l\in \{0,1,\ldots, s-1\}$, we have $$\tilde{\bf{v}}[l] = labMult(\tilde{\bf{v}}_1[l/s_2], \tilde{\bf{v}}_2[l\%s_2]).$$
Only one bit in $\tilde{\bf{v}}$ for $A'$ will be encrypted 1 and the others will be encrypted 0s. When merging more than two attributes, \system can use the $genLabMult()$ described in Section~\ref{genlab} to speed up computation.


%Let $D_1$ and $D_2$  be the encrypted one-hot-coding corresponding to two  values $v_1$ and $v_2$ (integral representation) for attributes $A_1$ and $A_2$ respectively. The corresponding encrypted one-hot-encoding for the two-dimensional attribute $A_1\times A_2$ is given by  \begin{gather} D_{1\times 2}[(i-1)\cdot s_{A_2}+j] = labMult(D_1[i], D_2[j])\\ i \in [s_{A_1}], j \in [s_{A_2}]\end{gather} For this particular case, only $D_{1 \times 2}[(v_1-1)\cdot s_{A_2}+v_2]=Enc(1)$ while all other indices will equate to $Enc(0)$. Note that when computing the one-hot-encoding for a t-dimensional attribute $t > 2$,  for the actual implementation, instead of calling $t$ iterative instances of \textsf{CrossProduct}() we use the $genLabMult()$ operator of labeled homomorphic encryption to speed up the computation.

\stitle{\textsf{Project}} $\project_{\bar{A}}(\cdot)$: The implementation of this primitive simply drops off all but the attributes in $\bar{A}$ from the input table $\encT$ and returns the truncated table $\encT'$.

\stitle{\textsf{Filter}} $\filter_{\phi}(\cdot)$: The predicate $\phi$ in this primitive is a conjunction of range conditions over $\bar{A}$, defined as: for a row $r$ in input table $\encT$,
$\phi(r) = \bigwedge_{A_j\in \bar{A}} ~~(r.{A_j} \in V_{A_j}),$ where $r.A_j$ is the value of attribute $A_j$ in row $r$ and $V_{A_j} \subseteq \{0,1,\ldots,s_{A_j}\}$ (the indices for attribute values of $A_{j}$ with domain size $s_{A_j}$).

First, we will show how to evaluate whether a row $r$ satisfies $r.{A_j} \in V_{A_j}$. Let $\tilde{\bf{v}}_j$ be the encrypted one-hot-encoding of $A_j$, then the indicator function can be computed as
$$I_{r.{A_j}\in V_{A_j}}=\bigoplus_{l\in V_{A_j}}\tilde{\bf{v}}_j[l].$$
If the attribute of $A_j$ in $r$ has a value in $V_{A_j}$, then $I_{r.{A_j}\in V_{A_j}}$ equals $1$; otherwise, $0$.

Next, we can multiply all the indicators using $genLabMult()$ (Section~\ref{genlab}) to check whether all attributes in $A_j\in \bar{A}$ of $r$ satisfy the conditions in $\phi$. Let $\bar{A} = \{A_1,\ldots,A_m\}$, then $$\phi(r) = genLabMult(I_{A_1\in V_{A_1}},\ldots, I_{A_m\in V_{A_m}}).$$

Last, we update the bit of $r$ in $\encB$, i.e., $\encB'[i] = labMult(\encB[i], \phi(r))$, given $r$ is the $i$th row in the input table. This step zeros out some additional records which were found to be extraneous by some preceding filter conditions.

Note that when the \textsf{Filter} transformation is applied for the very first time in a \system program and the input predicate is conditioned on a single attribute $A \in V_A$, we can directly compute the new bit vector using $I_{r.A\in V_{A}}$, i.e., for the $i$th record $r$ in input table $\encT$, we have $\encB'[i] =\bigoplus_{l\in V_A} \tilde{\bf{v}}_j[l]$.  This avoids the unnecessary multiplication $labMult(\encB[i],\phi(r))$.


%As discussed in the preceding section, the predicate $\phi$ is expressed in a special form of conjunctions of range conditions given by eq \ref{phi}. Now for a range condition $A \in \{v_1,...v_t\}$, assuming $\mathbf{\tilde{R}_A}[i]$ is the corresponding one-hot-coding for the $i^{th}$ record's value for attribute $A$,  consider the following \begin{gather}\mathbf{c}_A^i=\bigoplus_{j=1}^{t}\tilde{\mathbf{R}}_{A}[i][v_1]\end{gather} where $\tilde{\mathbf{R}}_{A}[i][v]$ is the $v^{th}$ index of corresponding one-hot-coding. Clearly if the $i^{th}$ record satisfies the condition $A \in \{v_1,...v_t\}$, then exactly one of the values in $\{\tilde{\mathbf{R}}_{A}[i][v_j]\}, j \in \{1,...,t\}$ will be a cipher for $1$. Thus $c_A^i=1$ if record $i$ satisfies the range condition and 0 otherwise. If the condition is instead an equality predicate $A==v$ then $\mathbf{c}_A^i=\tilde{\mathbf{R}}_{A}[i][v]$. Now considering $\phi$ is given by eq \ref{phi}, let us define\begin{gather}\mathbf{c}^i=genLabMult(\mathbf{c}^i_{A_1},...,\mathbf{c}^i_{A_r})\\A^*=\bigcup_{j=1}^rA_j\end{gather} It is easy to see that $c^i$=1 iff record $i$ satisfies $\phi$. Let $\mathbf{B}'$ be the indicator vector before the execution of the current instance of the \textsf{Filter} transformation. The final step is to multiply the $\mathbf{c}^i$s with the corresponding indicator bits and obtain the updated indicator vector $\mathbf{B}$ as follows \begin{gather}\mathbf{B}[i]=labMult(\mathbf{c}^i,\mathbf{B}'[i])\end{gather}
%The above step zeros out some additional records which were found to be extraneous by some preceding filter conditions. Clearly $\textbf{B}$ is the output of the \textsf{Filter} transformation.


%Avoid Indicator Vector Multiplication

%When the \textsf{Filter} transformation is applied for the very first time in a Crypt$\epsilon$ program and the input predicate is conditioned on a single attribute $A \in \{v_1,...,v_k\}$, then we can do the following optimization. Consider \begin{gather}\mathbf{b}[i]=\bigoplus_{j=1}^k \mathbf{\tilde{R}}_A[i][v_j], i \in [m]\end{gather} where $\mathbf{\tilde{R}}_A[i]$ is the one-hot-coding for attribute $A$ for the $i^{th}$ record. Since this is the first instance of the \textsf{Filter} primitive, the current indicator vector $\mathbf{B}$  will be all 1-vector. Thus $\mathbf{b}$ is itself the updated indicator vector  and we can avoid the unnecessary multiplication $labMult(\mathbf{b[i]},\mathbf{B}[i])$.

\stitle{\textsf{Count}} $\countagg(\cdot)$: To evaluate this primitive on its input table $\encT$, \system simply  adds up the bits in the corresponding $\encB$, i.e., $\bigoplus_{i}^m \encB[i]$.

%The \textsf{Count} primitive takes the associated bit vector $\mathbf{B}$ of its input table $\encT$  and simply adds up its entries to return  \begin{gather}\mathbf{c}=\bigoplus_{i=1}^m\mathbf{B}[i]\end{gather}%\item GroupBy*($\mathbf{V},sk$)- This primitive is an extension of the previous GroupBy transformation.

\stitle{\textsf{GroupBy*}} $\groupbystar_{A}(\cdot)$: The implementation steps for \textsf{Project}, \textsf{Filter} and \textsf{Count} are reused here. First, \system projects the input table $\encT$ on attribute $A$, i.e. $\encT_1 = \project_A(\encT)$. Then, \system loops each possible value of $A$. For each value $v$, \system initializes a temporary $\encB_v=\encB$ and filters $\encT'$ on $A=v$ to get an updated $\encB'_v$. Last, \system counts the number of 1s in $\encB'_v$ and release the counts.

\eat{
\begin{enumerate}[label=\alph*)] \item $\mathbf{\tilde{T}}_1$=\textsf{Project}($\mathbf{\tilde{T}}$, $A$) \item $\mathbf{B}$ =  current indicator bit vector \item  for $i = 1:s_A $ \\Intialize bit vector to $\mathbf{B}$  \\$\phi_i= (A==v_{i,A}) $ \\$\hat{\mathbf{T_2}}$ = \textsf{Filter}($\mathbf{\tilde{T}}_1, \phi_i$)\\ $\mathbf{C}[i]$ = \textsf{Count}($\hat{\mathbf{T_2}}$) \\ end for \item Output $\mathbf{C}$ 
\end{enumerate}
}

\stitle{\textsf{GroupByCount}} $\groupbystar_A(\cdot)$: The implementation detail of this primitive is given in the main text (\xh{Section}).

\stitle{\textsf{CountDistinct}}  $\countdistinct(\cdot)$: The implementation of this primitive involves both \AS and \CPS. Given the input encrypted vector of counts $\encV$ of length $s$, the AS first masks $\encV$ to form a new encrypted vector ${\bf \mathcal{V}}$ with a vector of random numbers $M$, i.e., for $i\in \{0,1,\ldots, s-1\}$,
${\bf \mathcal{V}}[i] = {\encV}[i] \oplus labEnc_{pk}(M[i]).$
This masked encrypted vector is then sent to \CPS and decrypted by \CPS to a plaintext vector $\mathcal{V}$ using the secret key.

Next, \CPS generates a garbled circuit which takes (i) the mask $M$ from the \AS, and (ii) the plaintext  masked vector $\mathcal{V}$ and a random number $r$  from the \CPS as the input. This circuit first removes the mask $M$ from $\mathcal{V}$ to get $V$ and then counts the number of non-zero entries in $V$, denoted by $c$. A masked count $c'=c+r$ is outputted by this circuit. \CPS send both the circuit and the encrypted random number $labEnc_{pk}(r)$  to \AS.

Last, the \AS evaluates this circuit to the masked count $c'$ and obtains the final output to this primitive: ${\bf c} = labEnc_{pk}(c') - labEnc_{pk}(r)$.

\eat{
   ($\mathbf{V},\epsilon$) - The \textsf{CountDistinct} primitive is implemented as follows \begin{enumerate}[label=\alph*)]\item Firstly the \textsf{AS} creates a mask vector drawn uniformly at random from $[m]^{s_A}$, i.e.,  \begin{gather*} M[i] \in_R [m], i \in [|V|]\end{gather*} \item \textsf{AS} masks the encrypted true count vector $\mathbf{V}$  as follows \begin{gather*}\boldsymbol{\mathcal{V}}[i]= \mathbf{V}[i] \oplus labEnc_{pk}(M[i])\end{gather*} and sends it to the \textsf{CSP} \item \textsf{CSP} decrypts the masked encrypted vector as \begin{gather*}\mathcal{V}[i]=labDec_{sk}(\mathbf{V}[i]), i \in [|V|]\end{gather*} \item Next the \textsf{CSP} generates the following garbled circuit that\begin{enumerate}[label=\roman*)]  \item takes the mask $M$ as an input from the \textsf{AS} \item takes a random number $r$  as an input from the \textsf{CSP}\item takes the decrypted masked vector $\mathcal{V}$ as an input from the \textsf{CSP} \item removes the mask $M$ from $\mathcal{V}$ as \begin{gather*}V[i]=\mathcal{V}[i]-M[i], i \in [|V|]\end{gather*}\item  counts the number of non-zero entries of $V$ as C \item adds the laplace noises \begin{gather*}\mathcal{C}=C+r\end{gather*} and returns $\mathcal{C}$ \end{enumerate} \item The \textsf{AS} evaluates the above circuit and gets output $\mathcal{C}$ \item The \textsf{AS} gets $labEnc_{pk}(r)$ from the \textsf{CSP} and generates $labEnc_{pk}(\mathcal{C})$ to compute\begin{gather*}\mathbf{C}=labEnc_{pk}(\mathcal{C})-labEnc_{pk}(r)\end{gather*} \end{enumerate}
}


\stitle{\textsf{Laplace}} $\lap_{\epsilon,\Delta}(\mathbf{V})$: Given an encrypted vector counts $\encV$ of size $s$, both \AS and \CPS have to add Laplace noise in this primitive. Hence, this implementation involves two steps.

First, the \AS adds encrypted Laplace noise vector to $\encV$, i.e., for $i\in \{0,1,\ldots,s\}$, $\hat{\encV}[i]  = \encV[i] \oplus labEnc_{pk}(\eta_i),$ where $\eta_i\sim Lap(\Delta/\epsilon)$. This encrypted noisy vector $\hat{\encV}$ is then sent to the \CPS.

Next, the \CPS decrypts $\hat{\encV}$ using the secret key and add another Laplace noise vector, i.e., for  $i\in \{0,1,\ldots,s\}$, $\hat{V}[i] = labDec_{sk}(\hat{\encV}[i]) +\eta'_i$, where $\eta'~\sim Lap(\Delta/\epsilon)$. This plaintext noisy vector is returned as the final output of this primitive.



\eat{
($\mathbf{V},\epsilon$)- Recall that both \textsf{AS} and \textsf{CSP} have to add Laplace noise to the output in Crypt$\epsilon$. Hence the \textsf{Laplace} primitive has two components. The first component is executed by the \textsf{AS} wherein,
\begin{enumerate} \item \textsf{AS} generates a noisy vector $\eta$ such that $\eta \in [Lap(\frac{1}{\epsilon})]^{|V|}$ \item encrypts $\eta$ and adds it to the input vector as \begin{gather*}\boldsymbol{\eta}=labEnc_{pk}(\eta)\\\mathbf{\hat{V}}[i]=\mathbf{V}[i]\oplus \boldsymbol{\eta}[i], i \in [|V|]\end{gather*} \end{enumerate} This encrypted noisy vector $\mathbf{\hat{V}}$ is the input for the second phase of the \textsf{Laplace} primitive which is executed by the \textsf{CSP} as follows \begin{enumerate}\item Decrypts $\mathbf{\hat{V}}$ \begin{gather*}\hat{V}=labDec_{sk}(\mathbf{\hat{V}})\end{gather*}  \item Generates a noisy vector $\eta'$ such that $\eta' \in [Lap(\frac{1}{\epsilon})]^{|\hat{V}|}$ \item Finally adds the noise $\eta'$ to $\hat{V}$ \begin{gather*}\hat{\mathcal{V}}[i]=\hat{V}[i]+\eta'[i], i \in [|\hat{V}|]\end{gather*} \item Returns $\hat{\mathcal{V}}$ to \textsf{AS} \end{enumerate} 
% Note that in the Crypt$\epsilon$ implementation we need to add two instances of the Laplace noise as opposed to a single instance in the standard central differential privacy setting. After the addition of the first instance of the laplace noise, $\eta$ (by the AS),  the encrypted answer is sent to the CSP. becuse of CSP has only a differential private view Hence the addition of the second instance of the laplace noise can be looked upon as a post-processing step  However and differential privacy is immune to post processing 
}


\stitle{\textsf{NoisyMax}} $\noisymax_{\epsilon,\Delta}^k(\cdot)$: The input to this primitive is an encrypted vector of counts $\encV$ of size $s$. Similar to \textsf{Laplace} primitive, both \AS and \CPS are involved. First,  the \AS adds to $\encV$
an encrypted Laplace noise vector and a mask $M$, i.e., for $i\in \{0,1,\ldots,s\}$,
$\hat{\encV}[i]  = \encV[i] \oplus labEnc_{pk}(\eta_i) \oplus M[i],$
where $\eta_i\sim Lap(\Delta/\epsilon)$. This encrypted noisy, masked vector $\hat{\encV}$ is then sent to the \CPS.

Next, the \CPS decrypts $\hat{\encV}$ using the secret key, i.e., for  $i\in \{0,1,\ldots,s\}$, $\hat{V}[i] = labDec_{sk}(\hat{\encV}[i])$. The \CPS generates a garbled circuit which takes  (i) the noisy, masked vector $\hat{V}$ from the \CPS, and (ii) the mask $M$ from the \AS as the input. This circuit will remove the mask from $\hat{V}$ to get the noisy counts $\hat{V}'$ and find the indices of the top-$k$ values in $\hat{V}'$.

Finally, the \AS evaluates the circuit above and returns the indices as the output of this primitive.


\eat{
($\mathbf{V},\epsilon,k$)- The input to the NoisyMax primitive is an encrypted vector $\mathbf{V}$ where each entry $V[i]$ is a count. The primitive is implemented via the following steps.  \begin{enumerate}
\item First the \textsf{AS} adds noise to the input encrypted vector as follows \begin{gather*} \eta \in [Lap(\frac{1}{\epsilon})]^{|V|}\\\boldsymbol{\eta}=labEnc_{pk}(\eta)\\\mathbf{\hat{{V}}}[i]=\mathbf{V}[i]+ \boldsymbol{\eta}[i], i \in [|V|] \end{gather*} \item Next the \textsf{AS} creates a mask vector $M$ drawn uniformly at random from $[m]^{s_A}$, i.e.,  \begin{gather*} M[i] \in_R [m], i \in [|V|]\end{gather*} \item \textsf{AS} masks the encrypted noisy vector $\mathbf{\hat{V}}$  as follows \begin{gather*}\boldsymbol{\mathcal{V}}[i]= \mathbf{\hat{V}}[i] \oplus labEnc_{pk}(M[i]), i \in [|V|]\end{gather*} and sends it to the \textsf{CSP} \item \textsf{CSP} decrypts the masked encrypted noisy vector as \begin{gather*}\mathcal{V}[i]=labDec_{sk}(\mathbf{\hat{V}}[i]), i \in [|V|]\end{gather*} \item Next, the following garbled circuit is evaluated which
    \begin{enumerate}[label=\roman*]\item takes noisy masked  vector $\mathcal{V}$ as an input from the \textsf{CSP} \item takes mask $M$ as the input from \textsf{AS}  \item removes the mask from  $\mathcal{V}$  as \begin{gather*} \hat{V}[i]=\mathcal{V}[i]-M[i], i \in [|V|]\end{gather*}  \item computes the top $k$ element over  $\hat{V}$ and returns $arg_{\textit{top k}}\max{\hat{V}[i])}$
    \end{enumerate}
    \end{enumerate}
}


\subsection{DP Index Optimization Implementation}\label{index-imp}
 The DP index optimization can be implemented via a garbled circuit that \begin{enumerate}\item takes the entire database $\boldsymbol{\mathcal{\tilde{D}}}$ as an input and the attribute $A$ as an input from the \textsf{AS}.
\item takes the secret key $sk$ as an input from  the \textsf{CSP} \item Decrypts $\boldsymbol{\mathcal{\tilde{D}}}$ \item Sorts the decrypted database on $A$, i.e., the first $ct_{A,v_1}$ rows are the ones with value $v_1$ for attribute $A$, the next $ct_{A,v_2}$ are  the records with value $v_2$ for attribute $A$ and so on. \item  re-encrypts the sorted database \item Divide the domain of $A$ into $k$ bins such that each bin contains $s_A/k$ consecutive domain values. \item Construct a $k$ lengthed vector $\hat{V}$ such that $\hat{V}[i]=\sum_jct_{A,j}+\eta_i, i \in [k], j \in [\frac{s_A}{k}(i-1)+1,\frac{s_A}{k}i]$ where $\eta_i$ is a random laplace noise drawn from the distribution $Lap(\frac{k}{\epsilon})$ \item Return $\hat{V}$ and sorted $\boldsymbol{\mathcal{\tilde{D}}}_{sort}$\end{enumerate}
\begin{figure}\includegraphics[height=4cm,width=8cm]{kk.png} \caption{ $genLabMult()$ - Batching of multiplicands for \textsf{labHE}} \label{genlab-fig}\end{figure}

\section{Related Works}
\subsection{Differential Privacy}
Introduced by Dwork et al. in \cite{Dork}, differential privacy has enjoyed immense attention from both academia and industry in the last decade. In this section, we will discuss some of the most recent directions in differential privacy in both the \textsf{CDP} and \textsf{LDP} models. An interesting line of work in the \textsf{CDP} model has been towards proposing "derived" mechanisms \cite{MVG} or  "revised algorithms" \cite{Blocki} whose privacy guarantee can be deduced from basic mechanisms (like exponential mechanism, Laplace mechanism etc) via the composition theorems and
the post-processing immunity property \cite{Dork}. Such mechanisms have the advantage of better utility as their design leverages on  specific properties of the query and the data.   One such technique is based on data partition and aggregation  \cite{AHP,DAWA,hist1,hist2,hist3,hist4,hist6,hist7,hist8} and is helpful in answering histogram queries. Another technique involves 
non-uniform data weighting where each data sample is weighed based on their query contribution. Research in this line of work include \cite{u1,u2,MWEM}. Yet another popular method is to utilize past/auxiliary information
to improve the utility of the query answers. Examples are \cite{A1,A2,A3,A4,A5,A6,A7,A8}
%Another interesting line of work has been towards developing programming frameworks to enable non-experts to write easy differentially private programs. This line of work was started by the PINQ platform \cite{PINQ} and there has been a series of follow up work  \cite{FWPINQ,p2, airavat}. The most recent one is the Ektelo \cite{ektelo} framework where all existing algorithms for answering linear counting queries can be expressed as a composition of its operators. 
\par
The local differential privacy model was first introduced by Kasiviswanathan
et al. in\cite{Kasivi}. Possibly the most simple \textsf{LDP} technique is the randomized response \cite{RR} protocol which was proposed by Warner in 1960s.  In the recent times, constructing a frequency oracle to estimate the frequencies of any value in the domain, is perhaps the most fundamental \textsf{LDP} problem. The state-of-the-art locally differentially private histogram estimator solutions are \cite{LDP1, LDP2, Rappor1}.  However, when the domain size of the input values is extremely large, it might be computationally infeasible to construct the histogram over the entire domain. To tackle this, specialized algorithms to compute the most frequently occurring values, also known as the heavy hitters, have been proposed \cite{HH,Rappor2,HH2}. Another practical setting can be when the user's data is a set of items and the aggregator is interested in  the $k$ most frequent item sets. This problem is addressed in \cite{15,itemset}. In \cite{Cormode, CALM} the authors propose efficient constructions of marginal tables in the local differential privacy setting. Due to their attractive trust model, \textsf{LDP} has also enjoyed significant industrial adoption.  Google has integrated RAPPOR \cite{Rappor1, Rappor2} with Chrome. It is primarily tasked with collecting user statistics like default browser homepage, default search engine et al in order to monitor malicious hijacking of user settings. Apple \cite{Apple} has also deployed differential privacy to collect of data like most frequent emojis, help with auto-completion of spellings etc.  Samsung \cite{Samsung} proposed a similar system 
which enables the collection of both categorical 
(like screen resolution) as well as numerical data (like
time of usage, battery volume), although it is not clear
whether they went ahead with the actual deployment. \par
Recently it has been showed that augmenting the local differential privacy setting by an additional layer of anonymity can improve the privacy
guarantees. The first work to study this was PROCHLO \cite{Prochlo} implementation by Google. In \cite{Prochlo} the authors propose a  Encode, Shuffle, Analyze (ESA) architecture
 which relies on an explicit intermediate shuffler that processes the randomized LDP reports
from users to ensure their anonymity. PROCHLO necessitates  this intermediary to be trusted, this is implemented via trusted hardware enclaves (Intel's SGX). However, as showcased by recent attacks \cite{Foreshadow}, it is notoriously difficult to design a  truly secure hardware in practice. Motivated by PROCHLO, the authors in \cite{amplification}, present a tight upper-bound on the worst-case privacy loss. Formally, they show that  any permutation invariant
algorithm satisfying $\epsilon$-local differential privacy will satisfy $O(\epsilon\sqrt{\frac{\log(\frac{1}{\delta})}{n}},\delta)$ -central differential
privacy. Cheu et al in \cite{mixnets} demonstrate privacy amplification by the same factor for 1-bit randomized response by using a mixnet architecture to provide the anonymity. Another important result from this work is that, they prove that the power of the mixnet model lies strictly between those of the central and local
models.
\par A parallel line of work involves efficient use of cryptographic primitives for differentially private
functionalities. In \cite{kamara}  Agarwal et al. propose an algorithm for computing histogram over encrypted data. Rastogi et al. \cite{Rastogi} and
Shi et al. \cite{Shi} proposed algorithms that allow an untrusted aggregator to periodically
estimate the sum of $n$ users' values in a  privacy preserving fashion. However both the schemes are irresilient to user failures. Chan et al. tackle this in \cite{Shi2} by constructing binary interval trees over the users.
\subsection{Two-Server Model}
The two-server model is a popular choice for privacy preserving machine learning techniques. In \cite{Boneh1}, \cite{LReg}, \cite{Ver} and \cite{Ridge2}, the authors propose privacy preserving ridge regression systems with the help of a cryptographic service provider. While \cite{Ridge2} uses a hybrid multi-party computation scheme with a secure inner product technique, \cite{Boneh1} proposes a hybrid approach combining homomorphic encryptions and Yao's garbled circuits. Gascon et al. \cite{Ver} extended the results in \cite{Boneh1} to include vertically partitioned data and \cite{LReg} solves the problem using just linear homomorphic encryption.  Zhang et al in \cite{secureML} also propose secure machine learning protocols using a privacy-preserving stochastic gradient descent method. Their main contribution includes developing efficient algorithms for secure arithmetic
operations on shared decimal numbers and proposing alternatives to non-linear
functions such as sigmoid and softmax tailored for MPC computations. 
In \cite{Boneh2} and \cite{Matrix2} the authors solve the problem of privacy-preserving matrix factorization. In both the papers, use a hybrid approach combining homomorphic encryptions and Yao's garbled circuits for their solutions. 

\subsection{Homomorphic Encryption}
With improvements made in implementation efficiency and new constructions developed in the recent past, there has been a surge in practicable privacy preserving solutions employing homomorphic encryptions. A lot of the aforementioned two-server models employ homomorphic encryption \cite{Boneh1,Boneh2,LReg,Matrix2}.  
In \cite{CryptoDL,CryptoNet,NN} the authors enable neural networks to be applied to homomorphic-ally encrypted data.
Linear homomorphic encryption is used in \cite{Irene2} to enable privacy-preserving machine learning for ensemble methods while %\cite{FHEReg} 
uses  fully-homomorphic encryption
to approximate the coefficients of a logistic-regression model.
\cite{grid} uses somewhat-
homomorphic encryption scheme to compute the forecast
prediction of consumer usage for smart grids. 
%Privacy preserving multi-party machine learning with homomorphic encryption


\section{Discussion}
\subsection{Joint Laplace Noise Generation}\label{jointLap}
Recall that in \system,  both the servers, \textsf{AS} and \textsf{CSP} have to add two separate instances of Laplace noise to true answer. Thus the expected error incurred in \system is quantitatively twice that of the traditional Laplace mechanism in traditional \textsf{CDP} model. However, there is an alternative way of jointly computing a single instance of the Laplace noise via a secure multi party computation protocol~\cite{Djoin}. For this, the \CPS generates a garbled circuit that takes one $l$-bit random string from \CPS and \AS each as an input, denoted by $S_1$ and $S_2$ respectively. This circuit performs$S=S_1 xor S_2$  and uses it to generate an instance of random noise, $\eta$ drawn from the distribution $Lap(\frac{1}{\epsilon})$ following the fundamental law of transformation of probabilities.
Last, the circuit encrypts $\eta$ and returns $\boldsymbol{\eta}=labEnc_{pk}(\eta)$.
Hence, this approach adds just one instance of the Laplace noise, resulting the same accuracy guarantee as the \textsf{CDP} model. However owing to the garbled circuit, this implementation is computationally heavier. Therefore, we choose the two phase noise addition implementation for \system in this paper.

\eat{\begin{enumerate}\item takes a $l$-bit random string, $S_1$ as an input from the \textsf{CSP}
    \item takes another $l$-bit random string $S_2$ as an input from the \textsf{AS} \item performs $S=S_1 xor S_2$  and uses it to generate an instance of random noise, $\eta$ drawn from the distribution $Lap(\frac{1}{\epsilon})$ following the fundamental law of transformation of probabilities \item encrypts $\eta$ and returns $\boldsymbol{\eta}=labEnc_{pk}(\eta)$\end{enumerate}}


\subsection{Separation from LDP model}\label{app:sepldp}
As discussed in the introduction, the \textsf{LDP} model is less powerful in query answering than the \textsf{CDP} model~\cite{Kasivi,mixnets}. However, by virtue of secure computation, we can potentially implement all the functionalities of the \textsf{CDP} model in our two-server model. However, by virtue of secure computation, we can potentially implement all the functionalities of the \textsf{CDP} model in our two-server model. Functional efficiency might be a point of contention in certain cases, but nothing in the architecture of  \system can restrict its algorithmic expressibility. Recall that the power of the \textsf{LDP} model is equivalent to that of the statistical query model \cite{SQ1} from learning theory and there exists an exponential separation between the accuracy and sample complexity of local and central algorithms~\cite{Kasivi}.

We will showcase three different queries which can be computed efficiently in \system but infeasible in the standard \textsf{LDP} model.

\stitle{DNF Queries.}
The class of DNF queries fall outside the scope of statistical query models \cite{DNF}. Hence it is infeasible to answer counting queries based on a predicate with a disjunction in the \textsf{LDP} model. However, we can answer them in \system as follows.
Consider a DNF query 
$$\phi = (A_{11}\land ...\land A_{1k}) \vee ... \vee (A_{t1}\land ... A_{tl}), t, k,l \in \mathcal{Z}_{\geq 0}.$$

Let $Attribute(\phi)$ denote the set of all attributes in $\mathcal{A}$ that appear in the boolean condition $\phi$. For example, if $$\phi = \big((\mathcal{A}_1==v_1) \land \mathcal{A}_2==v_2) \vee \mathcal{A}_3==v_3 \big),$$ then we have $Attribute(\phi)=\{\mathcal{A}_1, \mathcal{A}_2,\mathcal{A}_3\}$.

Firstly, the \AS computes the attribute set $\bar{A}=Attribute(\phi)$ and projects $\encD$ on $\bar{A}$. Then, the \AS constructs the encrypted one-hot-coding of over all attributes in $\bar{A}$ using the cross product transformation. We denote the new attribute $A'$ and the new table $\encD'$. Now, the \AS simply filters this new table $\encD'$ with predicate $\phi'$  such that $\phi'$ is the equivalent of $\phi$ when expressed in terms of the new  attribute $A'$. Finally, \AS  performs the \textsf{Count} transformation and the \textsf{Laplace} transformation to obtain the final result.

\eat{
\begin{enumerate}\item Firstly, the \textsf{AS} computes the attribute set $A^*=Attribute(\phi)$.
\item Next the \textsf{AS} performs a \textsf{Project} transformation on inputs attribute set $A^*$ and the entire encrypted database $\boldsymbol{\tilde{\mathcal{D}}}$. 
\item Let $A^*= \{A^*_1,A^*_2,\ldots,A^*_t\}, t \leq k$. The \textsf{AS} constructs the encrypted one-hot-coding over the entire $t$-dimension $\lq$attribute' $\mathcal{A}^*=\times_{i=1}^t A^*_i$ by $(t-1)$ iterative application of the cross product transformation. 
\item Note that the result of the preceding step is a $m\times 1$ table where the $i^{th} , i \in [m]$ record corresponds to the encrypted one-hot-coding over the entire $t$-dimension domain space of $\mathcal{A}^*$ of data owner $\textsf{DO}_i$. Now the \textsf{AS} simply applies the \textsf{Filter} transformation on this table with predicate $\phi'$  such that $\phi'$ is the equivalent of $\phi$ when expressed in terms of the new  attribute $\mathcal{A}^*$.
\item This is followed by performing the \textsf{Count} transformation and the \textsf{Laplace} transformation to obtain the final result. 
\end{enumerate}
}

\stitle{Variable Selection Problem.} The variable selection problem is an optimization problem described as follows. Given a set of counting queries, the problem finds the query with nearly the largest value, i.e., computes an approximate argmax. Cheu et al.~\cite{mixnets} prove that the sample complexity of this problem in the "one-message" mixnet model (i.e., each user send only a single message into the shuffle) is exponentially larger than that of the \textsf{CDP} model. The variable-selection problem is actually equivalent to the exponential mechanism~\cite{Dork} in the \textsf{CDP} model. Moreover, the exponential mechanism is simply a variant of the "Report Noisy-Max" algorithm with a different noise distribution \cite{Nm}.  Thus essentially, the \textsf{NoisyMax} primitive in \system is capable of solving the variable-selection problem accurately.

\stitle{Count Distinct Values.} Consider the problem of computing the number of distinct values out of a set of $m$ user data where the domain of the values is $S$ and $m<<|S|$. In the \textsf{LDP} model, for small sizes of $S$, one can construct a frequency oracle and compute the number of values with non-zero count with some careful thresholding~\cite{LDP1}. However, if the size of $S$ is huge then it becomes computationally infeasible to deploy the aforementioned mechanism. For example, if the values correspond to different URLs, since the total domain size is $2^{64}$, computation limitations make this problem infeasible to be solved in the \textsf{LDP} setting. There are mechanisms that use techniques like dividing the data into groups and then performing specialised processing etc to find top k heavy hitters in this setting \cite{HH,HH2} but still these work only for low values of $k<60$.  Although for our discussion in the paper we have considered the one-hot-coding as our preferred data encoding scheme, \system architecturally can support any arbitrary encoding scheme.  For instance, for URLs the data owners can instead use the domain name based encoding (i.e., subdomain.secondleveldomain.topleveldomain) for encrypting their data. Following this, an appropriate garbled circuit to count the number of distinct values from this encrypted dataset (which can be defined as a new Crypt$\epsilon$ primitive) can answer the above query in the \system setting.

\begin{comment}\subsection{Answering queries with disjunctions in predicate} Now let us consider a DNF query predicate $\phi=\phi_1 \vee \phi_2$ where $\phi_1=(A_1==v_1 \wedge \ldots \wedge A_n==v_n)$ and $\phi_2=(A'_1==v'_1 \wedge \ldots \wedge A'_n==v'_n)$ are two conjunctive clauses. For a given record assume, \begin{gather*}\mathbf{d}=\mathbf{c_1}\oplus \mathbf{c_2}-labMult(\mathbf{c_1,c_2}) \\
\mathbf{c_1}=genLabMult(\mathbf{\tilde{R}}_{A1}[v_1], \ldots ,\mathbf{\tilde{R}}_{An}[v_n] ) \\ \mathbf{c_1}=genLabMult(\mathbf{\tilde{R}}_{A1}[v_1], \ldots ,\mathbf{\tilde{R}}_{An}[v_n] )\end{gather*} Note that $d=1$  only iff  the record satisfies $\phi$. Thus for a two clause  DNF predicate as above, the optimized Filter transformation takes as input $x \times y$ encrypted table $\tilde{\mathbf{T}}$ with attribute set $\bigcup_{i=1}^n Attribute(\phi_i)$ and outputs a $x \times 1$ encrypted table $\mathbf{\tilde{T}}'$ such that \begin{gather} \mathbf{\tilde{T}}'[i]= \mathbf{c_1}\oplus \mathbf{c_2}-labMult(\mathbf{c_1,c_2}) \\
\mathbf{c_1}=genLabMult(\mathbf{\tilde{R}}_{A1}[v_1], \ldots ,\mathbf{\tilde{R}}_{An}[v_n] ) \\ \mathbf{c_1}=genLabMult(\mathbf{\tilde{R}}_{A1}[v_1], \ldots ,\mathbf{\tilde{R}}_{An}[v_n] )\end{gather} For $t>2$ clauses in a DNF, apply the Filter transformation pairwise for $\lceil \log t \rceil$ iterations. 
\end{comment}

An important point to be noted here is that the power of the shuffler or mixnet model  (which is obtained by augmenting \textsf{LDP} with anonymization via shuffling)~\cite{Prochlo, mixnets,amplification},  lies strictly between that of traditional \textsf{LDP} and \textsf{CDP}. Thus the two-server model of Crypt$\epsilon$ differs from this line of work in three major ways. First, \system results in no reduction in expressibility as compared to that of the \textsf{CDP} model. Second, the mixnet/shuffler model results in an approximate DP guarantee $(\epsilon\sqrt{\frac{\log\frac{1}{\delta}}{n}},\delta)$ which incurs an expected error of $O(\epsilon\sqrt{\log\frac{1}{\delta}})$.  In practice, $\delta$ has to be smaller than $\frac{1}{n}$ in order to get  meaningful privacy. In contrast, \system achieves the same order of accuracy guarantees as that of the \textsf{CDP} model. Finally, the trust assumptions of the shuffler/mixnet model differ from that of \system. Google's implementation relies on a trusted intermediary shuffler which they implement via trusted hardware enclaves. However truly secure hardware enclaves are notoriously difficult to achieve in practice \cite{Foreshadow}. The mixnet model on the other hand requires a  mix network or mixnet which is a protocol involving several computers that inputs a sequence of encrypted messages, and outputs a uniformly random permutation of those messages' plaintexts.  Their trust assumption is that at least one of the servers needs to behave honestly. For \system, both the servers are completely untrusted under the constraint that they are non-colluding and follow the protocols semi-honestly.

%\section{Additional Experimental Results}
\begin{figure}
     \includegraphics[width=0.5\linewidth]{scale2.pdf}
        \caption{Scalability of \system Programs Cntd.}\label{scale}
    \end{figure}
    \begin{figure}
     \includegraphics[width=0.5\linewidth]{2_final.pdf}
        \caption{Accuracy Analysis of \system Program 2 }\label{scale}
    \end{figure}
    \begin{figure}
     \includegraphics[width=0.5\linewidth]{4_final.pdf}
        \caption{Accuracy Analysis of \system Program 4 }\label{scale}
    \end{figure}
    \begin{figure}
     \includegraphics[width=0.5\linewidth]{6_finals.pdf}
        \caption{Accuracy Analysis of \system Program 6 }\label{scale}
    \end{figure}
\eat{\begin{figure*}
    \begin{subfigure}[b]{0.25\linewidth}
        \centering
        %\includegraphics[width=5cm,height=3.1cm]{t1.pdf}
         \includegraphics[width=1\linewidth]{2_final.pdf}
        \caption{ Program 2}
        \label{fig:P2}
    \end{subfigure}
    \begin{subfigure}[b]{0.25\linewidth}
    \centering \includegraphics[width=1\linewidth]{4_final.pdf}
        \caption{Program 4}
        \label{fig:P4}\end{subfigure}
    \begin{subfigure}[b]{0.25\linewidth}
    \centering    \includegraphics[width=1\linewidth]{6_finals.pdf}
        \caption{Program 6}
        \label{fig:P6}\end{subfigure}
   \caption{Accuracy Analysis of Crypt$\epsilon$ Programs Cntd.}
   \label{accuracy}
\end{figure*}

 
\stitle{Metrics:}\\
\textit{Accuracy}}


\end{document}
