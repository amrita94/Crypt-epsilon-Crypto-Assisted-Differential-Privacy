
\section{Background}

\subsection{Differential Privacy}
\begin{definition} An algorithm $\mu$
satisfies $\epsilon$-differential privacy ($\epsilon$-DP), where $\epsilon \geq 0$ is a privacy parameter, iff
 for any two datasets $D$ and $D'$ that differ in a single record, we have
\begin{gather}
\forall t \in Range(\mu), Pr \big[\mu(D) = t\big] \leq e^{\epsilon}Pr\big[\mu(D') = t\big]
\end{gather}
where $Range(\mu)$ denotes the set of all possible outputs
of the algorithm $\mu$.
\end{definition}


\begin{comment}\subsubsection{Local Differential Privacy}
\arc{Not final-placeholder}
In the local setting, there is no trusted third party. An aggregator
wants to gather information from users. Users
are willing to help the aggregator, but do not fully trust
the aggregator for privacy. For the sake of privacy, each
user perturbs her own data before sending it to the aggregator
(via a secure channel). 

Consider a setting where each user has a single value $v$, which can be viewed
as the user’s answer to a given question. The aggregator
aims to find out the frequencies of values among the
population. Such a data collection protocol consists of
the following algorithms:
\begin{enumerate}
\item Encode is executed by each user. The algorithm
takes an input value v and outputs an encoded value
x.
\item  Perturb, which takes an encoded value x and outputs
y. Each user with value v reports y =
Perturb(Encode(v)). For compactness, we use
PE(·) to denote the composition of the encoding
and perturbation algorithms, i.e., PE(·) =
Perturb(Encode(·)). PE(·) should satisfy $\epsilon$-local
differential privacy, as defined below.
\item Aggregate is executed by the aggregator; it takes all
the reported values, and outputs aggregated information.
\end{enumerate}
\begin{definition}
 Local Differential Privacy- An algorithm
$A$ satisfies $\epsilon$-local differential privacy ($\epsilon$-LDP),
where $\epsilon \leq 0$, if and only if for any input $v_1$ and $v_2$, we
have
\begin{gather} \forall y  \in Range(A) : Pr [A(v_1) = y] \leq e^{\epsilon} Pr [A(v_2) = y] \end{gather}
where $Range(A)$ denotes the set of all possible outputs
of the algorithm $A$.
\end{definition}
This notion is related to randomized response [24],
which is a decades-old technique in social science to collect
statistical information about embarrassing or illegal
behavior. To report a single bit by random response, one
reports the true value with probability p and the flip of the
true value with probability 1-p. This satisfies
(
ln p
1-p
)
-
LDP.
Comparing to the setting that requires a trusted data
curator, the local setting offers a stronger level of protection,
because the aggregator sees only perturbed data.
Even if the aggregator is malicious and colludes with all
other participants, one individual’s private data is still
protected according to the guarantee of LDP.
\subsubsection{Computational Differential Privacy}
\begin{definition}
 (IND-CDP privacy) An ensemble $\{f_\kappa\}\kappa  \in N$ of randomized
functions $f_\kappa : D \rightarrow R_\kappa$ provides $(\epsilon,
\kappa)$-ind-cdp if there exists a negligible function $negl(\cdot)$ such that for every nonuniform p.p.t turing machine (“distinguisher”) $A$, every polynomial $p(\cdot)$, every sufficiently large $\kappa \in N$ all
data sets $D,D' \in \mathcal{D}$ of size at most $p(\kappa)$ such that $|D-D'|\leq  1$, and
every advice string $z_{\kappa}$ of size at most $p(\kappa)$, it holds that \begin{gather}
Pr [A_{\kappa}(f_{\kappa}(D)) = 1] \leq e^{\epsilon} \times Pr[A_{\kappa}(f_{\kappa}(D')) = 1]
+ negl(\kappa)\end{gather}
where we write $A_\kappa(x)$ for $A(1^{\kappa}, z_{\kappa}, x)$ and the probability is taken over
the randomness of mechanism $f_\kappa$ and adversary $A_\kappa$.
\end{definition}


\end{comment}
\subsection{Cryptographic Primitives}
\subsubsection{Linearly Homomorphic Encryption (\textsf{LHE})}
 Let $(\mathcal{M}, +)$ be a finite group. A \textsf{LHE} scheme 
for messages in $\mathcal{M}$ is defined by three algorithms:\begin{itemize}
\item Key Generation ($Gen$) -The key generation algorithm $Gen$ takes the security parameter $\kappa$ as input and outputs
a matching pair of secret and public keys, $(s_k, p_k) \leftarrow Gen(\kappa)$.
\item Encryption ($Enc$) - This is a randomized algorithm that encrypts a message $m \in \mathcal{M}$ via the public key $p_k$, to generate ciphertext $c \leftarrow Enc_{pk}(m)$.
\item Decryption ($Dec$) - The decryption algorithm $Dec$ is a deterministic function that uses the secret key $s_k$ to
recover the original plaintext from a ciphertext c.
\end{itemize}
In addition, linearly homomorphic encryption scheme supports the following two operations
\begin{itemize}\item Operator $\oplus$ - Let $c_1 \leftarrow Enc_{pk}(m1), \ldots, c_a \leftarrow Enc_{pk}(m_a), a \in \mathcal{Z}_{>0}$. Then we have  $Pr\big[Dec_{sk}(c_1\oplus c_2 ...\oplus c_a)=    m_1 + \ldots   + m_a\big]=1$  
\item Operator $cMult(a,c)$ - Let $c\leftarrow  Enc_{sk}(m)$. Then  \\ $Dec_{sk}\big(cMult(a,c)\big)=am$ where $cMult(a,c)=c\oplus \ldots \oplus c$ (a times) \end{itemize}
\subsubsection{Labeled Homomorphic Encryption(\textsf{labHE})}
Let $(Gen,Enc,Dec)$ be an \textsf{LHE} scheme with security parameter $\kappa$ and message space $\mathcal{M}$. Assume that a multiplication operation is given in $\mathcal{M}$, i.e., is a finite ring. Let also $\mathcal{F}:\{0,1\}^s \times \mathcal{L}\rightarrow \mathcal{M}$ be a pseudo-random function with speed space $\{0,1\}^s$( s= poly($\kappa $)) and the label space $\mathcal{L}$. Define
\begin{itemize}
\item $labGen(\kappa)$ - On input $\kappa$, it runs $Gen(\kappa)$ and outputs $(sk,pk)$
\item $localGen(pk)$ -  For each user $i$ and with the public key as input, it samples a random seed $\sigma_i \in \{0,1\}^s$ and computes $pk_i = Enc_{pk}(\ddot{\sigma_i})$ where $\ddot{\sigma_i}$ is an  encoding of $\sigma_i$ as an  element of $\mathcal{M}$. It outputs $(\sigma_i,pk_i)$.
\item $labEnc_{pk}(\sigma_i, m , \tau)$: On input a message $m \in \mathcal{M} $ with label $\tau \in \mathcal{L}$  from user $i$, it computes $b=\mathcal{F}(\sigma_i, \tau))$ and outputs the labeled ciphertext $\mathbf{c}=(a,c) \in \mathcal{M} \times \mathcal{C}$ with $ a= m- b$ in $\mathcal{M}$ and $c=Enc_{pk}(b)$.
\item $labDec_{sk}(\mathbf{c})$ - This functions inputs a cipher $\mathbf{c}=(a,c) \in \mathcal{M} \times \mathcal{C}$ encrypted under the labHE scheme and decrypts it as $m=a-Dec_{sk}(c)$.
\end{itemize}
In addition to the aforementioned operations supported by a linear homomorphic encryption scheme, labeled homomorphic encryption supports multiplication of two labeled ciphers as follows.
\begin{itemize}\item $labMult(\mathbf{c}_1,\mathbf{c}_2)$ -
On input two labeled ciphers $\mathbf{c}_1=(a_1,c_1)$ and $\mathbf{c}_2=(a_2,c_2)$, it computes a "multiplication" ciphertext $\mathbf{d}=labMult(\mathbf{c_1,c_2})=Enc_{pk}(a_1,a_2)\odot cMult(c_1,a_2) \odot cMult(c_2,a_1)$. Observe that $Dec_{sk}(\mathbf{d})=m_1\cdot m_2 -b_1 \cdot b_2$.
\item $labMultDec_{sk}(c_1,c_2,\mathbf{d})$ - On input two labels $c_1,c_2$ of two labHE ciphers $\mathbf{c_1},\mathbf{c_2}$ and the output $\mathbf{d}$ of $labMult(\mathbf{c_1},\mathbf{c_2})$, it decrypts the product as $m_3=Dec_{sk}(\mathbf{d})+Dec_{sk}(c_1)\cdot Dec_{sk}(c_2) = m_1\cdot m_2$ .    \end{itemize}
In this paper we propose an efficient way of extending the $labMult()$ operation for a n-way multiplication in section \ref{implementation}.
\subsubsection{Secure Computation}
%\arc{Not final:placeholder}
Garbled circuit, also known as Yao's protocol \cite{Yao,yao2},  is a generic method for secure multi-party computation. In this paper we just discuss the two-party scenario but it can be generalized to $n > 2$ parties. In its basic form, a garbled circuit allows two-party evaluation of a function $f(x_1,x_2)$ in the presence of semi-honest adversaries. The protocol is run between two data owners with respective private inputs $x_1$ and $x_2$ such that at the end, no party learns more  
than what is revealed from the output value $f(x_1,x_2)$. In the protocol, one of the parties called
garbler or generator, builds a "garbled" version of a circuit computing $f$ and sends it over to the second party, called evaluator, alongside the garbled input values 
corresponding to $x_1$.  Upon receiving the circuit, the evaluator 
engages in an oblivious transfer protocol with the generator to obliviously obtain the garbled input for $x_2$. Finally the evaluator can securely calculate the desired output $f(x_1, x_2)$ from the garbled circuit and the corresponding garbled inputs for $\{x_1,x_2\}$.
\begin{comment}\subsection{Data \& Queries}
\subsubsection{One-hot-encoding} - One-hot-coding is a way of representation for categorical attributes. First the data is converted to an integral representation as follows. Let us assume that the total number of categories is $k$ then each category is represented by an unique integer in $\{1,..., k\}$. Now the one-hot-encoding for a category $x$ with integral representation $t , t \in [k]$ is given by a $k$-lengthed vector $\tilde{x}$ such that $tilde{x}[t]=1, \forall i \in [k], i\neq t, \tilde{x}[i]=0$. 
For example consider an attribute $Age$ with domain $\{1,...,100\}$. In this case since the categories itself are integral, we can skip the first step. Thus the one-hot-encoding corresponding to a value $x=30, type(x)=Age$ is given by $\tilde{x}[30]=1, $equivalent to. In fact the one-hot-coding can be generalized to  represent data across $n$ different attributes. For example, in addition to the aforementioned $Age$ attribute consider  another attribute  $Gender$ with domain $\{Male, Female, Other\}$. Let the categorical values $\textit{Male, Female }$ and $Other$ be represented by integers $1, 2$ and $3$ respectively. Thus a one-hot-encoding for a value $X=<1,30> \in Age \times Gender$ is given by \begin{gather*}i \in [300] \\\tilde{X}[i]=\begin{cases}1, \textit{ if } i =30,\\ 0 , \textit{ otherwise }\end{cases}\end{gather*}
\begin{comment}\subsubsection{Unary Query Type}
We support unary queries that is queries that operate on a single dataset
\am{Single Table, categorical attributes, in tabular form (and not in vector form), Counting queries ...}
Let condition formula, $\phi$ be a Boolean
condition that can be evaluated on any tuple of $D$ and let $\phi(D )$ 
denote the number of tuples in $D$ for which $\phi$ is true. A number of
operators in Crypt$\epsilon$ answer linear queries over the table. A linear
query is the linear combination of any finite set of condition counts:
\begin{definition}
 (Linear counting query (declarative)). A linear query
$q$ on $D$ is defined by conditions $\phi_1 \ldots \phi_k$ and coefficients $c_1 \ldots c_k \in \mathcal{Z}$
 and returns \begin{equation}q(D ) = c_1\phi_1(D) + \ldots + c_k\phi_k (D ) \label{countingq}\end{equation}
\end{definition}
\end{comment}