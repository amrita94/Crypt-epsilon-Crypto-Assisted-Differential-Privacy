
\section{Background}
In this section we give a brief introduction to definitions and primitives relevant to \system. 

\subsection{Differential Privacy}
\begin{definition} An algorithm $\mathcal{A}$
satisfies $\epsilon$-differential privacy ($\epsilon$-DP), where $\epsilon > 0$ is a privacy parameter, iff
 for any two datasets $D$ and $D'$ that differ in a single record, we have
\begin{gather}
\forall t \in Range(\mathcal{A}), Pr \big[\mathcal{A}(D) = t\big] \leq e^{\epsilon}Pr\big[\mathcal{A}(D') = t\big]
\end{gather}
where $Range(\mathcal{A})$ denotes the set of all possible outputs
$\mathcal{A}$.
\end{definition} 
When applied multiple times, the differential privacy guarantee degrades gracefully as follows.
\begin{theorem}(Sequential Composition). If $\mathcal{A}_1$ and
$\mathcal{A}_2$ are $\epsilon_1$-DP and $\epsilon_2$-DP algorithms that use independent randomness, then releasing the outputs $\mathcal{A}_1(D)$ and
$\mathcal{A}_2(D)$ on database $D$ satisfies $\epsilon_1+\epsilon_2$-DP.\end{theorem} 
%There exist advanced composition theorems that give tighter privacy lossbounds, but weuse Theorem 1 for our paper.
Another important result for differential privacy is that any post-processing computation performed on the noisy output of a differentially private algorithm does not cause any loss in privacy.
\begin{theorem}(Post-Processing). Let $\mathcal{A}: D \mapsto R$ be a randomized
algorithm that is $\epsilon$-DP. Let $f : R \mapsto R'$ be an
arbitrary randomized mapping. Then $f \circ \mathcal{A} : D \mapsto R'$ is $\epsilon$-
DP. \label{post}\end{theorem}
We also define the \emph{stability} of a data transformation operation.
\begin{definition}A transformation $\mathcal{T}$ is defined to be $t$-stable if for two datasets $D$ and $D'$, we have\begin{gather}|\mathcal{T}(D)\ominus \mathcal{T}(D')| \leq t \cdot |D\ominus D'|  \end{gather} where  (i.e.,  $D \ominus D' = (D-D') \cup (D'-D)$. \end{definition}
Transformations with bounded stability scale the differential privacy guarantee of their outputs, by their stability constant \cite{PINQ}.
\begin{theorem}
Given $\mathcal{T}$ be an arbitrary $t$-stable transformation on dataset $D$ and an $\epsilon$-DP algorithm $\mathcal{A}$ which takes output of $\mathcal{T}$ as input, the composite computation $\mathcal{A} \circ \mathcal{T}$ provides $(\epsilon \cdot t)$-DP.\end{theorem}

%For other basic properties of differential privacy we refer the readers to the brilliant book by Dwork et al. \cite{Dork}.
\subsection{Cryptographic Primitives}
\stitle{Linearly Homomorphic Encryption (\textsf{LHE}).}
Let $(\mathcal{M}, +)$ be a finite group. A \textsf{LHE} scheme
for messages in $\mathcal{M}$ is defined  as \squishlist
\item \textbf{Key Generation }($Gen$) -This  algorithm takes the security parameter $\kappa$ as input and outputs
a pair of secret and public keys, $(s_k, p_k) \leftarrow Gen(\kappa)$.
\item \textbf{Encryption} ($Enc$) - This is a randomized algorithm that encrypts a message $m \in \mathcal{M}$ via the public key $p_k$, to generate ciphertext $\mathbf{c} \leftarrow Enc_{pk}(m)$.
\item \textbf{Decryption} ($Dec$) - This is a deterministic function that uses the secret key $s_k$ to
recover the plaintext $m$ from ciphertext $\mathbf{c}$.
\squishend
In addition, \textsf{LHE} supports the operator $\oplus$ that allows the summation of ciphers as follows:
\\ \textbf{Operator} $\oplus$ - Let $c_1 \leftarrow Enc_{pk}(m1), \ldots, c_a \leftarrow Enc_{pk}(m_a), a \in \mathcal{Z}_{>0}$. Then we have  $Dec_{sk}(c_1\oplus c_2 ...\oplus c_a)=    m_1 + \ldots   + m_a$.  \\
One can multiply a cipher $c\leftarrow  Enc_{sk}(m)$ by a plaintext positive integer $a$ by $a$ repetitions of $\oplus$. We denote this operation by $cMult(a,c)$ such that $Dec_{sk}\big(cMult(a,c)\big)=a\cdot m$.\\
\stitle{Labeled Homomorphic Encryption(\textsf{labHE}).}
Let $(Gen,Enc,Dec)$ be an \textsf{LHE} scheme with security parameter $\kappa$ and message space $\mathcal{M}$. Assume that a multiplication operation is given in $\mathcal{M}$, i.e., is a finite ring. Let also $\mathcal{F}:\{0,1\}^s \times \mathcal{L}\rightarrow \mathcal{M}$ be a pseudo-random function with seed space $\{0,1\}^s$( s= poly($\kappa $)) and the label space $\mathcal{L}$. A \textsf{labHE} scheme is defined as
\squishlist
 \item $\textbf{labGen}(\kappa)$ - On input $\kappa$, it runs $Gen(\kappa)$ and outputs $(sk,pk)$
\item $\textbf{localGen}(pk)$ -  For each user $i$ and with the public key as input, it samples a random seed $\sigma_i \in \{0,1\}^s$ and computes $pk_i = Enc_{pk}(\underline{\sigma_i})$ where $\underline{\sigma_i}$ is an  encoding of $\sigma_i$ as an  element of $\mathcal{M}$. It outputs $(\sigma_i,pk_i)$.
\item $\textbf{labEnc}_{pk}(\sigma_i, m , \tau)$: On input a message $m \in \mathcal{M} $ with label $\tau \in \mathcal{L}$  from user $i$, it computes $b=\mathcal{F}(\sigma_i, \tau)$ (known as the mask) and outputs the labeled ciphertext $\mathbf{c}=(a,d) \in \mathcal{M} \times \mathcal{C}$ with $ a= m- b$  (known as the hidden message) in $\mathcal{M}$ and $d=Enc_{pk}(b)$. For brevity we just use notation $\textbf{labEnc}_{pk}(m)$ to denote the above functionality, in the rest of paper. 
\item $\textbf{labDec}_{sk}(\mathbf{c})$ - This functions inputs a cipher $\mathbf{c}=(a,d) \in \mathcal{M} \times \mathcal{C}$ encrypted under  \textsf{labHE} and decrypts it as $m=a-Dec_{sk}(d)$.\squishend
Both \textsf{LHE} and \textsf{labHE} provides semantic security guarantee \cite{Katz}.
\stitle{Secure Computation}
%\arc{Not final:placeholder}
Garbled circuit, also known as Yao's protocol \cite{Yao,Yao2},  is a generic method for secure computation. Two data owners with respective private inputs $x_1$ and $x_2$ run the protocol such that, no party learns more  
than what is revealed from the output value $f(x_1,x_2)$ for a function $f$.  One of the data owners, called
generator, builds a "garbled" version of a circuit computing $f$ and sends it over to the other data owner, called evaluator, alongside the garbled input values for $x_1$.  Upon receiving the circuit, the evaluator 
engages in an oblivious transfer protocol with the generator to obliviously obtain the garbled input for $x_2$ and subsequently computes the  output $f(x_1, x_2)$.
%A more detailed discussion is presented in Appendix A.
