\section{Introduction}


There is a growing need for releasing aggregate properties from sensitive datasets in several domains  including social science, healthcare, and advertising. Differentially private algorithms \cite{dwork}, whose outputs are insensitive to adding or removing a single row in the input dataset, have become the gold standard for these situations. Differential privacy provides a provable and persuasive guarantee of privacy to individuals in a dataset, and has seen adoption by government \cite{machanavajjhala08onthemap,Vilhuber17Proceedings} and commercial organizations \cite{Rappor1,Apple, Samsung}. %It is defined with respect to a privacy parameter $\epsilon > 0$ where lower the value of $\epsilon$, greater is the privacy guaranteed.

Differential privacy (DP) is typically implemented in one of two models -- \textit{centralized differential privacy}, or \cdp, and \textit{local differential privacy}, or \ldp. In \cdp, data from individuals are collected and stored \textit{in the clear} in a \textit{trusted} centralized data curator. The trusted curator executes DP programs on the sensitive data  and releases outputs to a mistrustful data analyst. %A canonical algorithm in \cdp is the Laplace mechanism where the curator releases the output of a function $f$ by adding noise drawn from the Laplace distribution to hide the presence or absence of one row in the input database. 

%Depending on infrastructural constraints, like the viable trust model in a given setting, differential private algorithms in practice can have two different styles of implementation.  The more common and historically precedent implementation is the central differential privacy (\textsf{CDP}) model where a trusted data curator collates data from all individuals into a centrally held dataset and processes it in a privacy preserving way. %For example, the data curator could publish differentially private statistics of the data, that allows analysis on the data, without revealing individual information. 
%The curator is trusted to store the data \emph{in the clear} and mediates upon queries posed by a mistrustful analyst, interested in learning some synopsis of this dataset. Privacy is enforced by the curator by adding uncertainty to the answers for analyst's queries before releasing them. The other competing model of implementation, is the local differential privacy model (\textsf{LDP}) where the central data aggregating server is untrusted. Thus every data owner has to individually randomize his/her input before communicating it to the central aggregator. Hence the private data is concealed from the untrusted aggregator who attempts to infer statistics about the true population from the perturbed data instead. 

In \ldp, there is no trusted centralized data curator. Rather, each individual perturbs their own data using a (local) differentially private algorithm. The data analyst has direct access to these perturbed data, and uses it to infer aggregate statistics of the datasets. %A canonical \ldp algorithm is the randomized response mechanism \cite{RR} where each data owner flips some of his/her input bits based on a coin toss to provide \emph{plausible deniability} \cite{Dork}.


Recent deployments of DP in commercial organizations \cite{Rappor1, Apple} have preferred  \ldp over \cdp. \cdp's assumption of a trusted server that stores data in the clear is ill-suited for many applications as it constitutes a single point of failure for data breaches, and saddles the trusted curator with legal and ethical obligations to keep the user data secure. %For instance, Google Chrome uses the \ldp model rather than \cdp to detect changes in browser properties of its userbase as it does not want the legal burden of storing highly sensitive browser fingerprints in the clear on its servers \cite{Rappor1}. 
  

However, \ldp's attractive security properties come with a utilitarian price tag. All DP algorithms ensure privacy by introducing noise into the computation. Under the \cdp model, one can release an aggregate count over the dataset having $n$ rows with an expected additive error of at most $\Theta(1/\epsilon)$ and ensure $\epsilon$-DP (e.g., using the Laplace mechanism \cite{dwork}). In contrast, under the \ldp model, at least $\Omega(\sqrt{n}/\epsilon)$ additive error in expectation must be incurred by any $\epsilon$-DP program for this task \cite{error1,error2,error3}, owing to the individual coin tosses of each data owner \cite{Prochlo,Rappor1,Rappor2,LDP1}. %This is because, each individual perturbs their data values independently in the \ldp model, contributing to the increased noise . 

%There has been a growing line of work \cite{}{} aimed at 
%Thus, on a dataset with a billion individuals, properties that are common to a population as large as 30,000 individuals can be missed under \ldp. 

The \ldp model imposes additional penalties in terms of the  algorithmic expressibility.  Kasiviswanathan et al. in \cite{Kasivi} showed that the power of \ldp is equivalent to that of the statistical query model \cite{SQ1} from learning theory and there exists an exponential separation between the accuracy and sample complexity of local and centralized DP algorithms.  As a consequence, \ldp requires enormous amounts of data to obtain reliable population statistics. Unsurprisingly, only large corporations  like Google \cite{Rappor1,Rappor2,Prochlo} and Apple \cite{Apple} have attempted deploying \ldp.
 
\eat{  Specifically, computation in \textsf{LDP} results in an additional error of $\Omega(\sqrt{n})$ where $n$ is the total number of data owners contributing to the noisy estimate \cite{error1,error2,error3}. In contrast, we get a constant error bound for \textsf{CDP}.
For e.g., for a single counting query, in the \textsf{CDP} model, the trusted data curator first computes the true count in the clear. Next, adding a single instance of noise from the distribution $Lap(\frac{1}{\epsilon})$ to this true count guarantees $\epsilon$-differential privacy. Since the s.t.d of the distribution $Lap(\frac{1}{\epsilon})$ is given by $\frac{1}{\epsilon}$, we get an expected error of $O(\frac{1}{\epsilon})$. On the other hand, owing to the independent coin flips of each reporting data owner, the resulting noise in \textsf{LDP} induces a binomial distribution. This binomial distribution can be approximated by a normal distribution; the magnitude of this random Gaussian
noise however can be very large, its
standard deviation grows in proportion to the square root of
the report count $ \Omega\big(\frac{\sqrt{n}}{\epsilon}\big)$, and the noise is in practice higher by an
order of magnitude \cite{Prochlo,Rappor1,Rappor2,LDP1}. %often growing with the data dimension

 Thus, if a billion individuals'
reports are analyzed, then a common signal from even
up to a million reports may be missed. 


The construct of the \textsf{LDP} model in fact imposes additional penalties in terms of the  algorithmic expressibility.  Kasiviswanathan et al. in \cite{Kasivi} showed that the power of \textsf{LDP} is equivalent to that of the statistical query model \cite{SQ1} from learning theory and there exists an exponential separation between the accuracy and sample complexity of local and central algorithms.  As a consequence, \textsf{LDP} requires enormous amounts of data \cite{Kasivi}
to obtain reliable population statistics. Unsurprisingly, only large corporations  like Google \cite{Rappor1,Rappor2,Prochlo} and Apple \cite{Apple}, with  billions of user base have had successful commercial deployment of \textsf{LDP}.
} %More abstractly, the power of the local model is equivalent tothe statistical query model from learning theory and t.
Thus, naturally the above discussion motivates a quest for a system that can achieve the \textit{best of both worlds}, i.e., high accuracy guarantees and algorithmic expressibility of the \cdp model without having to trust a third-party server.
Conceptually, the aforementioned goal can be achieved by implementing the required differentially private mechanism using off-the-self secure-multiparty computation (SMPC) tools like \cite{EMP,MPCtools,ScaleMAMBA,ABY}. This notion is similar in spirit with the works in \cite{DworkOurData,BeimelSFE+DP}. For example, consider a database schema  $\langle\textit{Age,Sex,NativeCountry,Race}\rangle$ and a garbled-circuit based SMPC implementation. Let a data analyst be interested in learning the output of the query "Count the number of male employees of Mexico having age 30". This can be implemented via a garbled circuit that takes in the database records as input, computes the true answer and outputs it after adding an instance of suitable Laplace noise generated in the given distributed setting \cite{DworkOurData}. However, when it comes to real world deployment, such approaches are rife with challenges. \textit{Firstly}, the performance of naive SMPC protocols can degrade rapidly with increase in the dataset size. This would result in prohibitive computation costs for any practical implementation where datasets of size in thousands have to regularly dealt with. \textit{Secondly}, without a modular back end framework, the implementation for every DP algorithm has to be ad-hoc. For example, revisiting our above mentioned database schema, let us assume that the data analyst is additionally interested in learning the output of a second query "Count the number of age values having at least 100 records". Typically, both of these queries will require two separate garbled circuits. Thus, in a practical set-up, to be able to answer multiple queries on-the-fly the data analyst has to be well-versed with the technical know-how of the MPC primitives as well as the relevant software package. Moreover, the problem is compounded in our setting as there exists no one-size-fits-all solution for achieving utility-optimal differential privacy guarantees for arbitrary algorithms. This makes the real-world deployment of  such a system extremely challenging for non-experts (in SMPC and DP). \textit{Finally}, the security (privacy) proofs for just stand-alone cryptographic and DP mechanisms can be notoriously tricky \cite{BellareCryptoError,DPSVTProof}. Combining the two thus exacerbates the technical complexity, making the design vulnerable to faulty proofs.
 
In this paper, we strive to bridge the gap between LDP and CDP by proposing a system that is at the same time practical for real world usage. We propose, \system, a system for executing differentially private programs that: 
\squishlist
\item never stores or computes on sensitive data in the clear, and still
\item achieves the accuracy guarantees and algorithmic expressibility of the \cdp model \squishend 
In \system a single trusted data curator is replaced by a pair of untrusted but non-colluding servers -- the Analytics Server (\textsf{AS}) and the Cryptographic Service Provider (\textsf{CSP}). The \textsf{AS} plays the role of  data curator in \cdp in executing the differentially private programs, but on \textit{encrypted} data records. The \textsf{CSP} initializes and manages the cryptographic primitives, and collaborates with the \textsf{AS} when the DP program needs to generate outputs. Under the assumption that the \textsf{AS} and the \textsf{CSP} are semi-honest and do not collude (a common assumption in the cryptography-assisted computation literature \cite{Boneh1,Boneh2,Ridge2,Matrix2,secureML,LReg,Ver}), \system is designed to reveal no extra information beyond what can be learned by the release of the outputs of a differentially private program. This is achieved using cryptographic primitives like linear homomorphic encryption and Yao's garbled circuits. 

 \system provides a data analyst with a programming framework  that permits him/her to author logical DP programs just like in the \cdp model.  Like in prior work \cite{PINQ, FWPINQ, ektelo}, access to the sensitive data is restricted via a set of predefined data transformations operators (inspired by relational algebra) and differentially private measurement operators (Laplace mechanism and Noisy-Max \cite{Dork}). Thus any program that is expressed as a composition of the above operators is automatically guaranteed to differentially private (in \cdp model) giving the analyst a proof of privacy for free. \system programs support constructs like looping and conditionals, and can arbitrarily post-process outputs of measurement operators. State-of-the art DP algorithms under the \cdp model can be  expressed via such logical programs. \system automatically compiles these logical programs into \system protocols that work with the underlying cryptographic primitives on the \textsf{AS} and \textsf{CSP}. Thus, \system's programming framework abstracts out all the low-level physical implementation details from the data analysts; the analysts are relieved of the onus of making a host of complex implementation choices like choosing representation for input data, translating queries to that representation, choosing among available SMPC protocols, key management scheme et al. In fact, this separation of the logical and physical layers in \system comes with an additional advantage, a \system program can be compiled down to an implementation based on the cryptographic primitives of choice. As mentioned above, in this paper we present a prototype \system that is based on the primitives LHE and garbled circuits. However, an alternative implementation based on say secret sharing scheme can be achieved independent of the logical programming framework of \system.
 
For boosting \system's performance, we propose a novel DP indexing optimization that leverages on the fact that since the final output is noisy, the mechanism can tolerate DP leakage of certain intermediate computations. This is tune with the works in \cite{Mazloom:2018:SCD,He:2017:RecordLinkage,Chan:2019:FDO:3310435.3310585,Groce}. The DP index optimization along with three other crypto-engineering optimizations result in significant performance improvement for \system as is demonstrated in Section \ref{sec:experiments}.  

The main contributions of this work are
\squishlist
\item \textbf{New Approach:} We present the design and implementation of \system, a novel system for executing DP programs over encrypted data on two non-colluding untrusted servers. %\system integrates the constant error bounds of \textsf{CDP} with the low trust assumption of \textsf{LDP}.
\item \textbf{Algorithm Expressibility:} \system supports data analysis via a rich class of state-of -the-art differentially private programs expressed in terms of a small set of transformation and measurement operators. Thus, \system achieves the accuracy guarantees of the \cdp model without the need for a trusted curator.  
\item \textbf{Ease Of Use:} \system abstracts out all the low-level implementation details like the choice of input data representation, SMC protocol, key management scheme etc from the data analyst thereby reducing his/her burden of complex decision making. The data analysts are only required to encode the DP program logic in terms of the \system operators. \system automatically compiles this down to the appropriate underlying implementation and provides a DP guarantee (in the \cdp model) for free. 
\item \textbf{Performance Optimizations:} Existing techniques to compile logical \system programs into cryptographic protocols often result in inefficient programs. We present optimizations that speed up computation on encrypted data by at least an order of magnitude. A novel contribution of this work are DP indexing optimizations that leverage on the fact that differentially private programs can reveal statistical information about the data. 
\item \textbf{Practical for Real World Usage} We demonstrate the accuracy and practical efficiency of \system via extensive  evaluation. For the same tasks, \system programs achieve accuracy comparable to \textsf{CDP} and orders of magnitude (at least 2) more accuracy than that of \textsf{LDP}. Moreover, \system is efficient and runs within 5 min for a large class of programs on a dataset with 32,561 rows and 4 attributes. 
\item \textbf{Generalized Multiplication Using Linear Homomorphic encryption:} Our implementations leverages on an efficient method of performing $n$-way multiplications using linear homomorphic encryptions, which maybe of independent interest.
\squishend

