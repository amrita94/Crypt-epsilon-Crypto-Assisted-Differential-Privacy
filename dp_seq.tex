\section{Background cntd}
\stitle{Differential Privacy Cntd}\begin{theorem}(Sequential Composition). If $\mu_1$ and
$\mu_2$ are $\epsilon_1$-DP and $\epsilon_2$-DP algorithms that use independent randomness, then releasing the outputs $\mu_1(D)$ and
$\mu_2(D)$ on database $D$ satisfies $\epsilon_1+\epsilon_2$-DP.\end{theorem} There exist advanced composition theorems that give tighter
bounds on privacy loss, but we
use sequential composition as defined above for our paper.

\stitle{Secure Computation}
Add more details


\stitle{Stability of a transformation}
\begin{definition}A transformation $\mathcal{T}$ is defined to be $t-$ stable if for two datasets $D$ and $D'$, we have\begin{gather}|\mathcal{T}(D)\ominus \mathcal{T}(D')| \leq t \cdot |D\ominus D'|  \end{gather} where $\ominus$ denotes the symmetric difference (i.e.,  $D \ominus D'$ is the set of records in $D$ or $D'$, but not both. \end{definition}
Transformations with bounded stability scale the differential privacy guarantee of their outputs, by their stability constant \cite{PINQ}.
\begin{theorem}
Let $\mu$ provide $\epsilon$-differential privacy, and let $\mathcal{T}$
be an arbitrary $t$-stable transformation. The composite computation $\mu \circ \mathcal{T}$ provides $(\epsilon \cdot t)$-differential privacy.\end{theorem}
\stitle{Noisy Max Cntd}
The Noisy Max algorithm has two fold advantage over the naive implementation of finding the maximum count.
Firstly, noisy-max applies "information minimization" as rather than releasing all the noisy counts
and allowing the analyst to find the max and its index, only the
index corresponding to the maximum is made public.
Secondly, the noise added is much smaller than that in the case of the naive implementation (it has sensitivity $m\cdot \Delta$).