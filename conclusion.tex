\section{Future Work and Conclusion}
%There are a number of future work directions for Crypt$\epsilon$. 
One possible extension of the current work can be the development of a
Crypt$\epsilon$ Compiler. Recall that currently the data analyst spells out the explicit Crypt$\epsilon$ program  (i.e., the sequence of Crypt$\epsilon$ primitives and their arguments) to the AS. Thus a useful future work can be constructing a Crypt$\epsilon$ compiler that takes as input user specified queries in a high-level-language instead and a total privacy budget for the query. The Crypt$epsilon$
compiler should then be able to formalize an optimized Crypt$\epsilon$ program expressed in terms of Crypt$\epsilon$ primitives with automated sensitivity analysis and subsequent optimal per operator privacy budget allocation. 
Another future work direction can be developing a formal performance-accuracy trade-off framework for choosing the optimal privacy budgeting for optimizations. Yet another goal for future work is to support a larger class of queries like joins et al.
\par In this paper we have proposed a new implementation setting for differential privacy that allow us to achieve the constant error bounds of the central differential privacy setting without the need for any trusted server. This is achieved via the assistance of cryptographic primitives specifically linear homomorphic encryption and Yao's garbled circuits. 