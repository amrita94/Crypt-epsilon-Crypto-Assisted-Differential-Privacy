
\section{Future Extensions and Conclusion}
%There are a number of future work directions for Crypt$\epsilon$. 
In this paper we have proposed a new implementation model for differential privacy that achieves the constant error bound and  algorithmic expressibility of \textsf{CDP} without the need for any trusted party. This is achieved via two non-colluding servers with the assistance of cryptographic primitives specifically \textsf{LHE} and garbled circuits. Our proposed system \system can execute a rich class of programs that can run  efficiently by virtue of four optimizations.
\par  One possible extension of the current work can be the development of a
Crypt$\epsilon$ compiler. Recall that currently the data analyst spells out the explicit Crypt$\epsilon$ program  (i.e., the sequence of Crypt$\epsilon$ primitives and their arguments) to the \textsf{AS}. Thus a useful future work can be constructing a compiler that takes as input only a user specified query in a high-level-language and a total privacy budget for the query. The
compiler should then be able to formalize an optimized Crypt$\epsilon$ program expressed in terms of Crypt$\epsilon$ primitives with automated sensitivity analysis and subsequent optimal per measurement primitive privacy budget allocation. 
Another very logical future work can be to support a larger class of programs in \system. For e.g., extension of the current functionality of \system to include aggregation operators such as sum, median, average etc should be easily achievable. Supporting  multi-table queries like joins in \system based on existing works along the lines of \emph{elastic sensitivity} \cite{elastic} etc would also be an useful extension.  Yet another interesting direction can be enabling learning algorithms on \system.   Comparatively simpler algorithms like linear regression can based on a previous work \cite{LReg} which also uses \textsf{LHE} and a two-server model. For this, we need to extend \system with a new primitive that performs matrix multiplications. For more involved models like deep learning, we might need to combine the differential privacy results of \cite{DLDP} with the homomorphic encryption techniques of  CryptoNet \cite{CryptoNet}. As mentioned in section 3.6, an alternative implementation for \system  can be based on secret shares modulo the assumption that both the servers are benefit from learning the differential privacy output. Hence another useful extension might be re-implementing \system with  secret shares. For this, the functionality of the existing primitives would mostly be the same, only the respective implementations will change. 
Yet another extension can be removing the second server (\textsf{CSP}) altogether and instead capturing its functionalities within a trusted execution environment (TEE).
