\section{Introduction}
%Thre has been a growing consensus to papers like Contributions of our work is three fold%$\begin{enumerate}\item we have by achieving the \end{enumerate}
Differential privacy has steadily emerged as a compelling privacy definition that allows us to glean useful information about data alongside providing privacy at the granularity of individuals. %Formal privacy guarantees in tandem with quantifiable utility-privacy trade-offs have been instrumental in making it the de-facto standard for achieving data privacy.
It is essentially a property of a function such that its output  should disallow any inference about the
 presence (or absence equivalently) of any individual record in the function's input dataset \cite{dwork}. 
Formally, it requires that for any outcome of a randomized function, that outcome should be nearly equally likely with and without any one record. \par
Depending on infrastructural constraints, like the viable trust model in a given setting, differential private algorithms in practice can have two different styles of implementation.  The more common and historically precedent implementation is the central differential privacy (\textsf{CDP}) model where a trusted data curator collates data from all individuals into a centrally held dataset and processes it in a privacy preserving way. %For example, the data curator could publish differentially private statistics of the data, that allows analysis on the data, without revealing individual information. 
 The curator is trusted to store the data in the clear and mediates upon queries posed by a mistrustful analyst, interested in learning some synopsis of this dataset. Privacy is enforced by the curator by adding uncertainty (e.g. random noise drawn from Laplace distribution of scale parameter $\frac{1}{\epsilon}$) to the answers for analyst's queries before releasing them.  This differentially private noise has a standard deviation of $O(\frac{1}{\epsilon})$. Herein lies an important observation - since every query requires only a single instance of random noise to be added to the answer (by the curator), central differential privacy can thus achieve only a constant error of $O(\frac{1}{\epsilon})$ where $\epsilon$ is our chosen differential privacy parameter. 
\par Thus a major attribute of the central differential privacy setting is its prerequisite of a centralized server (the data curator) that can be fully trusted to store the data in the clear and release only differentially private computations. However, such a strong trust assumption is ill-suited for many practical scenarios and a single point of data breach in the \textsf{CDP} setting can be catastrophic for the entire dataset. Moreover, in the wake of stringent data protection regulations like \textsf{GDPR}, \textsf{HIPAA} etc, a trusted data curator is saddled with legal and ethical obligations to uphold the privacy of its data, making its practical implementation to be cumbersome. These premises have lead to the rise of a competing model called the local differential privacy (\textsf{LDP}). In this model too, each data owner sends in their data to a central aggregator. However, unlike in the central differential privacy setting, the aggregator in the LDP model is untrusted and every individual has to randomize its inputs before communicating it to the central aggregator. Hence the private data is concealed from the untrusted aggregator who attempts to infer statistics about the true population from the perturbed data instead. LDP techniques thus enable private data analysis without reliance on any  trusted third party. Most notable use cases for LDP include observing most frequent software features and measuring
their performance and failure characteristics; collecting data about users' default browser homepage and search engine et al. \par Regrettably, this ease of adopt-ability for \textsf{LDP} comes with an utilitarian price tag. Since each reporting data owner has to 
perform independent coin flips, the resulting noise induces a binomial distribution. This binomial distribution can be approximated by a normal distribution; the magnitude of this random Gaussian
noise however can be very large, its
standard deviation grows in proportion to the square root of
the report count $ O\big(\frac{\sqrt{n}}{\epsilon}\big)$, and the noise is in practice higher by an
order of magnitude \cite{Prochlo,Rappor1,Rappor2,LDP1}. Thus, if a billion individuals'
reports are analyzed, then a common signal from even
up to a million reports may be missed. The construct of the \textsf{LDP} model in fact imposes additional penalties in terms of the  algorithmic expressibility.  Kasiviswanathan et al. in \cite{Kasivi} showed that the power of the local model is equivalent to that of the statistical query model \cite{SQ1} from learning theory and there exists an exponential separation between the accuracy and sample complexity of local and central algorithms.  As a consequence, the local model requires enormous amounts of data \cite{Kasivi}
to obtain reliable population statistics. Unsurprisingly, only large corporations  like Google \cite{Rappor1,Rappor2,Prochlo} and Apple \cite{Apple}, with  billions of user base have had successful commercial deployment of \textsf{LDP}. %More abstractly, the power of the local model is equivalent tothe statistical query model from learning theory and t.
\par From the above discussion, thus we see that although the minimal trust assumption of \textsf{LDP} makes it conducive for easy practical adoption, its accuracy guarantees are strictly limited as compared to that of \textsf{CDP}. In order to bridge this gap, we propose a new model for differential privacy, Crypt$\epsilon$ that  \begin{enumerate}\item achieves the accuracy guarantees of the \textsf{CDP} model without the need of a trusted server \item provides the same algorithmic expressibility as that of the \textsf{CDP} model \end{enumerate} 
In Crypt$\epsilon$ the differentially private computations are outsourced to two non-colluding but untrusted servers, namely the Analytics Server (\textsf{AS}) and the Cryptographic Service Provider (\textsf{CSP}). The relaxation of not having trusted servers is effectuated by the use of cryptographic primitives, specifically linear homomorphic encryption and Yao's garbled circuits. The \textsf{AS} plays a role akin to that of the data curator in \textsf{CDP} or the data aggregator in \textsf{LDP} while the \textsf{CSP} initializes and manages the cryptographic primtives in \system. \system enables the computation of differentially private algorithms by executing programs expressed as a sequence of \system primitives. At the very outset, the data owners submit their encrypted data to the \textsf{AS} which collates all the data records into a single encrypted database.  On receiving a data query from an external analyst expressed in the form of a \system program, the \textsf{AS} executes it on the encrypted database via some interactions with the \textsf{CSP}.  Crypt$\epsilon$ is designed to reveal no extra information (beside that released by the differentially private outputs itself) to
the two servers under the assumption of  non-collusion (this condition, for example, can be enforced by law). 
%Our solution is  based on linearly-homomorphic
%encryption (LHE) scheme and two instances of two-party Yao's garbled circuit; both these primitives can be implemented in practice via very efficient constructions. Additionally we propose two differentially private optimizations for Crypt$\epsilon$ that provide substantial performance gain over the base case implementation.
The main contributions of this work are\\
$\bullet$\textbf{New approach-} We propose a new two-server model for implementing differential privacy which integrates the constant error bounds of the \textsf{CDP} model with the low trust assumption of the \textsf{LDP} model. \\
$\bullet$ \textbf{Expressive Language} - \system supports a set of $9$ primitives which can be used to compose differentially private programs to answer a large class useful queries. \\
$\bullet$ \textbf{Optimizations} - We propose two differentially private optimizations  for Crypt$\epsilon$ that provide substantial performance gain over the base case implementation.\\
$\bullet $\textbf{Generalized multiplication using linear homomorphic encryption}- We propose a very efficient way of performing $n-$ way multiplications using linear homomorphic encryptions.\\
$\bullet $\textbf{Practical for real-world usage} -  We showcase the practicality of \system via extensive evaluation on real datasets.  
%\xh{The pargraph above should summarize our solution \system (a summary of Section 3) and then follows by our contributions in designing this system.}

%\xh{Add contributions in bullet points: e.g. (1) our system design; (2) support for language; (3) opimization; (4) novel cryptoprimitives on labHE; (5) emprical evaluation to show practicality of the solution on real data sets }
\par The rest of the paper is organized as follows. The next section provides a brief discourse on the necessary background followed by the description of the Crypt$\epsilon$ system in section 3. In section 4 we talk about the functionality and implementation of Crypt$\epsilon$ primitives. Section 5 includes running examples of Crypt$\epsilon$ programs while section 6 proposes two optimizations. We empirically evaluate \system and  review the existing literature in sections 7 and 8 respectively. In section 9 we discuss about some interesting aspects of Crypt$\epsilon$ and finally conclude with future research directions in section 10. 



