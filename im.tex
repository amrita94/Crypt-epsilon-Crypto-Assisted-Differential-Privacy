\section{Implementation}\label{sec:implementaion}
%To demonstrate the use of Crypt$\epsilon$ primitives let us look at the following example.
\label{implementation}
In this section we describe the implementation of Crypt$\epsilon$. First we discuss our novel proposed technique for extending the $labMult$ operation of \textsf{labHE} to support $n > 2$ multiplicands. Then we describe the implementations for each primitive. Last, we use the example programs from the previous section to illustrate the performance of \system.

\subsection{\textbf{General $n$-way Multiplication for \textsf{labHE}}}\label{genlab}
In addition to the operations supported by a \textsf{LHE}  scheme, \textsf{labHE} supports multiplication of two \textsf{labHE} ciphers. 
\\ $\bullet \textbf{labMult}(\mathbf{c}_1,\mathbf{c}_2)$ -
On input two \textsf{labHE} ciphers $\mathbf{c}_1=(a_1,d_1)$ and $\mathbf{c}_2=(a_2,d_2)$, it computes a "multiplication" ciphertext  $\mathbf{e}=labMult(\mathbf{c_1,}$ $\mathbf{c_2})=Enc_{pk}(a_1,a_2)\oplus cMult(d_1,a_2) \oplus cMult(d_2,a_1)$. Observe that $Dec_{sk}(\mathbf{e})=m_1\cdot m_2 -b_1 \cdot b_2$.\\
 $\bullet \textbf{labMultDec}_{sk}(d_1,d_2,\mathbf{e})$ - On input two encrypted masks $d_1,d_2$ of two \textsf{labHE} ciphers $\mathbf{c_1},\mathbf{c_2}$ and the output $\mathbf{e}$ of $labMult(\mathbf{c_1},\mathbf{c_2})$, it decrypts the product as $m_3=Dec_{sk}(\mathbf{e})+Dec_{sk}(d_1)\cdot Dec_{sk}(d_2) = m_1\cdot m_2$ .   \\
In this paper we propose an efficient way of extending the $labMult$ operation for a $n$-way multiplication.
\begin{algorithm}
\caption{$genLabMult$ - generate label for $labMult$}
\begin{algorithmic}[1]
\STATEx
\textbf{Input}: $\mathbf{c_1}=labEnc_{pk}(m_1)=(a_1,d_1)$  where\STATEx $\hspace{0.9cm} a_1= m_1-b_1, d_1=Enc_{pk}(b_1)$\STATEx $\hspace{0.9cm} \mathbf{c_2}=labEnc_{pk}(m_2)$
\STATEx$\hspace{0.9cm} a_2= m_2-b_2, d_2=Enc_{pk}(b_2)$
\STATEx \textbf{Output}: $\mathbf{e}=labEnc_{pk}(m_1\cdot m_2)$ 
\STATEx \textbf{\textsf{AS}:} \STATE Computes $\textbf{e}'=labMult(\mathbf{c_1,c_2}) \oplus Enc_{pk}(r)$ where $r$ is a random mask \STATE Sends $\mathbf{e'},d_1,d_2$ to \textsf{CSP}
\STATEx \textbf{\textsf{CSP}:}
\STATE Decrypts $\mathbf{e'}$, to get $Dec_{sk}(\mathbf{e}')=m_1\cdot m_2 -b_2\cdot b_1 + r$
\STATE Computes $b_1 \cdot b_2$ from $d_1$ and $d_2$.
\STATE Removes $b_1\cdot b_2$ from $e'$ to compute $e''=m_1\cdot m_2+r$
\STATE Picks a seed $\sigma'$ and label $\tau'$
\STATE Computes $\bar{a}=e''-b'=m_1\cdot m_2 +r -b'$, $b'=\mathcal{F}(\sigma',\tau')$ and $d'=Enc_{pk}(b')$
\STATE Send $\bar{e}=(\bar{a},d')$ to \textsf{AS}\STATEx \textbf{\textsf{AS}:}
\STATE Computes true cipher $\mathbf{e}=(a',d')$ where $a'=\bar{a}-r=m_1\cdot m_2 - b'$
 \end{algorithmic}
\end{algorithm}
Let's consider an example case where we want to multiply the respective ciphers of  $4$ messages $\{m_1,m_2,m_3,m_4\} \in \mathcal{M}^4$ and obtain $\mathbf{e}=labEnc_{pk}(m_1\cdot m_2\cdot m_3 \cdot m_4)$. Note that the reason why we can't directly use  $labMult$ for a $4-$ way multiplication is because, the "multiplication" cipher $\mathbf{e}=labMult(\mathbf{c_1},\mathbf{c_2})$ does not have  a corresponding label, i.e., it is not in the correct \textsf{labHE} cipher representation. Thus for generalizing the $labMult$ operation for $n>2$ multiplicands what we have to do is to generate a label and a seed for every intermediary product of two multiplicands. This can be done as shown by Algorithm 1. Note that the mask $r$ protects the value of $m_1\cdot m_2$ from the \textsf{CSP} in step 5. Similarly since $b'$ is not known to the \textsf{AS}, $m_1\cdot m_2$ remains hidden from the \textsf{AS} in step 9. In the aforementioned example, the \textsf{AS} first generates $\mathbf{e_{12}}=labEnc_{pk}(m_1\cdot m_2)$ and $\mathbf{e_{34}}=labEnc_{pk}(m_3\cdot m_4)$ using Algorithm 1. Both of this operations can be done in parallel in just one interaction round between the \textsf{AS} and the \textsf{CSP}. In the next round, now the \textsf{AS} can again use Algorithm 1 with inputs $\mathbf{e_{12}}$ and $\mathbf{e_{34}}$ to obtain the final answer $\mathbf{e}$. %consider a case of mu Now with the true \textsf{labHE} cipher $\mathbf{c}=(a',d')$ for the product the \textsf{AS} can compute further multiplications on it. 
Thus for a generic $n-way$ multiplication the order of multiplication can be, in fact, parallelized as  shown in Figure ~\ref{genlab-fig} (Appendix) to require a total of $\lceil \log n\rceil$ rounds of communication with the \textsf{CSP}.\\
\subsection{Primitive Implementation}
%Now let us explain the implementation details of the aforementioned Crypt$\epsilon$ primitives.  
Here we explain the implementation details of two of the  \system primitives, the rest are covered in Appendix  B.\\ (1)\textbf{ \textsf{GroupByCount }}$\groupbystar_A(\mathbf{\tilde{T}})$- The \textsf{GroupByCount} primitive is implemented by Algorithm ~\ref{groupby-imp}. \begin{algorithm}
\small
\caption{\textsf{GroupByCount }$\groupby_A(\mathbf{\tilde{T}})$}
\begin{algorithmic}[1]
\STATEx
\textbf{Input}: $\mathbf{\tilde{T}}$
\STATEx \textbf{Output}: $\tilde{\encV}$
\STATEx \textbf{\textsf{AS}:} \STATE Computes $\mathbf{V}=\groupbystar_{A}(\encT)$.
\STATE Masks the encrypted histogram $\mathbf{V}$ for attribute $A$ as follows \begin{gather*}\boldsymbol{\mathcal{V}}[i]= \mathbf{V}[i] \oplus labEnc_{pk}(M[i])\\M[i] \in_R [m], i \in [|V|]\end{gather*}
\STATE Sends $\boldsymbol{\mathcal{V}}$ to \textsf{CSP}.
\STATEx \textbf{\textsf{CSP}:}
\STATE Decrypts  $\boldsymbol{\mathcal{V}}$ as $\mathcal{V}[i]=labDec_{sk}(\boldsymbol{\mathcal{V}})$.\STATE Converts each entry of $\mathcal{V}$ to its corresponding one-hot-coding and encrypts it, $\boldsymbol{\tilde{\mathcal{V}}}[i]=labEnc_{pk}(\tilde{\mathcal{V}[i]})$
\STATE Sends $\boldsymbol{\tilde{\mathcal{V}}}$ to \textsf{AS}.
\STATEx \textbf{\textsf{AS}}:
\STATE  Rotates every entry by its corresponding mask value to obtain the desired  encrypted one-hot-coding $\boldsymbol{\tilde{V}}[i]$. \begin{gather*}\boldsymbol{\tilde{V}}[i]=RightRotate(\boldsymbol{\tilde{\mathcal{V}}},M[i])\end{gather*} 
 \end{algorithmic} \label{groupby-imp}
\end{algorithm} 
In the first step the \textsf{AS} uses the $\textsf{GroupByCount}^*$ primitive to generate the encrypted histogram $\encV$ for attribute $A$. Note that since each entry of $\mathbf{V}$ is a count of records, its value ranges from $\{0,...,m\}$. The \textsf{AS} then masks $\encV$ (step 2) and sends it to the \textsf{CSP}. The purpose of this mask is to hide the true histogram from the \textsf{CSP}. Next the \textsf{CSP} generates the encrypted one-hot-coding representation for this masked histogram $\boldsymbol{\tilde{\mathcal{V}}}$ (steps 4-5) and returns it back to the \textsf{AS}. %Notice that each entry of $\boldsymbol{\tilde{\mathcal{V}}}$ is a $m$-length vector. 
The \textsf{AS} can simply rotate $\boldsymbol{\tilde{\mathcal{V}}}[i], i \in [|V|]$ by its respective mask value $M[i]$ (step 7) and get back the true encrypted histogram in one-hot-coding $\tilde{\encV}$.
Note that the \textsf{GroupByCount} primitive could have an alternative implementation using a Yao's garbled circuit that takes an input the encrypted vector and outputs the corresponding one-hot-coding representation. However this would require the circuit to decrypt and re-encrypt $O(m)$ data inside it which would be very computationally heavy. 
 \\ (2)\textbf{ \textsf{Laplace }}$\lap_{\epsilon,\Delta}(\mathbf{V})$ -  
\eat{\begin{algorithm}
\small
\caption{\textsf{Laplace }$\lap_{\epsilon,\Delta}(\mathbf{V})$}
\begin{algorithmic}[1]
\STATEx
\textbf{Input}: $\encV$
\STATEx \textbf{Output}: $\hat{V}$
\STATEx \textbf{\textsf{AS}:} \STATE Generates a noisy vector $\hat{\encV}$  as \begin{gather*}\hat{\mathbf{V}}[i] = \mathbf{V}[i]\oplus labEnc_{pk}(\eta[i]),\\ \eta \sim [Lap(\frac{1}{\epsilon})]^{|V|}, i \in [|V|] \end{gather*}
\STATE Sends $\hat{\mathbb{\mathcal{V}}}$  to \textsf{CSP}
\STATEx \textbf{\textsf{CSP}:}
\STATE Decrypts $\mathbf{\hat{\mathcal{V}}}$ to get $\hat{\mathcal{V}}[i]=labDec_{sk}(\mathbf{\hat{\mathcal{V}}}[i]), i \in [|V|]$
\STATE Generates a the final noisy vector $\hat{V}$ as follows 
\begin{gather*} \hat{V}[i]=\hat{\mathcal{V}}[i]+\eta'[i], i \in [|V|], \eta' \sim [Lap(\frac{1}{\epsilon})]^{|\hat{V}|} \end{gather*}
\STATE Returns $\hat{V}$ to \textsf{AS}
 \end{algorithmic} \label{lap}
\end{algorithm}} %The implementation for the \textsf{Laplace} primitive is given by Algorithm ~\ref{lap}.
Recall that both \textsf{AS} and \textsf{CSP} have to add Laplace noise to the output in Crypt$\epsilon$. Hence the \textsf{Laplace} primitive has two phase. In the first phase,  the \textsf{AS} adds an instance of encrypted Laplace noise to the encrypted input \eat{(step 1 in Algorithm ~\ref{lap})} to generate $\mathbf{\hat{\mathcal{V}}}$. This acts as the input to the second phase which is executed by the \textsf{CSP}. Here the \textsf{CSP} decrypts $\mathbf{\hat{\mathcal{V}}}$ and adds a second instance of the Laplace noise to generate the final noisy output $\hat{V}$ in the clear. %(steps 3-4)
 The \textsf{Laplace} primitive with an encrypted scalar $\encC$ as the input is implemented in a similar way.\\
{\textbf{Note:}} Thus in Crypt$\epsilon$ we add two separate instances of the Laplace noise as opposed to just one instance of noise like in the traditional \textsf{CDP} setting. Thus although the errors of both Crypt$\epsilon$ and \textsf{CDP} are of the same order $O\big(\frac{1}{\epsilon}\big)$, yet quantitatively the error values of Crypt$\epsilon$ are twice that of the traditional $\textsf{CDP}$ setting. This can be addressed by a joint computation of a single instance of noise by both \textsf{AS} and \textsf{CSP} as discussed in appendix E.1. Another observation is that this double noise addition does not affect the differential privacy guarantee. After the addition of the first instance of noise by the \textsf{AS}, the second can be seen as a post-processing step and differential privacy is post-processing immune. Hence our results Crypt$\epsilon$ programs are still differentially private.
%\textit{Note:} Following our discussion on the primitive implementations in this section and Appendix C, we see that the major chunk of the work for almost all the transformation primitives is carried out the \textsf{AS} by itself. This conforms to our second requirement in section.
\subsection{Classification of \system Programs}\label{sec:interactions}
Crypt$\epsilon$ programs can be grouped into three classes based on the number and type of interaction between the \textsf{AS} and the \textsf{CSP}.  \\
(1)\stitle{ \textbf{Class I - Single Decrypt Interaction Programs:}}\\
Recall that the output of all the transformation primitives are encrypted.  Since  the \textsf{CSP} has exclusive access to the secret key, it is the only entity in the Crypt$\epsilon$ setting capable of decryption. Thus for releasing any result (albeit noisy) in the clear, we need to interact at least once with the \textsf{CSP} so that it can decrypt the encrypted noisy answer. Crypt$\epsilon$ supports this type of interactions via the two measurement primitives. Some Crypt$\epsilon$ programs require only one round of interaction at the very end to release the noisy output. All other transformations can be performed by the \textsf{AS} via homomorphic operations on the encrypted data records. Typically these programs are counting queries on a single attribute or noisy max on a single attribute. Examples of this type of programs are P1 and P2 from Table \ref{tab:programexamples}.\\
(2)\stitle{ \textbf{Class II : \textsf{LabHE} Multiplication Interaction Programs-}}\\
Recall that labeled homomorphic encryption allows multiplication of two ciphers. Generalization to a $n$-multiplicants, $n > 2$ case can be done using the protocol described in section \ref{genlab}. However it requires intermediate interactions with the \textsf{CSP}. Thus all Crypt$\epsilon$ programs that require multiplication of more than two ciphers need interaction with the \textsf{CSP}. 
All programs with more than three attributes in its Boolean predicate would thus fall under this class. P3, P4 and P5 from table 3 fall in this class of \system programs.\\
(3)\stitle{ \textbf{Class III : Other Interaction Programs-}}\\
 The \textsf{GroupBy} primitive requires an intermediate interaction with the \textsf{CSP} (for generating the encrypted one-hot-coding). The \textsf{CountDistinct} primitive also uses a Yao's garbled circuit (see section C) and hence requires  interaction with the \textsf{CSP}. This is in addition to the interaction required for decrypting the noisy answer (as explained above). Thus any program with the \textsf{GroupBy} or \textsf{CountDistinct} primitive requires two rounds of interaction in the least. P6 and P7 from Table \ref{tab:programexamples} are examples of this class of \system programs. \\
