%!TEX root = main.tex

\section{\system Primitives}\label{sec:primitives}
\system provides a set of primitives for the data analyst.
A sequence of these primitives forms a program which can be run by \system.
In this section, we first define these primitives and then show how to use these primitives to write  programs.
The analyst is given a database schema $\langle A_1,\ldots,A_k \rangle$ and an encrypted instance of this database, $\encD$.
There are two types of Crypt$\epsilon$ primitives:  (i) transformations and (ii) measurements, which are summarized in Table~\ref{tab:primitives}.

% Please add the following required packages to your document preamble:
% \usepackage{multirow}


\begin{table*}[]
\small{
\caption {\system Primitives}\label{tab:primitives}
\begin{tabular}{|l|l|l|l|l|l|}
\hline
\bf{Types}                           & \bf{Name}         & \bf{Notation} & \bf{Input} & \bf{Output} & \bf{Functionality} \\ \hline \hline
\multirow{7}{*}{Transformation} & \multirow{2}{*}{\textsf{CrossProduct}} &  \multirow{2}{*}{$\crossproduct_{A_i,A_j\rightarrow A'}(\cdot)$} & 

\multirow{2}{*}{$\encT$}   &  \multirow{2}{*}{$\encT'$}   & Generates a new attribute  $A'$ (in one-hot-coding) to represent\\ & & & & & the data for both the attributes $A_i$ and $A_j$   \\ \cline{2-6}
                                & \textsf{Project}     & $\project_{A^*}(\cdot)$  &  $\encT$       &  $\encT'$       &  Discards all attributes but $A*$ \\ \cline{2-6}
                                & \textsf{Filter}       & $\filter_{\phi}(\cdot)$   &  $\encT$      &   $\encB'$     &  Zeros out records not satisfying $\phi$    in $\encB$          \\ \cline{2-6}
                                & \textsf{Count}             & $\countagg(\cdot)$         & $\encT$  &  $\encC$    &  Counts the number of 1s in $\encB$               \\ \cline{2-6}
                                & \textsf{GroupByCount*}             & $\groupbystar_{A}(\cdot)$ &  $\encT$  & $\encV$    & Returns encrypted histogram of $A$                \\ \cline{2-6}
                                & \textsf{GroupByCount}              & $\groupbystar_{A}(\cdot)$    &$\encT$       & $\tilde{\encV}$        &  Returns encrypted histogram of $A$ in one-hot-encoding    \\ \cline{2-6}
                                & \textsf{CountDistinct}             & $\countdistinct(\cdot)$     &    $\encV$   & $\encC$   &  Counts the number of non-zero values in $\encV$       \\ \hline
\multirow{2}{*}{Measurement}    & \textsf{Laplace}     & $\lap_{\epsilon,\Delta}(\cdot)$    &  $\encV$      &   $\hat{V}$    &  Adds Laplace noise to $\encV$    \\ \cline{2-6}
                                & \textsf{NoisyMax}     & $\noisymax_{\epsilon,\Delta}^k(\cdot)$         & $\encV$       &  $\hat{\mathcal{P}}$      &  %Adds Laplace noise to $\encV$ and 
                                Returns indices of the top $k$ noisy values              \\ \hline
\end{tabular}
}
\end{table*}


\eat{
\begin{table*}[h!]
\small
\caption {\system Primitives}
 \begin{tabular}{@{}l c@{}c@{}l@{}}  \toprule
\multicolumn{1}{c}{\textbf{Primitives}} & \textbf{Input}  & \textbf{Output}  & \multicolumn{1}{c}{\textbf{Functionality}}  \\ [0.5ex] 
 \midrule \midrule \textsf{CrossProduct} &$\tilde{\mathbf{T}}, A_i, A_j$ & $\tilde{\mathbf{T'}}$ & Generates  one-hot-coding for the\\&& & attribute $A_i\times A_j$ 
 \\\textsf{Project}&$\tilde{\mathbf{T}}, A^*$&$\tilde{\mathbf{T'}}$ &Discards all attributes but $A^*$
 \\\textsf{Filter}&$\tilde{\mathbf{T}},\phi$&$\tilde{\mathbf{B}}$&Zeros out records not satisfying $\phi$
 \\\textsf{Count}&$\tilde{\mathbf{T}}$&$\mathbf{C}$&Counts the number of records in $\textbf{B}$ 
 \\$\textsf{GroupBy}^*$ &$\tilde{\mathbf{T}},A$&$\mathbf{V}$& Returns encrypted histogram of $A$
 \\\textsf{GroupBy} &$\tilde{\mathbf{T}},A$ &$\tilde{\mathbf{V}}$ &Returns encrypted histogram of $A$\\&&& in one-hot-coding
 \\\textsf{CountDistinct} &$\mathbf{V}$&$\mathbf{C}$ &Counts the number of non-zero values in $\textbf{V}$
 \\\textsf{Laplace}&$\textbf{V},\epsilon$ &$\hat{\textbf{V}}$ & Both \textsf{CSP} and {AS} adds Laplace noise to $\textbf{V}$ 
 \\\textsf{NoisyMax} & $\textbf{V},\epsilon, k$&$P$ &Outputs top k values from noisy vector $\textbf{V}$\\
  [1ex] 
 \bottomrule
 \end{tabular}
\end{table*}
}



\subsection{Transformation Primitives}\label{sec:transformation_primitives}
Transformation primitives take an encrypted data as input and output a transformed encrypted data.  These primitives thus work completely on the encrypted data and do not expend any privacy budget. Three types of data are considered in this context: (1) an encrypted table of $x$ rows and $y$ columns/attributes, denoted by $\encT$, where each attribute value is represented by encrypted one-hot-encoding of the value; (2) an encrypted vector of numbers, denoted by $\encV$; and (3) an encrypted scalar, denoted by $\encC$. In addition, every encrypted table $\encT$ of $x$ rows has a encrypted bit vector $\encB$ of size $x$ to indicate whether the record is relevant to the program at hand. If the $i^{th}$ bit value of $B$ is $1$, then the $i$th row in $\encT$ will be used for answering the current program and vice versa. The input to the very first transformation primitive in \system program is the encrypted database $\encD$ with all bits of $\encB$ set to be $1$. For brevity, we just use $\encT$ to represent both the encrypted table $\encT$ and $\encB$. The transformation primitives are detailed below.
\\(1)\stitle{ \textsf{CrossProduct}} $\crossproduct_{(A_i,A_j)\rightarrow A'}(\encT)$: This primitive transforms the two encrypted one-hot-encodings for attributes $A_i$ and $A_j$ in $\encT$ into a single encrypted one-hot-encoding of a new attribute $A'$. The domain of the new attribute $A'$ is the cross product of the domains for $A_i$ and $A_j$. The resulting table $\encT'$ has one column less than $\encT$. Thus, the construction of the one-hot-coding of the entire $y$-dimensional domain can be computed by repeated application of this primitive. \\	
%1) \textbf{\textsf{CrossProduct}}:$\crossproduct_{A_1,A_2}(\tilde{\mathbf{T}})$ - Given encrypted one-hot-codings for two different attributes $A_1$ and $A_2$ of domain sizes $s_{A_1}$ and $s_{A_2}$ respectively, the goal of this transformation is to compute the encrypted one-hot-coding for the entire two-dimension domain of the new $\lq$attribute' $A_1\times A_2$ of size $s_{A_1}\cdot s_{A_2}$. Thus this transformation takes as input a $x \times y $ table, $\tilde{\mathbf{T}}$ defined over attribute set $A=\{A_1,A_2,...,A_y\}$ where each cell $\tilde{\mathbf{T}}[i,j] , i \in [x], j \in [y], 2 \leq y \leq k$ corresponds to the encrypted one-hot-coding for attribute $A_j$ for the data owner $\textsf{DO}_i$ and outputs a $x \times (y-1)$ table with attribute set $\{A_1\times A_2,A_3,\ldots,A_{y}\}$.  Note that the construction of the one-hot-coding of the full $y$-dimension domain can be computed by repeated application of this transform. 	
    (2)\stitle{ \textsf{Project}} $\project_{\bar{A}}(\encT)$: This primitive projects $\encT$ on  a subset of attributes $\bar{A}$ of the input table. All the attributes that are not in $\bar{A}$ are discarded in the output table $\encT'$.\\
	%2) \textbf{\textsf{Project}} : $\project_{A^*}(\tilde{\mathbf{T}})$- In addition to the $ x \times y$ table, $\tilde{\mathbf{T}}$ over attribute set $\{A_1, A_2, ..., A_y\}$, the \textsf{Project} transformation takes a set of attributes $A*=\{A^*_1,...A^*_p\}, p < y$ as inputs. The result of the transformation is defined as the $x \times p$ data source table where each record is just restricted to the attribute set $A^*$, i.e., it discards all other attributes. 
	%Infact it is analogous to the operation of marginalization which is described as follows.
	%Assuming  $A$ and $B$ to be two attributes with finite domains, let $x$ be a vector of counts representing a histogram over the cross product of the domain (with $|A|*|B|$ entries).
	%Marginalization over the attribute $B$ results in a vector of counts on the attribute $A$ alone by adding up counts corresponding to the same value of $A$.  
	(3)\stitle{ \textsf{Filter}} $\filter_{\phi}(\encT)$: This primitive specifies a filtering condition, represented by a boolean predicate $\phi$ defined over a subset of attributes $\bar{A}$ of the input table $\encT$. The predicate can be expressed as a conjunction of range conditions over $\bar{A}$, i.e. for a row $r \in [x]$ in $\encT$, $\phi(r) = \bigwedge_{A_i \in \bar{A}} ~~(r.{A_i} \in V_{A_i})$,  where $r.A_i$ is value of attribute $A_i$ in row $r$ and $V_A$ is a subset of values (can be a singleton too) that attribute $A_i$ can take.  For example, $Age\in [30,40]\wedge Gender=M$ can be one such  filtering condition. The \textsf{Filter} primitive affects only the associated encrypted bit vector of $\encT$ and keeps the actual table untouched.  In this primitive, if any row $r \in [x]$ in $\encT$ does not satisfy the filtering condition $\phi$, the corresponding $r^{th}$ bit in $\encB$ will be set to $labEnc_{pk}(0)$; otherwise, the corresponding bit value in $\encB$ is kept unchanged. 
Thus the \textsf{Filter} transformation suppresses all the records that are extraneous to answering the program at hand (i.e., does not satisfy $\phi$) by explicitly zeroing the corresponding indicator bits and outputs the table $\encT'$ with the updated indicator vector.\\
(4)\stitle{ \textsf{Count}} $\countagg(\encT)$: This primitive simply counts the number of rows in $\encT$ that are pertinent to the program at hand, i.e. the number of $1$s in its associated bit vector $\encB$.  This primitive outputs an encrypted scalar $\encC$. \\%Typically, this transformation is followed by a measurement primitive and is immediately preceded by a \textsf{Filter} primitive. \\
% 4) \textbf{\textsf{Count}}$:\countagg(\mathbf{T})$  - The \textsf{Count} transformation outputs the encrypted value of the non-noisy true count for the program at hand. For answering linear counting queries, typically \textsf{Count}  is the last transformation to be applied and is immediately preceded by a \textsf{Filter} transformation. Recall that the \textsf{Filter} transformation sets bit $i \in [m]$ to be 1 (encrypted) if the $i^{th}$ record satisfies the filter condition and 0 otherwise and outputs this encrypted $m\times 1$ vector. Hence the \textsf{Count} primitive simply adds up all the entries of this bit vector $\mathbf{B}$ and  outputs the sum which is a single encrypted value. 
(5)\stitle{ \textsf{GroupByCount*}} $\groupbystar_{A}(\mathbf{\tilde{T}})$: The  \textsf{GroupByCount*} primitive buckets the input table $\mathbf{\tilde{T}}$ into groups of records having the same value for an attribute $A$. The output of this transformation is thus, an encrypted  vector $\mathbf{V}$ which is the encrypted histogram for $A$.
    This primitive serves as a preceding transformation for other Crypt$\epsilon$ primitives such as \textsf{NoisyMax}, \textsf{CountDistinct}.\\
(6)\stitle{ \textsf{GroupByCount}} $\groupby_{A}(\mathbf{\tilde{T}})$ : This primitive is similar to the aforementioned \textsf{GroupByCount*} primitive. The only difference between the two is that, \textsf{GroupByCount} outputs the encrypted histogram of attribute $A$ with each count represented in one-hot-coding, $\tilde{V}$.  This transformation allows us to answer queries based on the count of a particular value for attribute $A$.\\
(7)\stitle{ \textsf{CountDistinct } }$\countdistinct(\mathbf{V})$: As mentioned above, this primitive is always preceded by a \textsf{GroupByCount*} primitive. Hence the input vector $\mathbf{V}$ is an encrypted histogram for some attribute $A$ and this primitive returns the number of distinct values of $A$ that appear in  $\boldsymbol{\tilde{\mathcal{D}}}$ by counting the non-zero entries of $V$.
%The input to Crypt$\epsilon$ is an encrypted instance of a database $\boldsymbol{\tilde{\mathcal{D}}}$ with a single relational schema $\langle \mathcal{A}_1,\mathcal{A}_2, . . . ,\mathcal{A}_l\rangle$. Each attribute $\mathcal{A}_i$ is assumed to be discrete (or suitably discretized) and represented in one-hot-coding form.
\eat{
Transformation primitives take as input an encrypted source variable (a table of size $x \times y, x,y \in \mathcal{Z}_{\geq 0}$) and output a transformed data source (again  a table $x' \times y', x',y' \in \mathcal{Z}_{\geq 0}$) that is still encrypted. Typically $x$ and $x'$ are equal to $m$, the total number of data owners, i.e., every tuple in the data source tables corresponds to the record of a single data owner. In case $x=1$ or $x'=1$ the data source is an encrypted vector and we represent it as $\mathbf{V}$.
The transformation primitives are mostly carried out by the \textsf{AS} on its own; this is enabled by our use of labeled homomorphic encryption scheme which allows us to perform certain operations, specifically multiplication and addition, directly over the encrypted data. %Only two transformations  (\textsf{GroupBy} and \textsf{CountDistinct}) need to be computed via a secure computation protocol between the \textsf{AS} and the \textsf{CSP}. 
Since these primitives work entirely on encrypted data, they do not expend the privacy budget. However these operators can affect the privacy analysis through their stability. Every transformation in Crypt$\epsilon$ has a well-established stability.
For each record of the database $\boldsymbol{\tilde{\mathcal{D}}}$ (i.e., data corresponding to a single data owner) we maintain an encrypted bit which indicates whether the record is relevant to the program at hand. Let $\mathbf{B}$ represent this bit vector where $\mathbf{B}[i]$ corresponds to this indicator bit for the $i^{th}$ record.  If $\mathbf{B}[i] =Enc_{pk}(1)$, then the $i^{th}$ record is to be considered for answering the current program and vice versa. Only one of the transformation, \textsf{Filter} alters the bit vector $\mathbf{B}$. Before every program execution, $\mathbf{B}$ is initialized to a 1-vector. 
\begin{enumerate}

	\item \textsf{CrossProduct} ($\tilde{\mathbf{T}}, A_i, A_j$) - Given encrypted one-hot-codings for two different attributes $A_1$ and $A_2$ of domain sizes $s_{A_1}$ and $s_{A_2}$ respectively, the goal of this transformation is to compute the encrypted one-hot-coding for the entire two-dimension domain of the new $\lq$attribute' $A_1\times A_2$ of size $s_{A_1}\cdot s_{A_2}$. Thus this transformation takes as input a $x \times y $ table, $\tilde{\mathbf{T}}$ defined over attribute set $A=\{A_1,A_2,...,A_y\}$ where each cell $\tilde{\mathbf{T}}[i,j] , i \in [x], j \in [y], 2 \leq y \leq k$ corresponds to the encrypted one-hot-coding for attribute $A_j$ for the data owner $\textsf{DO}_i$ and outputs a $x \times (y-1)$ table with attribute set $\{A_1\times A_2,A_3,\ldots,A_{y}\}$.  Note that the construction of the one-hot-coding of the full $y$-dimension domain can be computed by repeated application of this transform. 	
    
	
	\item \textsf{Project}($\tilde{\mathbf{T}},A^*$)- In addition to the $ x \times y$ table, $\tilde{\mathbf{T}}$ over attribute set $\{A_1, A_2, ..., A_y\}$, the \textsf{Project} transformation takes a set of attributes $A*=\{A^*_1,...A^*_p\}, p < y$ as inputs. The result of the transformation is defined as the $x \times p$ data source table where each record is just restricted to the attribute set $A^*$, i.e., it discards all other attributes. \\
	%Infact it is analogous to the operation of marginalization which is described as follows.
	%Assuming  $A$ and $B$ to be two attributes with finite domains, let $x$ be a vector of counts representing a histogram over the cross product of the domain (with $|A|*|B|$ entries).
	%Marginalization over the attribute $B$ results in a vector of counts on the attribute $A$ alone by adding up counts corresponding to the same value of $A$.  
  \item \textsf{Filter}($\tilde{\mathbf{T}},\phi$) - Let $\tilde{\mathbf{T}}$ be an encrypted table of one-hot-codings over attribute set $A=\{A_1,...,A_k\}$, $\phi$ be a  predicate defined over a subset of attributes $A^*\subseteq A$ and $\mathbf{B}$ be the current state of the indicator vector which is stored by the \textsf{AS}. The predicate $\phi$ has to be expressed as a conjunction of range conditions over $A^*$, i.e.,\begin{gather}\phi = \bigwedge_{A \in A^*}(A \in \{v_{1},\ldots,v_{t}\} ) \label{phi} \end{gather} If for some attribute $A \in A^*$, the condition is a equality condition as $A==v$ instead of a range condition, then simply put $v_1=v_t$. For e.g., a condition of this form is find the number of records that satisfy $Age$ in range [30,40] and $Gender$ is male. For each record $r_i, i \in [m]$, the Filter transformation zeros the corresponding indicator bit $\mathbf{B}[i] $ if $\phi(r_i)=False$. $\mathbf{B}[i] $ is kept unchanged otherwise. Thus the \textsf{Filter} transformation suppresses all the records that are extraneous to answering the program at hand (i.e., does not satisfy $\phi$) by explicitly zeroing the corresponding indicator bits and outputs the updated indicator vector. %It takes as input a $x \times 1$ table $\tilde{\mathbf{T}}$, whose every row is an encrypted one-hot-encoding (of the form $\mathbf{\tilde{R}}$) for the attribute of concern $A$, and a vector $\mathbf{C}$ which has encryptions of appropriate non-zero weights for indices that satisfy $\phi$. %Note that $A$ need not be an attribute of the original attribute set $\mathcal{A}$ but can be a new multi-dimension 'attribute' constructed over $\mathcal{A}^* \subseteq \mathcal{A}$, i.e., $A= \prod_{A*_i \in \mathcal{A}^*  }A^*_i$. 
    \item{\textsf{Count}($\mathbf{T}$) } - The \textsf{Count} transformation outputs the encrypted value of the non-noisy true count for the program at hand. For answering linear counting queries, typically \textsf{Count}  is the last transformation to be applied and is immediately preceded by a \textsf{Filter} transformation. Recall that the \textsf{Filter} transformation sets bit $i \in [m]$ to be 1 (encrypted) if the $i^{th}$ record satisfies the filter condition and 0 otherwise and outputs this encrypted $m\times 1$ vector. Hence the \textsf{Count} primitive simply adds up all the entries of this bit vector $\mathbf{B}$ and  outputs the sum which is a single encrypted value. 
    \item{\textsf{GroupBy*}($\mathbf{\tilde{T}},A$)}- The purpose of the \textsf{GroupBy*} transformation is to essentially bucket the input $x\times y$ table $\mathbf{\tilde{T}}$ into groups of records having the same value for an attribute of choice $A$. The output of this transformation is an $1\times s_A$ encrypted  vector $\mathbf{V}$  where each vector element $\mathbf{V}[i], i \in [s_A]$ represents the encrypted count of the number of records in $\boldsymbol{\tilde{\mathcal{D}}}$ having value $v_{i}$ for attribute $A$. Thus \textsf{GroupBy*} essentially returns an encrypted histogram for $A$.
    This primitive serves as a preceding transformation for other Crypt$\epsilon$ primitives like \textsf{NoisyMax}, \textsf{CountDistinct} et al.
     \item{\textsf{GroupBy}($\mathbf{\tilde{T}},A$)-} The \textsf{GroupBy} transformation is similar to the aforementioned \textsf{GroupBy*} transformation. The only difference between the two is that, the former outputs the encrypted one-hot-coding of the respective counts. That is, the output of \textsf{GroupBy}($A$)  is an $s_A$ lengthed encrypted vector $\tilde{\mathbf{V}}$ such that each element, $\tilde{\mathbf{V}}[i], i \in [s_A]$ represents the encrypted one-hot-encoding of the number of records in $\boldsymbol{\tilde{\mathcal{D}}}$ having value $v_{i}$ for attribuet $A$. This transformation allows us to answer queries based on the count of a particular value for attribute $A$.
     %Note that since for \textsf{GroupBy} we need to create the one-hot-coding of the counts, this requires an interaction with the \textsf{CSP}.
     \item {\textsf{CountDistinct}($\mathbf{V}$)-} As mentioned before, the \textsf{CountDistinct} primitive takes as input an encrypted vector $\mathbf{V}$ which is the output of a \textsf{GroupBy*}($A$) primitive for some attribute $A$. Thus the \textsf{CountDistinct} primitive  returns the number of distinct values of $A$ that appear in the records of $\boldsymbol{\tilde{\mathcal{D}}}$ by counting the non-zero entries of $V$.  
\end{enumerate}
%Note that the first four transformations namely \textsf{CrossProduct, Project, Filter} and Count are performed by the \textsf{AS} alone. Only for transformation \textsf{GroupBy} the \textsf{AS} engages in a secure computation protocol with the \textsf{CSP}.

}
\subsection{Measurement Primitives} \label{sec:measurement_primitives}
The measurement primitives take encrypted vector of counts $\encV$ (or a single count $\encC$) as input and return noisy measurements on it in the clear. These implement two of the most popular differentially private mechanisms: Laplace mechanism and Noisy-Max mechanism.
Both mechanisms add noise $\eta$ drawn from Laplace distribution, denoted by $Lap(b)$, where $\Pr[\eta =x]\propto e^{-{\frac{|x|}{b}}}$. The scale of the noise depends on the transformations applied on the base database $\mathcal{D}$. Let the sequence of transformations ($\bar{\mathcal{T}}=(\mathcal{T}_l,\ldots,\mathcal{T}_1)$) applied on $D$ to get $V$ be $\bar{\mathcal{T}}(D) = \mathcal{T}_l(\cdots \mathcal{T}_2((\mathcal{T}_1(D))))$. We define the \emph{sensitivity} of a sequence of transformations as  the maximum change to the output of this sequence of transformations when changing a row in the input database, i.e.,
$\Delta_{\bar{\mathcal{T}}} = \max_{D,D'} \|\bar{\mathcal{T}}(D)-\bar{\mathcal{T}}(D')\|_1$ where $D$ and $D'$ differ but a single record.
The sensitivity of $\bar{\mathcal{T}}$ can be upper bounded by the product of the stability (definition in Appendix A2) of these transformation primitives, i.e. $\Delta_{\bar{\mathcal{T}}=(\mathcal{T}_l,\ldots,\mathcal{T}_1)} = \prod_{i=1}^l \Delta \mathcal{T}_i$. The transformation operators in Table~\ref{tab:primitives} have a stability of 1, except for \textsf{GroupByCount} and \textsf{GroupByCount*} which are 2-stable. \\
(1)\stitle{ \textsf{Laplace}} $\lap_{\epsilon,\Delta}(\mathbf{V}/\encC)$:  This primitive implements the classic Laplace mechanism \cite{Dork}. Given an encrypted vector $\encV$ or an encrypted scalar $\encC$, a privacy parameter $\epsilon$ and sensitivity $\Delta$ of the preceding transformations, the primitive adds a noise vector (scalar) from $Lap(\frac{\Delta}{\epsilon})$. This primitive ensures $\epsilon$-differential privacy when data analyst views the plaintext answer.\\
(2)\stitle{ \textsf{NoisyMax }} $\noisymax_{\epsilon,\Delta}^k(\mathbf{V})$:  Noisy-Max is a type of differentially private selection mechanism \cite{Dork} where the goal is to determine the query with the highest value out of $n$ different noisy query outputs. The algorithm works as follows. First, generate each of the $N$ answers and then add independent Laplace noise from the distribution $Lap(\frac{\Delta}{\epsilon})$ 
to each of them. The index of the largest noisy value is then reported as the noisy max. 
\eat{
Thus they expend the privacy budget and require secure computation between the \textsf{AS} and the \textsf{CSP}. 
\begin{comment} All measurement operators must involve joint computation with the \textsf{CSP}. Note that the requisite noise to be added to ensure differentially privacy has to be jointly added by both the \textsf{AS} and the \textsf{CSP}. It is so because, had only either one of the servers added the noise, then that server would be able to retrieve the true non-noisy answer by simply de-noising the published differentially private answer. This means that the sensitivity of the program being executed should be known to both the servers. This poses no hindrance in our setting  since the program is public, the  sensitivity computation can be performed very easily by observing the sequence of the preceding transformations.
\end{comment}
\begin{enumerate}
	\item \textsf{Laplace}($\mathbf{V},\epsilon$) - In the Laplace mechanism, in order
to publish $f(D)$ where $f : D \mapsto R$, $\epsilon$-differentially private mechanism $\mathcal{M()}$ 
publishes $f(D) + Lap\Big(\frac{\Delta f}{\epsilon}\Big)$  
where $\Delta f = \max_{D,D'}||f(D)-f(D')||_1$ is known as the sensitivity of the query. The p.d.f of $Lap(b)$ is given by\begin{gather}\mathbf{f}(x)={\frac  {1}{2b}}e^{ \left(-{\frac  {|x-\mu |}{b}}\right)}\end{gather} The sensitivity of the function $f$ basically captures the magnitude by which a single individual's data can change the function $f$ in the worst case. Therefore, intuitively, it captures the uncertainty in the response that we must introduce in order to hide the participation of a single individual. For counting queries the sensitivity is 1. The \textsf{Laplace} primitive enables the \textsf{AS} and the \textsf{CSP} to add two separate instances of random laplace noise to the true result of a counting query for generating a differentially private output. It takes an input an encrypted vector $\mathbf{V}$ (could be a scalar too) and adds two instances of noise drawn from $[Lap(\frac{1}{\epsilon})]^{|V|}$ to it.

	\item \textsf{NoisyMax}($\mathbf{V},\epsilon, k$)-Noisy-Max is a type of differentially-private selection mechanism where the goal is to determine the counting query with the highest value out of $n$ different counts.  
	The algorithm works as follows. First, generate each of the counts and then add independent Laplace noise from the distribution $Lap(\frac{1}{\epsilon})$ to each of them. The index of the largest noisy count is then reported as the noisy max.
	This has two fold advantage over the naive implementation of finding the maximum count.
Firstly, noisy-max applies "information minimization" as rather than releasing all the noisy counts
and allowing the analyst to find the max and its index, only the
index corresponding to the maximum is made public.
Secondly, the noise added is much smaller than that in the case of the naive implementation (it has sensitivity $\Delta f=m$). Thus the \textsf{NoisyMax} primitive takes as input of encrypted vector where each vector element is a count. It then adds noise drawn from $Lap(\frac{1}{\epsilon})$ to each vector element and computes the indices of the top k elements.
\end{enumerate}
}
