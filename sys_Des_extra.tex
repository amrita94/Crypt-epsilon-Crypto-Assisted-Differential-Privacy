In the centralized DP model, the differentially private analysis is performed on a
trusted, centralized database platform that is presumed to uphold the long-term protection and integrity of both the individuals’
data and the analysis itself. However, this is an extremely strong trust model that entails in a single point of failure -  any platform compromise can have devastating repercussions from the privacy standpoint. At the other end of the trust spectrum, we have local differential privacy where the data owners do not trust any third party server with their private data in the clear. Regrettably, the utility guarantees of locally differentially-private
analyses are extremely limited. Since each reporting individual
performs independent coin flips, even the  best case theoretic error rate  grows in proportion to the square root of
the report count. In this paper we address the challenge of achieving the utility guarantees of the centralized DP model in the weakest trust setting of local DP analysis. 
 We propose an efficient solution in the
two-server model, where none of the parties are trusted party needs to be trusted to handle the data in the
clear. In this setting, the computation of a differentially private program from the merged data of all the private data owners is outsourced to
two non-colluding (but not necessarily trusted) third-parties. After a first phase of collecting
private data in encrypted form from possibly many data-owners, the two third parties then
engage in a second phase for the execution of the program itself. The system is designed in such a way that no extra information (beside that released by the differentially private  result of the executed program) is revealed to
these two parties if they do not collude (this prerequisite can be enforced, for example, by law).
Our solution is based only on a simple cryptographic primitive that can be implemented
via efficient constructions. Indeed, our system is designed using just a linearly-homomorphic
encryption (LHE) scheme, that is, an encryption scheme that enables computing the sum
of encrypted messages. 
 The protocols in this paper are designed
for the following parties\begin{enumerate}
 \item Data-Owners(DO): The data-owners $DO_i, i \in [m]$ are the participating entities each with  a
private dataset $D_i, i \in [m]$ and they are willing to share it only if encrypted. $D_i, i \in [m]$ is a set records ( can be a singleton) of the form $<A_1,...A_l>$ where $A_j$ is an attribute.  \item Analytics Server (AS) - This is the party that wants to run a set of differentially private programs  on the dataset D obtained by merging the local datasets $D_1, ... , D_m$, but has
access only to the encrypted copies of them. 
\item Crypto Service Provider (CSP)-
 The Crypto Service Provider (CSP) takes care of initializing the encryption scheme used
in the system and interacts with AS to help it in achieving its task (computing the
noisy answers). CSP manages the cryptographic keys and is the only entity
capable of decrypting.
\end{enumerate}
\textbf{Cryptographic tools} Our system relies on the usage of linearly homomorphic
encryption (LHE). Let $(\mathcal{M}, +)$ be a finite group. A LHE scheme 
for messages in $\mathcal{M}$ is defined by three algorithms:\begin{enumerate}
\item Key Generation ($Gen$) -The key generation algorithm $Gen$ takes the security parameter $\kappa$ as input and outputs
a matching pair of secret and public keys, $(s_k, p_k) \leftarrow Gen(\kappa)$.
\item Encryption ($Enc$)- This is a randomized algorithm that encrypts a message $m \in \mathcal{M}$ via the public key $p_k$, to generate ciphertext $c \leftarrow Enc_{pk}(m)$.
\item Decryption ($Dec$) - The decryption algorithm $Dec$ is a deterministic function that uses the secret key $s_k$ to
recover the original plaintext from a ciphertext c.
\end{enumerate}
The proposed system works in the following way. Each data-owner $DO_i, i \in [m]$ encrypts her/his data set $D_i$ and sends it to the AS. Recall that the goal of the AS is to execute certain differentially private programs on the merged data. In our setting, the programs are expressed as plans over a library of operators of the types transformation and measurement. The AS hosts a program executor that takes the encrypted merged data (often preprocessed) as input and makes a sequence of operator calls  to transform the data and compute noisy measurements on it. The AS relies on the CSP for the joint computation of certain operators ( all measurement operators and some transformations).  We elucidate each of the steps as follows.  \\
\textbf{Encryption by Data-Owners:} The natural data representation form of each attribute $A_j, j \in [l]$ can either be categorical or real. The data owner $DO_i, i \in [m]$ encrypts each attribute individually as follows:
\begin{itemize}\item The encryption policy for categorical data is very simple, every categorical attribute will be individually encoded by the one-hot coding and the data owner sends the corresponding encryption to the AS. \item  In case of real attributes, let us assume that $\mathcal{M}=Z_N$ for some suitably large integer $N$ and that the  data belongs to the real interval $[-\delta, \delta], \delta > 0$ with at most $t$ digits in their fractional part. Under this assumption, the conversion
from real values to elements in $\mathcal{M}$ can be easily done via appropriate rescaling 
and then mapping the integers in $Z_N$ using the modular operation. The data owners then send the corresponding encryptions of these remapped elements to the AS. \end{itemize} 
\textbf{Program Execution}
In our setting, the differentially private programs are described using
plans composed over a library of operators. For categorical data the inputs undergo a pre-processing transformation known as aggregation before being fed to the program executor. For every attribute $A_j$, this operation extracts the corresponding encrypted one-hot-coding from all the data sets $D_i, i \in [m]$ and sums them up to compute the vector $x_j$. Thus $x_j$ has a cell corresponding to every value of the domain of $A_j$ and each cell denotes the count of the records in the merged data having the corresponding value for the attribute $A_j$.  Aggregation is a 1-stable transformation. Note that in order to perform aggregation on the real valued attributes, the ciphers have to be first undergo an additional transformation called discretization which works as follows. For this the CSP generates a garbled circuit that takes the encryption for the real valued attribute and performs the following actions:\begin{enumerate} 
\item Decrypts the value.
\item Discretizes the value in the plaintext (by some preconceived notion of bucketing)
\item Convert it to the respective one-hot-coding.
\item Encrypts it suitably and outputs the cipher.
\end{enumerate}
The CSP generates and sends this garbled circuit to the AS, which the AS can now use to generate the encrypted one-hot-codings.\\ As mentioned before, the programs in our setting are a sequence of a small number of operators that perform basic functions.
Different
algorithms differ in (a) the sequence in which these operations
are performed on the data, and (b) the specific implementation of
operations from these classes. We illustrate our family of operators next.  \\
\textbf{Operators}
The operators are broadly of two types \begin{enumerate}\item Transformations-Transformation operators take
as input a data source variable (either a table or a vector) and output
a transformed data source (again, either a table or vector). Typically they do not expend privacy budget, though
they can affect the privacy analysis through their stability. Most transformations can be performed by the AS without any interaction with the CSP. These usually involve multiplication of vector $x$ by a publicly known matrix $M$ to generate $x'=Mx$. 
%The linearity of vector transformations is an important feature that is leveraged by downstream inference operators. 
The stability of vector transformations
is equal to the largest $L_1$ column norm of M.  \item Measurement - The measurement operator involves  revealing some information about the private data. Thus they expend the privacy budget and the the magnitude
of the noise has to be calibrated with the noisy measurements. Measurement operators are primarily of two types; one is obtaining noisy counts at different resolutions of the data and the other is computing the noisy max of a set of data values. All measurement operators must involve joint computation with the CSP. Note that the requisite noise to be added to ensure differentially privacy has to be jointly added by both the AS and the CSP. It is so because, had only either one of the servers added the noise, then that server would be able to retrieve the true non-noisy answer by simply de-noising the published differentially private answer. This means that the sensitivity of the program being executed should be known to both the servers. This poses no hindrance in our setting  since the program is public, the  sensitivity computation can be performed very easily by observing the sequence of the preceding transformations. \end{enumerate}
Note that we assume that each data owner $DO_i$ is off-line and is relieved of any duties once it offloads the ciphers for $D_i$  to the AS. If we are willing to relax this restriction and allow for them to be on-line, then some of the transformation operations can be pushed down to the data owners. Instead of sending the encryptions of the raw values, the data owners will now encrypt the transformed inputs thereby saving in some computation costs for the AS.